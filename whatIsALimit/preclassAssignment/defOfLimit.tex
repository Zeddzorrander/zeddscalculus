\documentclass{ximera}
\input{../../preamble.tex}

\outcome{Understand the definition of a limit.}

\author{Kevin James}

\begin{document}
   \begin{exercise}
   	This exercise is designed to introduce us to the interactive tool below.  We will use this tool to investigate $ \lim\limits_{x \to c}f(x)=L $.  It will be part of an activity during our next class.  \\
   	   	
   	First, let's investigate all of the interactive parts.  Move the Right Endpt slider to the right so that it isn't equal to $-4$.  Notice that this moves the point $b$ on the $x$-axis and creates the interval $(a,b)$. The interval is updated on the screen.  If you move the Left Endpt slider, it moves the point $a$ on the $x$-axis.  We can also change these sliders by dragging the points $a$ and $b$ on the $x$-axis.  Dragging the points causes $a$ and $b$ to "snap" to integer values.  We can also change the values by clicking on the point (or slider) to activate it and pressing the left and right arrow keys.  Clicking a point and using arrow keys will cause it to move in increments of 0.1.  Using arrow keys to move the sliders will change $a$ and $b$ in increments of 0.001. If we set Left Endpt (or move $a$) to a value that is greater than (or equal to) that of Right Endpt (or to the right of $b$), then the interval disapears.\\
   	
   	Next, we have the text entry boxes: "$L = $", "$c=$", and "Radius = ".  We can type numbers into all three values.  Changing the values of $L$ and $c$ change the limit that we are investigating and center the window on the point $ (c,L) $.  To see this, simply enter a number into one of these boxes and press Return (or enter) on your keyboard.  Notice, you can set $L$ and $c$ to any values, even if it doesn't make a valid limit.  We can enter any number between 0 and 5 into the Radius box (that is $ 0 \leq \text{Radius} \leq 5 $). \\
   	
   	Finally, we have the Zoom slider.  Moving this slider to the right will zoom the graph in on the point $ (c, L) $.  Moving it to the left will zoom back out.  As with the other sliders, we can also adjust this by clicking on the slider and using the left and right arrow keys.\\
   	
   	The questions below will give us a little more experience using this tool, but first, let's use the graph to estimate some function and limit values.
   	
   	\begin{center}
   		\begin{onlineOnly}
   			\geogebra{ddc9ehwf}{1000}{525}%%https://www.geogebra.org/classic/ddc9ehwf
   		\end{onlineOnly} 
   	\end{center}
   	
   	\begin{question} Use the graph above to estimate the following values.  Enter DNE if the value does not exist.
   		\begin{align*}
   			f(-2) &= \answer{0} \\
   			f(-1) &= \answer{3}  \\
   			f(0) &= \answer{1}  \\
   			f(1) &= \answer{3} \\
   			f(2) &= \answer{DNE} \\
   			f(3) &= \answer{0}
   		\end{align*}
   	\end{question}
   	
   	\begin{question} Use the graph above to estimate the following limit values.  Enter DNE if the limit doesn't exist.
   		\begin{align*}
   			\lim\limits_{x \to -1}f(x) &= \answer{3} \\
   			\lim\limits_{x \to 0}f(x) &= \answer{4} \\
   			\lim\limits_{x \to 1}f(x) &= \answer{DNE} \\
   			\lim\limits_{x \to 2}f(x) &= \answer{1} \\
   			\lim\limits_{x \to 3}f(x) &= \answer{0} 
   		\end{align*}
   	\end{question}
   	
   	The following question will help us understand the interactive tool a little better.  The same tool from above is given below to prevent a need to scroll back and forth.
   	
   	\begin{center}
   		\begin{onlineOnly}
   			\geogebra{ddc9ehwf}{1000}{525}%%https://www.geogebra.org/classic/ddc9ehwf
   		\end{onlineOnly} 
   	\end{center}
   	
    \begin{question}
   	  Change $ c $ to $-2$.  Notice, this updates the limit we are investigating to $\lim\limits_{x \to -2}f(x)=0$.  In order for this limit to be correct, we need to change the value for $ L $ to be: 
   	  \[
   	      L=\answer{0}
   	  \]
   	  \begin{feedback}
   	  	Notice, changing the value of $ c $ changes the value $x$ is approaching in the limit, and changing the value of $L$ changes the value of the limit.  We can set these to be anything we want, even values that are not correct.  For example, leave $ c=-2 $, but change $ L $ to 4.  The limit updates, even though this is not the correct value for the limit.
   	  \end{feedback}
      \begin{question}
      	Set c = $2$, L = $1$, and Radius = $1$.  Notice, this creates the interval $\left(\answer{0}, \answer{2}\right)$ on the y-axis.
      	\begin{feedback}
      		Notice, the interval $(0,2)$ is centered at $L=1$ with a Radius of 1.  The value of $ L $ sets the center of the interval on the $y$-axis, and the value of Radius determines its radius.  Recall, this represents all the points on the $y$-axis that are within $1$ unit of $1$.  More generally, the value of $ L $ and Radius work together to create an interval on the $y$-axis that is centered at $ L $ with the desired radius.
      	\end{feedback}
        \begin{question}
        	Leave $c=2$, $L = 1$, and Radius = $1$.  Now, drag the point $b$ on the $x$-axis (or set Right Endpt slider) until you get $b=3$.  Then, drag the point $a$ on the $x$-axis untl you get to $a=1$.  Notice, this forms the interval $(1,3)$ on the $x$-axis. \\
        	This interval is centered at $x=\answer{2}$ with a radius of $\answer{1}$.
        	\begin{feedback}
        		This interval represents all the $x$-values that are within 1 unit of $x=2$.
        	\end{feedback}
        	\begin{question}
        		Click on the point $a$ to activate it.  Then press the right arrow key on your keyboard one time.  Notice, this causes $a$ to move from $1$ to $\answer{1.1}$.\\
        		Click on the point $b$ to activate it.  Then press the left arrow key on your keyboard one time.  Notice, this causes $b$ to move from $3$ to $\answer{2.9}$.\\
        		Notice, the portion of the graph on the interval $(a,b)$ has \wordChoice{\choice{done nothing},\choice[correct]{turned green}}.
        		\begin{feedback}
        			Recall, once you select a point, you can move it in increments of 0.1 by pressing the left and right arrow keys on your keyboard.\\
        			Recall, if you select either the Left Endpt or Right Endpt slider, and use the arrow keys, you will move the slider (and hence $a$ or $b$) in increments of 0.001.  The values of $a$ and $b$ are set to round to the nearest thousandth.  
        		\end{feedback}
        	\end{question}
        
        \end{question}
      \end{question}
   \end{question}
	\begin{question}
		Set $ c=1, L=0 $ and Radius = 0.  Then set $ a=-1 $ and $ b = 3 $.  Now, move the Left Endpt slider (or point $a$) to the right.  What happened? \wordChoice{\choice{Nothing},\choice[correct]{A purple interval appears on the $y$-axis.}}
		\begin{feedback}
			Notice, $f$ maps the interval $ (a, 1) $ onto this purple interval.\\
			If we move $b$ so that it is between 1 and 3, we will see a similar purple interval that shows what $f$ maps $ (1,b) $ onto.
		\end{feedback}
	\end{question}
   \begin{question}
   	 Play around with this tool.  Use the graph to estimate various limits (that is, change $c$ and $L$ values to estimate limits on this graph).  For each limit you investigate, change the value of Radius and move the Left Endpt and Right Endpt sliders to see if you can make the graph turn green.  When you are ready, return to Homework 1 and move on to Problem 2.
   \end{question}
 \end{exercise}
\end{document}
