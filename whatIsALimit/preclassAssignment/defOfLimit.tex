\documentclass{ximera}
\input{../../preamble.tex}

\outcome{}
\outcome{}

\author{Kevin James}

\begin{document}
   \begin{exercise}
   	These exercises are designed to introduce us to the interactive tool below.  We will use this tool to investigate $ \lim\limits_{x \to c}f(x)=L $.  We will use this tool as part of an in-class activity.
   	\begin{center}
   	   \begin{onlineOnly}
   		   \geogebra{npwjtrqd}{800}{600}%%https://www.geogebra.org/classic/npwjtrqd
   	   \end{onlineOnly} 
   	\end{center}
    \begin{question}
   	  Change c to $-2$.  Notice, this updates the limit we are investigating to $\lim\limits_{x \to -2}f(x)=4$.  In order for this limit to be correct, we need to change the value for L to be: 
   	  \[
   	      L=\answer{0}
   	  \]
   	  \begin{feedback}
   	  	Notice, changing the value of c changes the value $x$ is approaching in the limit, and changing the value of $L$ changes the value of the limit.  We can set these to be anything we want, even values that are not correct.  For example, leave c=-2, but change L to 4.  The limit updates, even though this is not the correct value for the limit.
   	  \end{feedback}
      \begin{question}
      	Set c = $2$, L = $1$, and Radius = $1$.  Notice, this creates the interval $\left(\answer{0}, \answer{2}\right)$ on the y-axis.
      	\begin{feedback}
      		Notice, the interval $(0,2)$ is centered at $L=1$ with a Radius of 1.  The value of L sets the center of the interval on the $y$-axis, and the value of Radius determines its radius.  Recall, this represents all the points on the $y$-axis that are within $1$ unit of $1$.  More generally, the value of L and Radius work together to create an interval on the $y$-axis that is centered at L with the desired radius.
      	\end{feedback}
        \begin{question}
        	Leave c = $2$, L = $1$, and Radius = $1$.  Now, drag the point $b$ on the $x$-axis until you get $b=3$.  Then, drag the point $a$ on the $x$-axis untl you get to $a=1$.  Notice, this forms the interval $(1,3)$ on the $x$-axis. \\
        	This interval is centered at $x=\answer{2}$ with a radius of $\answer{1}$.
        	\begin{feedback}
        		This interval represents all the $x$-values that are within 1 unit of $x=2$.
        	\end{feedback}
        	\begin{question}
        		Click on the point $a$ to activate it.  Then press the right arrow key on your keyboard one time.  Notice, this causes $a$ to move from $1$ to $\answer{1.1}$.\\
        		Click on the point $b$ to activate it.  Then press the left arrow key on your keyboard one time.  Notice, this causes $b$ to move from $3$ to $\answer{2.9}$.\\
        		Notice, the portion of the graph on the interval $(a,b)$ has \wordChoice{\choice{done nothing},\choice[correct]{turned green}}.
        		\begin{feedback}
        			Once you select a point, you can move it in increments of 0.1 by pressing the left and right arrow keys on your keyboard.
        		\end{feedback}
        	\end{question}
        
        \end{question}
      \end{question}
   \end{question}
   \begin{question}
   	 
   \end{question}
 \end{exercise}
\end{document}
