\documentclass{ximera}
\input{../../preamble.tex}

\outcome{Describe the image of a function}
\outcome{Produce open intervals around domain number that adheres to given requirements}

\author{Kevin James}

\begin{document}
   \begin{exercise}
   	Use the application below to answer the questions that follow.  To set the restricted domain, click on point $ A $ or point $ B $ and move the selected point using the left and right arrow keys on your keyboard.  Alternatively, you can drag the point $ A $ to the desired left endpoint and drag the point $ B $ to the desired right endpoint.  Repeat this process with $ C $ and $ D $ to set an interval on the $y$-axis.  The arrow keys will move each point by 0.1 units.  Dragging will allow you to get more accurate positions.  
    
    \begin{center}
    	\begin{onlineOnly}
    		\geogebra{mYxPvqeJ}{800}{600}%%https://www.geogebra.org/classic/mYxPvqeJ
    	\end{onlineOnly} 
    \end{center}
    
    In this exercise, we will be thinking about intervals in a new way.  We will define an interval by picking a point to be its center and then defining a radius.  For example, the interval centered at $1$ with a radius of $2$ is $(-1, 3)$.  This interval represents all values that are within $2$ units of the point $1$ on the number line.
    
    Defining an interval this way on the $y$-axis will allow us to explore "closeness" in the context of limits.  This exercise will prepare us for an in-class activity.
    
    \begin{question}
    	What interval is centered at $-2$ with a radius of $ 0.5$? $\left( \answer{-2.5}, \answer{-1.5}\right)$
    	\begin{feedback}
    		The interval $(-2.5, -1.5)$ represents all points that are within $0.5$ units of $-2$.
    	\end{feedback}
    \end{question}
    \begin{question}
    	What interval represents all points that are within $0.01$ units of $1$?  $\left( \answer{0.99}, \answer{1.01} \right)$
    \end{question}
 	\begin{question}
 		What interval represents all points that are within $0.5$ units of $1$?  $\left( \answer{0.5}, \answer{1.5}\right)$
		\begin{question}
 			Set $C$ at $0.5$ and $D$ at $1.5$ to form the interval $(0.5, 1.5)$ on the $y$-axis.  Which interval(s) on the $x$-axis map to this interval? Select all that apply.
 			\begin{selectAll}
 				\choice{$(-1, 1)$}
 				\choice[correct]{$(-0.5, 0.6)$}
 				\choice[correct]{$(-3.75, -3.1)$}
 				\choice[correct]{$(5.23, 5.65)$}
 			\end{selectAll}
 		    \begin{feedback}
 		    	Notice: From the graph, we can estimate that $\lim\limits_{x \to 0}f(x)=1$ and this exercise shows that $f$ maps $x$-values in $(-0.5, 0.6)$ to $y$-values in $(0.5, 1.5)$.  In fact, you can use the application to see that $f$ maps $x$-values within $0.5$ units of $a=0$ to $y$-values that are within at least $0.5$ units of $L=1$.
 		    \end{feedback}
  		\end{question}	
 	\end{question}
 	\begin{question}
 		Given that $\lim\limits_{x \to 6}f(x)=2.5$, which of the following intervals represent an interval centered at $a=6$ such that $f$ maps any $x$-value in that interval to a $y$-value that is within $0.5$ units of $L=2.5$?  Select all that apply.
 		\begin{selectAll}
 			\choice{$(-2.2, -1.8)$}
 			\choice[correct]{$(5.9, 6.1)$}
 			\choice{$(5.84, 6.14)$}
 			\choice[correct]{$(5.95, 6.05)$}
 		\end{selectAll}
 		\begin{feedback}
 			Notice, any $x$-value in either $(-2.2, -1.8)$ or $ (5.84, 6.14) $ gets mapped to a $y$-values that are within $0.5$ units of $2.5$, but neither interval is centered at $a=6$.
 			
 			Also, notice that once we find an interval that works, like $ (5.9, 6.1)$, any interval that is centered at $a=6$, but has a smaller radius will also work.  If the tool allowed us to zoom in on the graph, we could verify this by seeing that $f$ maps any $x$-value in the interval $ (5.99, 6.01) $ to $y$-values that are within $0.5$ units of $L=2.5$.
 		\end{feedback}
 		\begin{question}
 			Which endpoint of $ (5.84, 6.14) $ needs to be changed so that the new interval satisfies the criteria in the original problem?  \wordChoice{\choice[correct]{5.84}\choice{6.14}}
 			\begin{question}
 				If we replace $5.84$ with $\answer{5.86}$, then the resulting interval will satisfy the criteria in the original problem. 
				\begin{feedback}
					$f$ maps $x$-values in the interval $ (5.84, 6.14) $ to $y$-values that are within 0.5 units, but it is NOT centered at $6$.  Since all $x$-values within this interval get mapped to $y$-values that are within $0.5$ units of $L=2.5$, we need to find a smaller interval that is contained in this interval, but that IS centered at $6$.  Since $5.84$ is $0.16$ units away from $6$, and $6.14$ is $0.14$ units, we need to pick the smaller value for the radius of our interval.  That is, we need an interval that is centered at $6$ with a radius of $0.14$.  
				\end{feedback}
			\end{question}
 		\end{question}
 	\end{question}
    \end{exercise}
\end{document}
