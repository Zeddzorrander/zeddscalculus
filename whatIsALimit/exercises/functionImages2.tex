\documentclass{ximera}
%\usepackage{todonotes}
%\usepackage{mathtools} %% Required for wide table Curl and Greens
%\usepackage{cuted} %% Required for wide table Curl and Greens
\newcommand{\todo}{}

\usepackage{esint} % for \oiint
\ifxake%%https://math.meta.stackexchange.com/questions/9973/how-do-you-render-a-closed-surface-double-integral
\renewcommand{\oiint}{{\large\bigcirc}\kern-1.56em\iint}
\fi


\graphicspath{
  {./}
  {ximeraTutorial/}
  {basicPhilosophy/}
  {functionsOfSeveralVariables/}
  {normalVectors/}
  {lagrangeMultipliers/}
  {vectorFields/}
  {greensTheorem/}
  {shapeOfThingsToCome/}
  {dotProducts/}
  {partialDerivativesAndTheGradientVector/}
  {../productAndQuotientRules/exercises/}
  {../motionAndPathsInSpace/exercises/}
  {../normalVectors/exercisesParametricPlots/}
  {../continuityOfFunctionsOfSeveralVariables/exercises/}
  {../partialDerivativesAndTheGradientVector/exercises/}
  {../directionalDerivativeAndChainRule/exercises/}
  {../commonCoordinates/exercisesCylindricalCoordinates/}
  {../commonCoordinates/exercisesSphericalCoordinates/}
  {../greensTheorem/exercisesCurlAndLineIntegrals/}
  {../greensTheorem/exercisesDivergenceAndLineIntegrals/}
  {../shapeOfThingsToCome/exercisesDivergenceTheorem/}
  {../greensTheorem/}
  {../shapeOfThingsToCome/}
  {../separableDifferentialEquations/exercises/}
  {vectorFields/}
}

\newcommand{\mooculus}{\textsf{\textbf{MOOC}\textnormal{\textsf{ULUS}}}}

\usepackage{tkz-euclide}\usepackage{tikz}
\usepackage{tikz-cd}
\usetikzlibrary{arrows}
\tikzset{>=stealth,commutative diagrams/.cd,
  arrow style=tikz,diagrams={>=stealth}} %% cool arrow head
\tikzset{shorten <>/.style={ shorten >=#1, shorten <=#1 } } %% allows shorter vectors

\usetikzlibrary{backgrounds} %% for boxes around graphs
\usetikzlibrary{shapes,positioning}  %% Clouds and stars
\usetikzlibrary{matrix} %% for matrix
\usepgfplotslibrary{polar} %% for polar plots
\usepgfplotslibrary{fillbetween} %% to shade area between curves in TikZ
\usetkzobj{all}
\usepackage[makeroom]{cancel} %% for strike outs
%\usepackage{mathtools} %% for pretty underbrace % Breaks Ximera
%\usepackage{multicol}
\usepackage{pgffor} %% required for integral for loops



%% http://tex.stackexchange.com/questions/66490/drawing-a-tikz-arc-specifying-the-center
%% Draws beach ball
\tikzset{pics/carc/.style args={#1:#2:#3}{code={\draw[pic actions] (#1:#3) arc(#1:#2:#3);}}}



\usepackage{array}
\setlength{\extrarowheight}{+.1cm}
\newdimen\digitwidth
\settowidth\digitwidth{9}
\def\divrule#1#2{
\noalign{\moveright#1\digitwidth
\vbox{\hrule width#2\digitwidth}}}





\newcommand{\RR}{\mathbb R}
\newcommand{\R}{\mathbb R}
\newcommand{\N}{\mathbb N}
\newcommand{\Z}{\mathbb Z}

\newcommand{\sagemath}{\textsf{SageMath}}


%\renewcommand{\d}{\,d\!}
\renewcommand{\d}{\mathop{}\!d}
\newcommand{\dd}[2][]{\frac{\d #1}{\d #2}}
\newcommand{\pp}[2][]{\frac{\partial #1}{\partial #2}}
\renewcommand{\l}{\ell}
\newcommand{\ddx}{\frac{d}{\d x}}

\newcommand{\zeroOverZero}{\ensuremath{\boldsymbol{\tfrac{0}{0}}}}
\newcommand{\inftyOverInfty}{\ensuremath{\boldsymbol{\tfrac{\infty}{\infty}}}}
\newcommand{\zeroOverInfty}{\ensuremath{\boldsymbol{\tfrac{0}{\infty}}}}
\newcommand{\zeroTimesInfty}{\ensuremath{\small\boldsymbol{0\cdot \infty}}}
\newcommand{\inftyMinusInfty}{\ensuremath{\small\boldsymbol{\infty - \infty}}}
\newcommand{\oneToInfty}{\ensuremath{\boldsymbol{1^\infty}}}
\newcommand{\zeroToZero}{\ensuremath{\boldsymbol{0^0}}}
\newcommand{\inftyToZero}{\ensuremath{\boldsymbol{\infty^0}}}



\newcommand{\numOverZero}{\ensuremath{\boldsymbol{\tfrac{\#}{0}}}}
\newcommand{\dfn}{\textbf}
%\newcommand{\unit}{\,\mathrm}
\newcommand{\unit}{\mathop{}\!\mathrm}
\newcommand{\eval}[1]{\bigg[ #1 \bigg]}
\newcommand{\seq}[1]{\left( #1 \right)}
\renewcommand{\epsilon}{\varepsilon}
\renewcommand{\phi}{\varphi}


\renewcommand{\iff}{\Leftrightarrow}

\DeclareMathOperator{\arccot}{arccot}
\DeclareMathOperator{\arcsec}{arcsec}
\DeclareMathOperator{\arccsc}{arccsc}
\DeclareMathOperator{\si}{Si}
\DeclareMathOperator{\scal}{scal}
\DeclareMathOperator{\sign}{sign}


%% \newcommand{\tightoverset}[2]{% for arrow vec
%%   \mathop{#2}\limits^{\vbox to -.5ex{\kern-0.75ex\hbox{$#1$}\vss}}}
\newcommand{\arrowvec}[1]{{\overset{\rightharpoonup}{#1}}}
%\renewcommand{\vec}[1]{\arrowvec{\mathbf{#1}}}
\renewcommand{\vec}[1]{{\overset{\boldsymbol{\rightharpoonup}}{\mathbf{#1}}}\hspace{0in}}

\newcommand{\point}[1]{\left(#1\right)} %this allows \vector{ to be changed to \vector{ with a quick find and replace
\newcommand{\pt}[1]{\mathbf{#1}} %this allows \vec{ to be changed to \vec{ with a quick find and replace
\newcommand{\Lim}[2]{\lim_{\point{#1} \to \point{#2}}} %Bart, I changed this to point since I want to use it.  It runs through both of the exercise and exerciseE files in limits section, which is why it was in each document to start with.

\DeclareMathOperator{\proj}{\mathbf{proj}}
\newcommand{\veci}{{\boldsymbol{\hat{\imath}}}}
\newcommand{\vecj}{{\boldsymbol{\hat{\jmath}}}}
\newcommand{\veck}{{\boldsymbol{\hat{k}}}}
\newcommand{\vecl}{\vec{\boldsymbol{\l}}}
\newcommand{\uvec}[1]{\mathbf{\hat{#1}}}
\newcommand{\utan}{\mathbf{\hat{t}}}
\newcommand{\unormal}{\mathbf{\hat{n}}}
\newcommand{\ubinormal}{\mathbf{\hat{b}}}

\newcommand{\dotp}{\bullet}
\newcommand{\cross}{\boldsymbol\times}
\newcommand{\grad}{\boldsymbol\nabla}
\newcommand{\divergence}{\grad\dotp}
\newcommand{\curl}{\grad\cross}
%\DeclareMathOperator{\divergence}{divergence}
%\DeclareMathOperator{\curl}[1]{\grad\cross #1}
\newcommand{\lto}{\mathop{\longrightarrow\,}\limits}

\renewcommand{\bar}{\overline}

\colorlet{textColor}{black}
\colorlet{background}{white}
\colorlet{penColor}{blue!50!black} % Color of a curve in a plot
\colorlet{penColor2}{red!50!black}% Color of a curve in a plot
\colorlet{penColor3}{red!50!blue} % Color of a curve in a plot
\colorlet{penColor4}{green!50!black} % Color of a curve in a plot
\colorlet{penColor5}{orange!80!black} % Color of a curve in a plot
\colorlet{penColor6}{yellow!70!black} % Color of a curve in a plot
\colorlet{fill1}{penColor!20} % Color of fill in a plot
\colorlet{fill2}{penColor2!20} % Color of fill in a plot
\colorlet{fillp}{fill1} % Color of positive area
\colorlet{filln}{penColor2!20} % Color of negative area
\colorlet{fill3}{penColor3!20} % Fill
\colorlet{fill4}{penColor4!20} % Fill
\colorlet{fill5}{penColor5!20} % Fill
\colorlet{gridColor}{gray!50} % Color of grid in a plot

\newcommand{\surfaceColor}{violet}
\newcommand{\surfaceColorTwo}{redyellow}
\newcommand{\sliceColor}{greenyellow}




\pgfmathdeclarefunction{gauss}{2}{% gives gaussian
  \pgfmathparse{1/(#2*sqrt(2*pi))*exp(-((x-#1)^2)/(2*#2^2))}%
}


%%%%%%%%%%%%%
%% Vectors
%%%%%%%%%%%%%

%% Simple horiz vectors
\renewcommand{\vector}[1]{\left\langle #1\right\rangle}


%% %% Complex Horiz Vectors with angle brackets
%% \makeatletter
%% \renewcommand{\vector}[2][ , ]{\left\langle%
%%   \def\nextitem{\def\nextitem{#1}}%
%%   \@for \el:=#2\do{\nextitem\el}\right\rangle%
%% }
%% \makeatother

%% %% Vertical Vectors
%% \def\vector#1{\begin{bmatrix}\vecListA#1,,\end{bmatrix}}
%% \def\vecListA#1,{\if,#1,\else #1\cr \expandafter \vecListA \fi}

%%%%%%%%%%%%%
%% End of vectors
%%%%%%%%%%%%%

%\newcommand{\fullwidth}{}
%\newcommand{\normalwidth}{}



%% makes a snazzy t-chart for evaluating functions
%\newenvironment{tchart}{\rowcolors{2}{}{background!90!textColor}\array}{\endarray}

%%This is to help with formatting on future title pages.
\newenvironment{sectionOutcomes}{}{}



%% Flowchart stuff
%\tikzstyle{startstop} = [rectangle, rounded corners, minimum width=3cm, minimum height=1cm,text centered, draw=black]
%\tikzstyle{question} = [rectangle, minimum width=3cm, minimum height=1cm, text centered, draw=black]
%\tikzstyle{decision} = [trapezium, trapezium left angle=70, trapezium right angle=110, minimum width=3cm, minimum height=1cm, text centered, draw=black]
%\tikzstyle{question} = [rectangle, rounded corners, minimum width=3cm, minimum height=1cm,text centered, draw=black]
%\tikzstyle{process} = [rectangle, minimum width=3cm, minimum height=1cm, text centered, draw=black]
%\tikzstyle{decision} = [trapezium, trapezium left angle=70, trapezium right angle=110, minimum width=3cm, minimum height=1cm, text centered, draw=black]


\outcome{Describe the image of a function}
\outcome{}

\author{Kevin James}

\begin{document}
   In this activity, we will be thinking about intervals in a new way.  We will define an interval by picking a point to be its center and then defining a radius.  For example, the interval centered at $1$ with a radius of $2$ is $(-1, 3)$.  This interval represents all values that are within $2$ units of the point $1$ on the number line.
   
   If we define an interval this way 
   \begin{exercise}
   	Use the application below to answer the questions that follow.  To set the restricted domain, click on point $ A $ or point $ B $ and move the selected point using the left and right arrow keys on your keyboard.  Alternatively, you can drag the point $ A $ to the desired left endpoint and drag the point $ B $ to the desired right endpoint.  Repeat this process with $ C $ and $ D $ to set an interval on the $y$-axis.  The arrow keys will move each point by 0.1 units.  Dragging will allow you to get more accurate positions.  
    
    \begin{center}
    	\begin{onlineOnly}
    		\geogebra{mYxPvqeJ}{800}{600}%%https://www.geogebra.org/classic/mYxPvqeJ
    	\end{onlineOnly} 
    \end{center}
    
    In this exercise, we will be thinking about intervals in a new way.  We will define an interval by picking a point to be its center and then defining a radius.  For example, the interval centered at $1$ with a radius of $2$ is $(-1, 3)$.  This interval represents all values that are within $2$ units of the point $1$ on the number line.
    
    Defining an interval this way on the $y$-axis will allow us to explore "closeness" in the context of limits.  This exercise will prepare us for an in-class activity.
    
    \begin{question}
    	What interval is centered at $-2$ with a radius of $ 0.5$? $\left( \answer{-2.5}, \answer{-1.5}\right)$
    	\begin{feedback}
    		The interval $(-2.5, -1.5)$ represents all points that are within $0.5$ units of $-2$.
    	\end{feedback}
    \end{question}
    \begin{question}
    	What interval represents all points that are within $0.01$ units of $1$?  $\left( \answer{0.99}, \answer{1.01} \right)$
    \end{question}
 	\begin{question}
 		What interval represents all points that are within $0.5$ units of $1$?  $\left( \answer{0.5}, \answer{1.5}\right)$
		\begin{question}
 			Set $C$ at $0.5$ and $D$ at $1.5$ to form the interval $(0.5, 1.5)$ on the $y$-axis.  Which interval(s) on the $x$-axis map to this interval? Select all that apply.
 			\begin{selectAll}
 				\choice{$(-1, 1)$}
 				\choice[correct]{$(-0.5, 0.6)$}
 				\choice[correct]{$(-3.75, -3.1)$}
 				\choice[correct]{$(5.23, 5.65)$}
 			\end{selectAll}
 		    \begin{feedback}
 		    	Notice: From the graph, we can estimate that $\lim\limits_{x \to 0}f(x)=1$ and this exercise shows that $f$ maps $x$-values in $(-0.5, 0.6)$ to $y$-values in $(0.5, 1.5)$.  In fact, you can use the application to see that $f$ maps $x$-values within $0.5$ units of $a=0$ to $y$-values that are within at least $0.5$ units of $L=1$.
 		    \end{feedback}
  		\end{question}	
 	\end{question}
 	\begin{question}
 		Given that $\lim\limits_{x \to 6}f(x)=2.5$, which of the following intervals represent an interval centered at $a=6$ such that $f$ maps any $x$-value in that interval to a $y$-value that is within $0.5$ units of $L=2.5$?  Select all that apply.
 		\begin{selectAll}
 			\choice{$(-2.2, -1.8)$}
 			\choice[correct]{$(5.9, 6.1)$}
 			\choice{$(5.84, 6.14)$}
 			\choice[correct]{$(5.95, 6.05)$}
 		\end{selectAll}
 		\begin{feedback}
 			Notice, any $x$-value in either $(-2.2, -1.8)$ or $ (5.84, 6.14) $ gets mapped to a $y$-values that are within $0.5$ units of $2.5$, but neither interval is centered at $a=6$.
 			
 			Also, notice that once we find an interval that works, like $ (5.9, 6.1)$, any interval that is centered at $a=6$, but has a smaller radius will also work.  If the tool allowed us to zoom in on the graph, we could verify this by seeing that $f$ maps any $x$-value in the interval $ (5.99, 6.01) $ to $y$-values that are within $0.5$ units of $L=2.5$.
 		\end{feedback}
 		\begin{question}
 			How could $ (5.84, 6.14) $ be modified so that it satisfies the criteria in the original problem?
			\[
			   \left(\answer{5.86}, \answer{6.14}\right)
			\]
			\begin{feedback}
				$f$ maps $x$-values in the interval $ (5.84, 6.14) $ to $y$-values that are within 0.5 units, but it is NOT centered at $6$.  Since all $x$-values within this interval get mapped to $y$-values that are within $0.5$ units of $L=2.5$, we need to find a smaller interval that is contained in this interval, but that IS centered at $6$.  Since $5.84$ is $0.16$ units away from $6$, and $6.14$ is $0.14$ units, we need to pick the smaller value for the radius of our interval.  That is, we need an interval that is centered at $6$ with a radius of $0.14$.  
			\end{feedback}
 		\end{question}
 	\end{question}
    \end{exercise}
\end{document}
