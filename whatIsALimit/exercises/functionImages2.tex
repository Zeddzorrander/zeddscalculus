\documentclass{ximera}
\input{../../preamble.tex}

\outcome{Describe the image of a function}
\outcome{}

\author{Kevin James}

\begin{document}
   In this activity, we will be thinking about intervals in a new way.  We will define an interval by picking a point to be its center and then defining a radius.  For example, the interval centered at $1$ with a radius of $2$ is $(-1, 3)$.  This interval represents all values that are within $2$ units of the point $1$ on the number line.
   
   If we define an interval this way 
   \begin{exercise}
   	Use the application below to answer the questions that follow.  To set the restricted domain, click on point $ A $ or point $ B $ and move the selected point using the left and right arrow keys on your keyboard.  Alternatively, you can drag the point $ A $ to the desired left endpoint and drag the point $ B $ to the desired right endpoint.  Repeat this process with $ C $ and $ D $ to set an interval on the $y$-axis.  As the Restricted Domain is changed, the "Maps to Range" interval will automatically update.  The Restricted Domain interval is represented on the graph in blue and the subinterval of the range that $ f $ maps the restricted domain to is represented in red.
    
    \begin{center}
    	\begin{onlineOnly}
    		\geogebra{mYxPvqeJ}{800}{600}%%https://www.geogebra.org/classic/mYxPvqeJ
    	\end{onlineOnly} 
    \end{center}
    
    In this exercise, we will be thinking about intervals in a new way.  We will define an interval by picking a point to be its center and then defining a radius.  For example, the interval centered at $1$ with a radius of $2$ is $(-1, 3)$.  This interval represents all values that are within $2$ units of the point $1$ on the number line.
    
    Defining an interval this way on the $y$-axis will allow us to explore "closeness" in the context of limits.  This exercise will prepare us for an in-class activity.
    
    \begin{enumerate}
    	\item What interval is centered at $-2$ with a radius of $ \frac{1}{2}$? $\left( \answer{-2.5}, \answer{-1.5}\right)$
    	\item 
    \end{enumerate}
    \end{exercise}
\end{document}
