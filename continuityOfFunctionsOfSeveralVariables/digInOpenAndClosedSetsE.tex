\documentclass{ximera}

%\usepackage{todonotes}
%\usepackage{mathtools} %% Required for wide table Curl and Greens
%\usepackage{cuted} %% Required for wide table Curl and Greens
\newcommand{\todo}{}

\usepackage{esint} % for \oiint
\ifxake%%https://math.meta.stackexchange.com/questions/9973/how-do-you-render-a-closed-surface-double-integral
\renewcommand{\oiint}{{\large\bigcirc}\kern-1.56em\iint}
\fi


\graphicspath{
  {./}
  {ximeraTutorial/}
  {basicPhilosophy/}
  {functionsOfSeveralVariables/}
  {normalVectors/}
  {lagrangeMultipliers/}
  {vectorFields/}
  {greensTheorem/}
  {shapeOfThingsToCome/}
  {dotProducts/}
  {partialDerivativesAndTheGradientVector/}
  {../productAndQuotientRules/exercises/}
  {../motionAndPathsInSpace/exercises/}
  {../normalVectors/exercisesParametricPlots/}
  {../continuityOfFunctionsOfSeveralVariables/exercises/}
  {../partialDerivativesAndTheGradientVector/exercises/}
  {../directionalDerivativeAndChainRule/exercises/}
  {../commonCoordinates/exercisesCylindricalCoordinates/}
  {../commonCoordinates/exercisesSphericalCoordinates/}
  {../greensTheorem/exercisesCurlAndLineIntegrals/}
  {../greensTheorem/exercisesDivergenceAndLineIntegrals/}
  {../shapeOfThingsToCome/exercisesDivergenceTheorem/}
  {../greensTheorem/}
  {../shapeOfThingsToCome/}
  {../separableDifferentialEquations/exercises/}
  {vectorFields/}
}

\newcommand{\mooculus}{\textsf{\textbf{MOOC}\textnormal{\textsf{ULUS}}}}

\usepackage{tkz-euclide}\usepackage{tikz}
\usepackage{tikz-cd}
\usetikzlibrary{arrows}
\tikzset{>=stealth,commutative diagrams/.cd,
  arrow style=tikz,diagrams={>=stealth}} %% cool arrow head
\tikzset{shorten <>/.style={ shorten >=#1, shorten <=#1 } } %% allows shorter vectors

\usetikzlibrary{backgrounds} %% for boxes around graphs
\usetikzlibrary{shapes,positioning}  %% Clouds and stars
\usetikzlibrary{matrix} %% for matrix
\usepgfplotslibrary{polar} %% for polar plots
\usepgfplotslibrary{fillbetween} %% to shade area between curves in TikZ
\usetkzobj{all}
\usepackage[makeroom]{cancel} %% for strike outs
%\usepackage{mathtools} %% for pretty underbrace % Breaks Ximera
%\usepackage{multicol}
\usepackage{pgffor} %% required for integral for loops



%% http://tex.stackexchange.com/questions/66490/drawing-a-tikz-arc-specifying-the-center
%% Draws beach ball
\tikzset{pics/carc/.style args={#1:#2:#3}{code={\draw[pic actions] (#1:#3) arc(#1:#2:#3);}}}



\usepackage{array}
\setlength{\extrarowheight}{+.1cm}
\newdimen\digitwidth
\settowidth\digitwidth{9}
\def\divrule#1#2{
\noalign{\moveright#1\digitwidth
\vbox{\hrule width#2\digitwidth}}}





\newcommand{\RR}{\mathbb R}
\newcommand{\R}{\mathbb R}
\newcommand{\N}{\mathbb N}
\newcommand{\Z}{\mathbb Z}

\newcommand{\sagemath}{\textsf{SageMath}}


%\renewcommand{\d}{\,d\!}
\renewcommand{\d}{\mathop{}\!d}
\newcommand{\dd}[2][]{\frac{\d #1}{\d #2}}
\newcommand{\pp}[2][]{\frac{\partial #1}{\partial #2}}
\renewcommand{\l}{\ell}
\newcommand{\ddx}{\frac{d}{\d x}}

\newcommand{\zeroOverZero}{\ensuremath{\boldsymbol{\tfrac{0}{0}}}}
\newcommand{\inftyOverInfty}{\ensuremath{\boldsymbol{\tfrac{\infty}{\infty}}}}
\newcommand{\zeroOverInfty}{\ensuremath{\boldsymbol{\tfrac{0}{\infty}}}}
\newcommand{\zeroTimesInfty}{\ensuremath{\small\boldsymbol{0\cdot \infty}}}
\newcommand{\inftyMinusInfty}{\ensuremath{\small\boldsymbol{\infty - \infty}}}
\newcommand{\oneToInfty}{\ensuremath{\boldsymbol{1^\infty}}}
\newcommand{\zeroToZero}{\ensuremath{\boldsymbol{0^0}}}
\newcommand{\inftyToZero}{\ensuremath{\boldsymbol{\infty^0}}}



\newcommand{\numOverZero}{\ensuremath{\boldsymbol{\tfrac{\#}{0}}}}
\newcommand{\dfn}{\textbf}
%\newcommand{\unit}{\,\mathrm}
\newcommand{\unit}{\mathop{}\!\mathrm}
\newcommand{\eval}[1]{\bigg[ #1 \bigg]}
\newcommand{\seq}[1]{\left( #1 \right)}
\renewcommand{\epsilon}{\varepsilon}
\renewcommand{\phi}{\varphi}


\renewcommand{\iff}{\Leftrightarrow}

\DeclareMathOperator{\arccot}{arccot}
\DeclareMathOperator{\arcsec}{arcsec}
\DeclareMathOperator{\arccsc}{arccsc}
\DeclareMathOperator{\si}{Si}
\DeclareMathOperator{\scal}{scal}
\DeclareMathOperator{\sign}{sign}


%% \newcommand{\tightoverset}[2]{% for arrow vec
%%   \mathop{#2}\limits^{\vbox to -.5ex{\kern-0.75ex\hbox{$#1$}\vss}}}
\newcommand{\arrowvec}[1]{{\overset{\rightharpoonup}{#1}}}
%\renewcommand{\vec}[1]{\arrowvec{\mathbf{#1}}}
\renewcommand{\vec}[1]{{\overset{\boldsymbol{\rightharpoonup}}{\mathbf{#1}}}\hspace{0in}}

\newcommand{\point}[1]{\left(#1\right)} %this allows \vector{ to be changed to \vector{ with a quick find and replace
\newcommand{\pt}[1]{\mathbf{#1}} %this allows \vec{ to be changed to \vec{ with a quick find and replace
\newcommand{\Lim}[2]{\lim_{\point{#1} \to \point{#2}}} %Bart, I changed this to point since I want to use it.  It runs through both of the exercise and exerciseE files in limits section, which is why it was in each document to start with.

\DeclareMathOperator{\proj}{\mathbf{proj}}
\newcommand{\veci}{{\boldsymbol{\hat{\imath}}}}
\newcommand{\vecj}{{\boldsymbol{\hat{\jmath}}}}
\newcommand{\veck}{{\boldsymbol{\hat{k}}}}
\newcommand{\vecl}{\vec{\boldsymbol{\l}}}
\newcommand{\uvec}[1]{\mathbf{\hat{#1}}}
\newcommand{\utan}{\mathbf{\hat{t}}}
\newcommand{\unormal}{\mathbf{\hat{n}}}
\newcommand{\ubinormal}{\mathbf{\hat{b}}}

\newcommand{\dotp}{\bullet}
\newcommand{\cross}{\boldsymbol\times}
\newcommand{\grad}{\boldsymbol\nabla}
\newcommand{\divergence}{\grad\dotp}
\newcommand{\curl}{\grad\cross}
%\DeclareMathOperator{\divergence}{divergence}
%\DeclareMathOperator{\curl}[1]{\grad\cross #1}
\newcommand{\lto}{\mathop{\longrightarrow\,}\limits}

\renewcommand{\bar}{\overline}

\colorlet{textColor}{black}
\colorlet{background}{white}
\colorlet{penColor}{blue!50!black} % Color of a curve in a plot
\colorlet{penColor2}{red!50!black}% Color of a curve in a plot
\colorlet{penColor3}{red!50!blue} % Color of a curve in a plot
\colorlet{penColor4}{green!50!black} % Color of a curve in a plot
\colorlet{penColor5}{orange!80!black} % Color of a curve in a plot
\colorlet{penColor6}{yellow!70!black} % Color of a curve in a plot
\colorlet{fill1}{penColor!20} % Color of fill in a plot
\colorlet{fill2}{penColor2!20} % Color of fill in a plot
\colorlet{fillp}{fill1} % Color of positive area
\colorlet{filln}{penColor2!20} % Color of negative area
\colorlet{fill3}{penColor3!20} % Fill
\colorlet{fill4}{penColor4!20} % Fill
\colorlet{fill5}{penColor5!20} % Fill
\colorlet{gridColor}{gray!50} % Color of grid in a plot

\newcommand{\surfaceColor}{violet}
\newcommand{\surfaceColorTwo}{redyellow}
\newcommand{\sliceColor}{greenyellow}




\pgfmathdeclarefunction{gauss}{2}{% gives gaussian
  \pgfmathparse{1/(#2*sqrt(2*pi))*exp(-((x-#1)^2)/(2*#2^2))}%
}


%%%%%%%%%%%%%
%% Vectors
%%%%%%%%%%%%%

%% Simple horiz vectors
\renewcommand{\vector}[1]{\left\langle #1\right\rangle}


%% %% Complex Horiz Vectors with angle brackets
%% \makeatletter
%% \renewcommand{\vector}[2][ , ]{\left\langle%
%%   \def\nextitem{\def\nextitem{#1}}%
%%   \@for \el:=#2\do{\nextitem\el}\right\rangle%
%% }
%% \makeatother

%% %% Vertical Vectors
%% \def\vector#1{\begin{bmatrix}\vecListA#1,,\end{bmatrix}}
%% \def\vecListA#1,{\if,#1,\else #1\cr \expandafter \vecListA \fi}

%%%%%%%%%%%%%
%% End of vectors
%%%%%%%%%%%%%

%\newcommand{\fullwidth}{}
%\newcommand{\normalwidth}{}



%% makes a snazzy t-chart for evaluating functions
%\newenvironment{tchart}{\rowcolors{2}{}{background!90!textColor}\array}{\endarray}

%%This is to help with formatting on future title pages.
\newenvironment{sectionOutcomes}{}{}



%% Flowchart stuff
%\tikzstyle{startstop} = [rectangle, rounded corners, minimum width=3cm, minimum height=1cm,text centered, draw=black]
%\tikzstyle{question} = [rectangle, minimum width=3cm, minimum height=1cm, text centered, draw=black]
%\tikzstyle{decision} = [trapezium, trapezium left angle=70, trapezium right angle=110, minimum width=3cm, minimum height=1cm, text centered, draw=black]
%\tikzstyle{question} = [rectangle, rounded corners, minimum width=3cm, minimum height=1cm,text centered, draw=black]
%\tikzstyle{process} = [rectangle, minimum width=3cm, minimum height=1cm, text centered, draw=black]
%\tikzstyle{decision} = [trapezium, trapezium left angle=70, trapezium right angle=110, minimum width=3cm, minimum height=1cm, text centered, draw=black]


\author{Bart Snapp and Jim Talamo}


\outcome{Define interior points, boundary points, and boundary of a set.}
\outcome{Generalize the notion of open and closed intervals to open and closed sets.}
\outcome{Define bounded and unbounded sets.}

\title[Dig-In:]{Open and Closed Sets}

\begin{document}
\begin{abstract}
We generalize the notion of open and closed intervals to open and closed sets in $\R^2$.
\end{abstract}
\maketitle



When we make definitions and discuss several important theorems for functions of a single variable, 
we need the notion of an open interval or a closed interval.  A typical example of an open interval is $(a,b)$, which represents the set
of all $x$ such that $a<x<b$, and an example of a closed interval is $[a,b]$, which represents the set of all $x$ such that $a\leq x\leq
b$. We need analogous definitions for open and closed sets in $\R^n$.

\begin{definition} 
A \dfn{set} is a collection of distinct objects.

Given a set $A$, we say that $a$ is an \dfn{element} of $A$ if $a$ is one of the distinct objects in $A$, and we write $a \in A$ to denote this.

Given two sets $A$ and $B$, we say that $A$ is a \dfn{subset} of $B$ if every element of $A$ is also an element of $B$ write $A \subseteq B$ to denote this.

\end{definition}

With these ideas in mind, we now discuss special types of subsets.

\begin{definition} (Open Balls)

  We give these definitions in general, for when one is working in
  $\R^n$ since they are really not all that different to define in $\R^n$ than in $\R^2$.

 An \dfn{open ball} $B_r(\mathbf{a})$ in $\R^n$ centered at $\mathbf{a} =\left(a_1,\dots a_n\right) \in \R^n$ with radius $r$ is the set of all  points $\mathbf{x} = \point{x_1,x_2,\dots,x_n} \in \R^n$ such that the distance between $\mathbf{x}$ and $\mathbf{a}$ is less than $r$.
    
    In $\R^2$ an open ball is often called an
    \dfn{open disk}.
    \begin{image}
      \begin{tikzpicture}
        \draw[ultra thick,dashed,penColor,fill=fill1] (0,0) circle (2cm);
        \draw[draw=none,fill=black] (0,0) circle (.1cm);
        \draw[dashed] (0,0)--(2,0);
        \node[above] at (1,0) {r};
        \node[below left] at (0,0) {$\mathbf{a}$};
      \end{tikzpicture}
    \end{image}
 
 \end{definition}
 
 \begin{definition} (Interior and Boundary Points)  Suppose that $S \subseteq \R^n$.
  \begin{itemize}    
  \item A point $\mathbf{p} \in S$ is an \dfn{interior point} of $S$ if there exists an open ball $B_r(\pt{a}) \subseteq S$.

    \begin{image}
      \begin{tikzpicture}
        \draw[ultra thick, penColor,fill=fill1] plot [smooth cycle] coordinates {(-1.5,.5) (.5,1) (1,2.5) (-1,2.5) (-2,1.5)};
        \draw[draw=none,fill=black] (-1,1) circle (.1cm);
        \draw[dashed] (-1,1) circle (.3cm);
        \node[penColor] at (-.5,2) {$S$};      
    \end{tikzpicture}
  \end{image}

Intuitively, $\pt{p}$ is an interior point of $S$ if we can squeeze an entire open ball centered at $\pt{p}$ within $S$.


  \item A point $\pt{p} \in \R^n$ is    a \dfn{boundary point} of $S$ if all open balls centered at $\mathbf{p}$
    contain both points in $S$ and points not in $S$.
    \begin{image}
      \begin{tikzpicture}
        \draw[ultra thick, penColor,fill=fill1] plot [smooth cycle] coordinates {(-1.5,.5) (.5,1) (1,2.5) (-1,2.5) (-2,1.5)};
        \draw[draw=none,fill=black] (.89,1.6) circle (.1cm);
        \draw[dashed] (.89,1.6) circle (.3cm);
        \node[penColor] at (-.5,2) {$S$};      
    \end{tikzpicture}
  \end{image}
  
    
  \item The \dfn{boundary} of $S$ is the set $\partial S$ that consists of all of the boundary points of $S$.
    
  \end{itemize}
\end{definition}

\begin{question}
Suppose that $S = \left\{ (x,y) \in \R^2 ~ \big| ~ x \geq 0, y > 0\right\}$.

Which of the following are elements of $S$?
\begin{selectAll}
\choice{$\point{0,0}$}
\choice[correct]{$\point{0,2}$}
\choice{$\point{2,0}$}
\choice[correct]{$\point{2,2}$}
\end{selectAll}

Which of the following are interior points of $S$?
\begin{selectAll}
\choice{$\point{0,0}$}
\choice{$\point{0,2}$}
\choice{$\point{2,0}$}
\choice[correct]{$\point{2,2}$}
\end{selectAll}

Which of the following are boundary points of $S$?
\begin{selectAll}
\choice{$\point{0,0}$}
\choice[correct]{$\point{0,2}$}
\choice{$\point{2,0}$}
\choice{$\point{2,2}$}
\end{selectAll}

The boundary $\partial S $  is \wordChoice{\choice{$\left\{ (x,y) \in \R^2 ~ \big| ~ x = 0, y > 0\right\}$}\choice{$ \left\{ (x,y) \in \R^2 ~ \big| ~ x = 0, y = 0\right\}$}\choice[correct]{$\left\{ (x,y) \in \R^2 ~ \big| ~ x > 0, y = 0\right\}$} \choice{$\left\{ (x,y) \in \R^2 ~ \big| ~ x > 0, y > 0\right\}$}}
\end{question}

We can now generalize the notion of open and closed intervals from $\R$ to open and closed sets in $\R^n$.

\begin{definition} (Open and Closed Sets)

\begin{itemize}
  \item A set $O \subseteq \R^n$ is \dfn{open} if every point in $O$ is an interior
    point.
    \begin{image}
      \begin{tikzpicture}
        \draw[ultra thick, dashed, penColor,fill=fill1] plot [smooth cycle] coordinates {(-1.5,.5) (.5,1) (1,2.5) (-1,2.5) (-2,1.5)};
        \node[penColor] at (-.5,2) {$O$};     
      \end{tikzpicture}
    \end{image}
  \item A set $C$ is \dfn{closed} if it contains all of its boundary
    points.
    \begin{image}
      \begin{tikzpicture}
        \draw[ultra thick, penColor,fill=fill1] plot [smooth cycle] coordinates {(-1.5,.5) (.5,1) (1,2.5) (-1,2.5) (-2,1.5)};
        \node[penColor] at (-.5,2) {$C$};     
      \end{tikzpicture}
    \end{image}
    
    
\end{itemize}
    
\end{definition}    
    
\begin{question}
Determine if the following sets are open, closed, or neither.

\begin{itemize}
\item The set $\left\{ (x,y) \in \R^2 \big| |x+y| < 1 \right\}$ is \wordChoice{\choice[correct]{open}\choice{closed}\choice{neither open nor closed}}.
\item The set $\left\{ (x,y) \in \R^2 \big| x \leq 0 , y \leq 0 \right\}$ is \wordChoice{\choice{open}\choice[correct]{closed}\choice{neither open nor closed}}.
\item The set $\left\{ (x,y) \in \R^2 \big| x \leq 0 , y<0\right\}$ is \wordChoice{\choice{open}\choice{closed}\choice[correct]{neither open nor closed}}.
\end{itemize}
\end{question}

 
 \begin{definition}   (Bounded and Unbounded) 

      \begin{itemize}
  \item A set $S$ is \dfn{bounded} if there is an open ball $B_M(\pt{0})$ such that
    \[
    S\subseteq B.
    \]
    
    Intuitively, this means that we can enclose all of the set $S$ within a large enough ball centered at the origin.
    
    \item A set that is not bounded is called \dfn{unbounded}.
  \end{itemize}

\end{definition}

\begin{question}
Which of the following sets are bounded?

\begin{selectAll}
\choice{$\left\{ (x,y) \in \R^2 \big| |x+y| < 1 \right\}$}
\choice{$\left\{ (x,y) \in \R^2 \big| x \leq 0 , y < 0 \right\}$}
\choice{$\left\{ (x,y) \in \R^2 \big| x+y^2 < 1 \right\}$}
\choice[correct]{$\left\{ (x,y) \in \R^2 \big| x^2+y^2 \leq 1 \right\}$}
\end{selectAll}
\end{question}
%\begin{example}
%  Consider a closed disk $D$ in $\R^2$. Describe $\partial D$ and $\partial \partial D$.
%  \begin{explanation}
%    Since $\partial D$ is the boundary of a closed disk in $\R^2$, $\partial D$ is \wordChoice{
%      \choice{a disk}
%      \choice[correct]{a circle}
%      \choice{a ball}
%      \choice{a line}}. Since $\partial\partial D$ is the boundary of the boundary, and a circle has no boundary, $\partial\partial D$ is \wordChoice{
%      \choice[correct]{empty} <--------There is no discussion of relative topology anywhere in here and needs to be do discuss this.  I do not need this example in Math 1172.  I will add it back in if someone else writes something about what "boundary of a boundary" means.
%      \choice{a line}
%      \choice{a circle}
%      }.
%  \end{explanation}
%\end{example}

Let's now look at a few examples.

\begin{example}
 Consider the function
  $F(x,y)=\sqrt{1-\frac{x^2}9-\frac{y^2}4}$.
  
  \begin{itemize}
  \item The domain $D$ of the function is the set of all $(x,y)$ for which $1-\frac{x^2}9-\frac{y^2}4 \geq 0$, which we can write in set 
  
    \[
    D = \{(x,y): \answer[given]{x^2/9+y^2/4}\leq 1\}.
    \]
  
  \item The point $(1,1)$ is \wordChoice{\choice[correct]{an interior point}\choice{a boundary point}\choice{not an element}} of $D$.
  \item The point $(1,2)$ is \wordChoice{\choice{an interior point}\choice{a boundary point}\choice[correct]{not an element}} of $D$.
  \item The domain is   \wordChoice{\choice{open}\choice[correct]{closed}\choice{neither open nor closed}} and \wordChoice{\choice[correct]{bounded}\choice{not bounded}} . 

\end{itemize}

  \begin{explanation}
    We've already found the domain of this function to be

    This is the region \textit{bounded} by the ellipse
    $\frac{x^2}9+\frac{y^2}4=1$. Since the region includes the
    boundary (indicated by the use of ``$\leq$''), the set
    \wordChoice{\choice[correct]{contains}\choice{does not contain}}
    all of its boundary points and hence is closed. The region is
    \wordChoice{\choice[correct]{bounded}\choice{unbounded}} as a disk
    of radius $4$, centered at the origin, contains $D$.
  \end{explanation}
\end{example}

\begin{example}
  Determine if the domain of $F(x,y) = \frac{1}{x-y}$ is open, closed,
  or neither, and if it is bounded.
  \begin{explanation}
    As we cannot divide by $0$, we find the domain to be
    \[
    D = \{(x,y):x-y\neq \answer[given]{0}\}.
    \]
    In other words, the domain is the set of all points $(x,y)$
    \textit{not} on the line $y=x$. For your viewing pleasure, we have
    included a graph:
    \begin{image}
      \begin{tikzpicture}
        \begin{axis}[
            tick label style={font=\scriptsize},axis y line=middle,axis x line=middle,name=myplot,axis on top,%
            xtick=\empty,
            ytick=\empty,
            ymin=-1,ymax=1,%
            xmin=-1,xmax=1%
          ]
          \filldraw [fill1,fill=fill1] (axis cs:-1,-1) rectangle (axis cs: 1,1);          
          \addplot [ultra thick,white]coordinates {(-1.,-1.)(1,1)};
        \end{axis}
        \node [right] at (myplot.right of origin) {\scriptsize $x$};
        \node [above] at (myplot.above origin) {\scriptsize $y$};
      \end{tikzpicture}
    \end{image}
    Note how we can draw an open disk around any point in the domain
    that lies entirely inside the domain, and also note how the only
    boundary points of the domain are the points on the line $y=x$. We
    conclude the domain is \wordChoice{\choice[correct]{an open
        set}\choice{a closed set}\choice{neither open nor closed
        set}}. Moreover, the set is \wordChoice{\choice{bounded}\choice[correct]{unbounded}}.
  \end{explanation}
\end{example}


\end{document}
