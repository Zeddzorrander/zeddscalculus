\documentclass{ximera}

%\usepackage{todonotes}
%\usepackage{mathtools} %% Required for wide table Curl and Greens
%\usepackage{cuted} %% Required for wide table Curl and Greens
\newcommand{\todo}{}

\usepackage{esint} % for \oiint
\ifxake%%https://math.meta.stackexchange.com/questions/9973/how-do-you-render-a-closed-surface-double-integral
\renewcommand{\oiint}{{\large\bigcirc}\kern-1.56em\iint}
\fi


\graphicspath{
  {./}
  {ximeraTutorial/}
  {basicPhilosophy/}
  {functionsOfSeveralVariables/}
  {normalVectors/}
  {lagrangeMultipliers/}
  {vectorFields/}
  {greensTheorem/}
  {shapeOfThingsToCome/}
  {dotProducts/}
  {partialDerivativesAndTheGradientVector/}
  {../productAndQuotientRules/exercises/}
  {../motionAndPathsInSpace/exercises/}
  {../normalVectors/exercisesParametricPlots/}
  {../continuityOfFunctionsOfSeveralVariables/exercises/}
  {../partialDerivativesAndTheGradientVector/exercises/}
  {../directionalDerivativeAndChainRule/exercises/}
  {../commonCoordinates/exercisesCylindricalCoordinates/}
  {../commonCoordinates/exercisesSphericalCoordinates/}
  {../greensTheorem/exercisesCurlAndLineIntegrals/}
  {../greensTheorem/exercisesDivergenceAndLineIntegrals/}
  {../shapeOfThingsToCome/exercisesDivergenceTheorem/}
  {../greensTheorem/}
  {../shapeOfThingsToCome/}
  {../separableDifferentialEquations/exercises/}
  {vectorFields/}
}

\newcommand{\mooculus}{\textsf{\textbf{MOOC}\textnormal{\textsf{ULUS}}}}

\usepackage{tkz-euclide}\usepackage{tikz}
\usepackage{tikz-cd}
\usetikzlibrary{arrows}
\tikzset{>=stealth,commutative diagrams/.cd,
  arrow style=tikz,diagrams={>=stealth}} %% cool arrow head
\tikzset{shorten <>/.style={ shorten >=#1, shorten <=#1 } } %% allows shorter vectors

\usetikzlibrary{backgrounds} %% for boxes around graphs
\usetikzlibrary{shapes,positioning}  %% Clouds and stars
\usetikzlibrary{matrix} %% for matrix
\usepgfplotslibrary{polar} %% for polar plots
\usepgfplotslibrary{fillbetween} %% to shade area between curves in TikZ
\usetkzobj{all}
\usepackage[makeroom]{cancel} %% for strike outs
%\usepackage{mathtools} %% for pretty underbrace % Breaks Ximera
%\usepackage{multicol}
\usepackage{pgffor} %% required for integral for loops



%% http://tex.stackexchange.com/questions/66490/drawing-a-tikz-arc-specifying-the-center
%% Draws beach ball
\tikzset{pics/carc/.style args={#1:#2:#3}{code={\draw[pic actions] (#1:#3) arc(#1:#2:#3);}}}



\usepackage{array}
\setlength{\extrarowheight}{+.1cm}
\newdimen\digitwidth
\settowidth\digitwidth{9}
\def\divrule#1#2{
\noalign{\moveright#1\digitwidth
\vbox{\hrule width#2\digitwidth}}}





\newcommand{\RR}{\mathbb R}
\newcommand{\R}{\mathbb R}
\newcommand{\N}{\mathbb N}
\newcommand{\Z}{\mathbb Z}

\newcommand{\sagemath}{\textsf{SageMath}}


%\renewcommand{\d}{\,d\!}
\renewcommand{\d}{\mathop{}\!d}
\newcommand{\dd}[2][]{\frac{\d #1}{\d #2}}
\newcommand{\pp}[2][]{\frac{\partial #1}{\partial #2}}
\renewcommand{\l}{\ell}
\newcommand{\ddx}{\frac{d}{\d x}}

\newcommand{\zeroOverZero}{\ensuremath{\boldsymbol{\tfrac{0}{0}}}}
\newcommand{\inftyOverInfty}{\ensuremath{\boldsymbol{\tfrac{\infty}{\infty}}}}
\newcommand{\zeroOverInfty}{\ensuremath{\boldsymbol{\tfrac{0}{\infty}}}}
\newcommand{\zeroTimesInfty}{\ensuremath{\small\boldsymbol{0\cdot \infty}}}
\newcommand{\inftyMinusInfty}{\ensuremath{\small\boldsymbol{\infty - \infty}}}
\newcommand{\oneToInfty}{\ensuremath{\boldsymbol{1^\infty}}}
\newcommand{\zeroToZero}{\ensuremath{\boldsymbol{0^0}}}
\newcommand{\inftyToZero}{\ensuremath{\boldsymbol{\infty^0}}}



\newcommand{\numOverZero}{\ensuremath{\boldsymbol{\tfrac{\#}{0}}}}
\newcommand{\dfn}{\textbf}
%\newcommand{\unit}{\,\mathrm}
\newcommand{\unit}{\mathop{}\!\mathrm}
\newcommand{\eval}[1]{\bigg[ #1 \bigg]}
\newcommand{\seq}[1]{\left( #1 \right)}
\renewcommand{\epsilon}{\varepsilon}
\renewcommand{\phi}{\varphi}


\renewcommand{\iff}{\Leftrightarrow}

\DeclareMathOperator{\arccot}{arccot}
\DeclareMathOperator{\arcsec}{arcsec}
\DeclareMathOperator{\arccsc}{arccsc}
\DeclareMathOperator{\si}{Si}
\DeclareMathOperator{\scal}{scal}
\DeclareMathOperator{\sign}{sign}


%% \newcommand{\tightoverset}[2]{% for arrow vec
%%   \mathop{#2}\limits^{\vbox to -.5ex{\kern-0.75ex\hbox{$#1$}\vss}}}
\newcommand{\arrowvec}[1]{{\overset{\rightharpoonup}{#1}}}
%\renewcommand{\vec}[1]{\arrowvec{\mathbf{#1}}}
\renewcommand{\vec}[1]{{\overset{\boldsymbol{\rightharpoonup}}{\mathbf{#1}}}\hspace{0in}}

\newcommand{\point}[1]{\left(#1\right)} %this allows \vector{ to be changed to \vector{ with a quick find and replace
\newcommand{\pt}[1]{\mathbf{#1}} %this allows \vec{ to be changed to \vec{ with a quick find and replace
\newcommand{\Lim}[2]{\lim_{\point{#1} \to \point{#2}}} %Bart, I changed this to point since I want to use it.  It runs through both of the exercise and exerciseE files in limits section, which is why it was in each document to start with.

\DeclareMathOperator{\proj}{\mathbf{proj}}
\newcommand{\veci}{{\boldsymbol{\hat{\imath}}}}
\newcommand{\vecj}{{\boldsymbol{\hat{\jmath}}}}
\newcommand{\veck}{{\boldsymbol{\hat{k}}}}
\newcommand{\vecl}{\vec{\boldsymbol{\l}}}
\newcommand{\uvec}[1]{\mathbf{\hat{#1}}}
\newcommand{\utan}{\mathbf{\hat{t}}}
\newcommand{\unormal}{\mathbf{\hat{n}}}
\newcommand{\ubinormal}{\mathbf{\hat{b}}}

\newcommand{\dotp}{\bullet}
\newcommand{\cross}{\boldsymbol\times}
\newcommand{\grad}{\boldsymbol\nabla}
\newcommand{\divergence}{\grad\dotp}
\newcommand{\curl}{\grad\cross}
%\DeclareMathOperator{\divergence}{divergence}
%\DeclareMathOperator{\curl}[1]{\grad\cross #1}
\newcommand{\lto}{\mathop{\longrightarrow\,}\limits}

\renewcommand{\bar}{\overline}

\colorlet{textColor}{black}
\colorlet{background}{white}
\colorlet{penColor}{blue!50!black} % Color of a curve in a plot
\colorlet{penColor2}{red!50!black}% Color of a curve in a plot
\colorlet{penColor3}{red!50!blue} % Color of a curve in a plot
\colorlet{penColor4}{green!50!black} % Color of a curve in a plot
\colorlet{penColor5}{orange!80!black} % Color of a curve in a plot
\colorlet{penColor6}{yellow!70!black} % Color of a curve in a plot
\colorlet{fill1}{penColor!20} % Color of fill in a plot
\colorlet{fill2}{penColor2!20} % Color of fill in a plot
\colorlet{fillp}{fill1} % Color of positive area
\colorlet{filln}{penColor2!20} % Color of negative area
\colorlet{fill3}{penColor3!20} % Fill
\colorlet{fill4}{penColor4!20} % Fill
\colorlet{fill5}{penColor5!20} % Fill
\colorlet{gridColor}{gray!50} % Color of grid in a plot

\newcommand{\surfaceColor}{violet}
\newcommand{\surfaceColorTwo}{redyellow}
\newcommand{\sliceColor}{greenyellow}




\pgfmathdeclarefunction{gauss}{2}{% gives gaussian
  \pgfmathparse{1/(#2*sqrt(2*pi))*exp(-((x-#1)^2)/(2*#2^2))}%
}


%%%%%%%%%%%%%
%% Vectors
%%%%%%%%%%%%%

%% Simple horiz vectors
\renewcommand{\vector}[1]{\left\langle #1\right\rangle}


%% %% Complex Horiz Vectors with angle brackets
%% \makeatletter
%% \renewcommand{\vector}[2][ , ]{\left\langle%
%%   \def\nextitem{\def\nextitem{#1}}%
%%   \@for \el:=#2\do{\nextitem\el}\right\rangle%
%% }
%% \makeatother

%% %% Vertical Vectors
%% \def\vector#1{\begin{bmatrix}\vecListA#1,,\end{bmatrix}}
%% \def\vecListA#1,{\if,#1,\else #1\cr \expandafter \vecListA \fi}

%%%%%%%%%%%%%
%% End of vectors
%%%%%%%%%%%%%

%\newcommand{\fullwidth}{}
%\newcommand{\normalwidth}{}



%% makes a snazzy t-chart for evaluating functions
%\newenvironment{tchart}{\rowcolors{2}{}{background!90!textColor}\array}{\endarray}

%%This is to help with formatting on future title pages.
\newenvironment{sectionOutcomes}{}{}



%% Flowchart stuff
%\tikzstyle{startstop} = [rectangle, rounded corners, minimum width=3cm, minimum height=1cm,text centered, draw=black]
%\tikzstyle{question} = [rectangle, minimum width=3cm, minimum height=1cm, text centered, draw=black]
%\tikzstyle{decision} = [trapezium, trapezium left angle=70, trapezium right angle=110, minimum width=3cm, minimum height=1cm, text centered, draw=black]
%\tikzstyle{question} = [rectangle, rounded corners, minimum width=3cm, minimum height=1cm,text centered, draw=black]
%\tikzstyle{process} = [rectangle, minimum width=3cm, minimum height=1cm, text centered, draw=black]
%\tikzstyle{decision} = [trapezium, trapezium left angle=70, trapezium right angle=110, minimum width=3cm, minimum height=1cm, text centered, draw=black]


\title[Dig-In:]{Unit tangent and unit normal vectors}

\outcome{Define normal vectors.}
\outcome{Define unit tangent and unit normal vectors.} 
\outcome{Define principal unit normal vectors.}

\begin{document}
\begin{abstract}
  We introduce two important unit vectors. 
\end{abstract}
\maketitle


Given a smooth vector-valued function $\vec{p}(t)$, \textit{any}
vector parallel to $\vec{p}'(t_0)$ is \textit{tangent} to the graph of
$\vec{p}(t)$ at $t=t_0$. It is often useful to consider just the
\textit{direction} of $\vec{p}'(t)$ and not its magnitude. Therefore we are
interested in the unit vector in the direction of $\vec{p}'(t)$. This
leads to a definition.
\begin{definition}
Let $\vec{p}(t)$ be a smooth function on an open interval $I$. The
\dfn{unit tangent vector} $\uvec{t}(t)$ is \index{unit tangent
  vector!definition} \index{unit vector!unit tangent vector}
\[
\utan(t) = \frac{\vec{p}'(t)}{|\vec{p}'(t)|}.
\]
\end{definition}

\begin{question}
  Let $\vec{p}(t) = \vector{3\cos(t), 3\sin(t), 4t}$. Find $\utan(t)$.
  \begin{prompt}
    \[
    \utan(t) = \vector{\answer{\frac{-3}{5}\sin(t)},\answer{\frac{3}{5}\cos(t)},\answer{\frac{4}{5}}}
    \]
    \begin{feedback}
      The unit tangent vector $\utan(t)$ \textbf{always} has a constant
      magnitude of $1$.
    \end{feedback}
  \end{prompt}
\end{question}


In previous courses, we found tangent lines to curves at given points.
Just as knowing the direction tangent to a path is important, knowing
a direction orthogonal to a path is important. When dealing with
real-valued functions, one defines the \dfn{normal line} at a point to
the be the line through the point perpendicular to the tangent line at
that point. We can do a similar thing with vector-valued
functions. Given $\vec{p}(t)$ in $\R^2$, we have $2$ directions
perpendicular to the tangent vector
\begin{image}
  \begin{tikzpicture}
    \begin{axis}%
      [width=175pt,tick label style={font=\scriptsize},axis on top,
	axis lines=center,
	view={115}{25},
	name=myplot,
	%xtick={-3,3},minor tick num=2,
	%ytick={-3,3},
	%ztick={-3,3},
	ymin=-3.5,ymax=3.5,
	xmin=-3.5,xmax=3.5,
	zmin=-15.9, zmax=15.9,
	every axis x label/.style={at={(axis cs:\pgfkeysvalueof{/pgfplots/xmax},0,0)},xshift=-3pt,yshift=-3pt},
	xlabel={\scriptsize $x$},
	every axis y label/.style={at={(axis cs:0,\pgfkeysvalueof{/pgfplots/ymax},0)},xshift=0pt,yshift=-5pt},
	ylabel={\scriptsize $y$},
	every axis z label/.style={at={(axis cs:0,0,\pgfkeysvalueof{/pgfplots/zmax})},xshift=0pt,yshift=4pt},
	zlabel={\scriptsize $z$}
      ]
      
      \addplot3[domain=-3.14:3.14,,thick,smooth,samples y=0,penColor,samples=30,] ({3*cos(x*180/3.14)},{3*sin(x*180/3.14)},{4*x});

      
      \draw[thick,->,penColor2] (axis cs: 3,0,0) -- (axis cs: 3,.6,.8);
      \draw[thick,->,penColor2] (axis cs: 1.6,2.5,4) -- (axis cs: 1.12,2.85,4.8);
    \end{axis}
  \end{tikzpicture}
\end{image}
The young mathematician wonders ``Is one of these two directions
preferable over the other?''  This question only gets harder in higher
dimensions.  Given $\vec{p}(t)$ in $\R^3$, there are infinite vectors
orthogonal to the tangent vector at a given point. Again, we might
wonder ``Is one of these infinite choices preferable over the others?
Is one of these the `right' choice?'' Well, we have several options
for finding vectors normal to curves. In $\R^2$ if $\vec{p}(t) =
\vector{x(t),y(t)}$ we could write the tangent vector as:
\[
\vec{p}'(t) =\vector{x'(t),y'(t)}
\]
and then a normal vector as
\[
\vec{n}(t) = \vector{y'(t),-x'(t)}
\]
for a vector normal to $\vec{p}'(t)$. You can check for yourself that
this vector is normal to $\vec{p}'(t)$ using the dot product. In
two-dimensions, the vector $\vec{n}$ defined above will always point
``outward'' for a closed curve drawn in a counterclockwise
fashion. Below we see a closed curve drawn in a counterclockwise
fashion with some normal vectors:
\begin{image}
  \begin{tikzpicture}
	\begin{axis}[
            xmin=-4.1,xmax=4.1,ymin=-4.1,ymax=4.1,
            axis lines=center,
            clip=false,
            unit vector ratio*=1 1 1,
            xlabel=$x$, ylabel=$y$,
            every axis y label/.style={at=(current axis.above origin),anchor=south},
            every axis x label/.style={at=(current axis.right of origin),anchor=west},
          ]        
          \addplot [ultra thick, penColor, smooth, samples=100,domain=(0:360)] ({cos(x)*(3+sin(5*x))},{sin(x)*(3+sin(5*x))});
          
          \foreach \x in {50,120,200,260,330}
          {
          \addplot[penColor2,->,ultra thick] plot coordinates {
            (
            {cos(\x)*(3+sin(5*\x))},
            {sin(\x)*(3+sin(5*\x))}
            )
            (
            { 5*cos(5*\x)*sin(\x)+2*cos(\x)*(3+sin(5*\x))},
            {-5*cos(\x)*cos(5*\x)+2*sin(\x)*(3+sin(5*\x))}
            )
          };
          }
          \node[penColor,left] at (axis cs: -1,3) {$\vec{p}(t) = \vector{x(t),y(t)}$};
          \node[penColor2,right] at (axis cs: 4,-1) {$\vec{n}(t) = \vector{y'(t),-x'(t)}$};
        \end{axis}
\end{tikzpicture}
\end{image}
On the other hand, there is no analogous trick for vector-valued
functions in higher dimensions. In this case, recall:
\begin{quote}
If $\vec{p}(t)$ has constant length, then $\vec{p}(t)$ is orthogonal
to $\vec{p}'(t)$ for all $t$.
\end{quote}
Since $\utan(t)$ is a unit tangent vector, it necessarily has
constant length. Therefore
\[
\utan(t)\text{ is orthogonal to }\utan'(t).
\]
\begin{warning}
  Even though $\utan(t)$ is a unit vector, this \textbf{does not}
  imply that $\utan'(t)$ is also a unit vector.
\end{warning}
Since constructing normal vectors using the derivative works in all
dimensions, we will often use this to construct our \textit{unit
  normal vector}.
\begin{definition}
Let $\vec{p}(t)$ be a vector-valued function where the unit tangent
vector, $\utan(t)$, is smooth on an open interval $I$. The
\dfn{unit normal vector} $\unormal(t)$ is \index{unit normal
  vector!definition}\index{unit vector!unit normal vector}
\[
\unormal(t) = \frac{\utan'(t)}{|\utan'(t)|}.
\]
Some folks call this the \dfn{principal unit normal vector}.
\end{definition}

The principal unit normal vector will always point toward the
``inside'' of how a curve is curving.
\begin{image}
  \begin{tikzpicture}
	\begin{axis}[
            xmin=-5,xmax=5,ymin=-1.2,ymax=1.2,
            axis lines=center,
            clip=false,
            unit vector ratio*=1 1 1,
            xlabel=$x$, ylabel=$y$,
            every axis y label/.style={at=(current axis.above origin),anchor=south},
            every axis x label/.style={at=(current axis.right of origin),anchor=west},
          ]        
          \addplot [ultra thick, penColor, smooth,domain=-5:5] {sin(deg(x))};
          
          \foreach \x in {-250,-202,...,250}
                   {
                     \addplot[penColor2,->,ultra thick] plot coordinates {
                       (
            {2*pi*\x/360},
            {sin(\x)}
            )
            (
            {2*pi*\x/360  + sqrt(1+(cos(\x))^2)*(cos(\x)/sin(\x))*sqrt(((sin(\x))^2)/((1+(cos(\x))^2)^2))},
            {sin(\x)      - sqrt(1+(cos(\x))^2)*(1/sin(\x))      *sqrt(((sin(\x))^2)/((1+(cos(\x))^2)^2))}
            )
          };
                   }
                    \node[penColor,above] at (axis cs: -5,1) {$\vec{p}(t) = \vector{x(t),y(t)}$};
                    \node[penColor2] at (axis cs: 2,-1.5) {$\unormal(t) = \frac{\utan'(t)}{|\utan'(t)|}$};
        \end{axis}
  \end{tikzpicture}
\end{image}



\begin{question}
  Let $\vec{p}(t) = \vector{3\cos t, 3\sin t, 4t}$ as before. Find the principal unit normal vector
  $\unormal(t)$.
  \begin{prompt}
    \[
    \unormal(t) =\vector{\answer{-\cos(t)},\answer{-\sin(t)},\answer{0}}.
    \]
    \begin{feedback}
      As a gesture of friendship, we present you with the following
      graph of the situation.
      \begin{image}
        \begin{tikzpicture}
          \begin{axis}%
            [width=175pt,tick label style={font=\scriptsize},axis on top,
	      axis lines=center,
	      view={135}{25},
	      name=myplot,
	      %xtick={-3,3},minor tick num=2,
	      %ytick={-3,3},
	      %ztick={-3,3},
	      ymin=-3.5,ymax=3.5,
	      xmin=-3.5,xmax=3.5,
	      zmin=-15.9, zmax=15.9,
	      every axis x label/.style={at={(axis cs:\pgfkeysvalueof{/pgfplots/xmax},0,0)},xshift=-3pt,yshift=-3pt},
	      xlabel={\scriptsize $x$},
	      every axis y label/.style={at={(axis cs:0,\pgfkeysvalueof{/pgfplots/ymax},0)},xshift=0pt,yshift=-5pt},
	      ylabel={\scriptsize $y$},
	      every axis z label/.style={at={(axis cs:0,0,\pgfkeysvalueof{/pgfplots/zmax})},xshift=0pt,yshift=4pt},
	      zlabel={\scriptsize $z$}
	    ]
            \addplot3[domain=-3.14:3.14,,thick,smooth,samples y=0,penColor,samples=30,] ({3*cos(x*180/3.14)},{3*sin(x*180/3.14)},{4*x});

            
            \draw[thick,->,penColor2] (axis cs: 0,3,6.28) -- (axis cs: -.6,3,7.08);
            \draw[thick,->,penColor2] (axis cs: 0,3,6.28) -- (axis cs: 0,2,6.28);
          \end{axis}
        \end{tikzpicture}
      \end{image}
    \end{feedback}
  \end{prompt}
\end{question}

\end{document}
