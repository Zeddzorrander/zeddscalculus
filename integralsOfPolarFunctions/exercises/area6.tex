\documentclass{ximera}

%\usepackage{todonotes}
%\usepackage{mathtools} %% Required for wide table Curl and Greens
%\usepackage{cuted} %% Required for wide table Curl and Greens
\newcommand{\todo}{}

\usepackage{esint} % for \oiint
\ifxake%%https://math.meta.stackexchange.com/questions/9973/how-do-you-render-a-closed-surface-double-integral
\renewcommand{\oiint}{{\large\bigcirc}\kern-1.56em\iint}
\fi


\graphicspath{
  {./}
  {ximeraTutorial/}
  {basicPhilosophy/}
  {functionsOfSeveralVariables/}
  {normalVectors/}
  {lagrangeMultipliers/}
  {vectorFields/}
  {greensTheorem/}
  {shapeOfThingsToCome/}
  {dotProducts/}
  {partialDerivativesAndTheGradientVector/}
  {../productAndQuotientRules/exercises/}
  {../motionAndPathsInSpace/exercises/}
  {../normalVectors/exercisesParametricPlots/}
  {../continuityOfFunctionsOfSeveralVariables/exercises/}
  {../partialDerivativesAndTheGradientVector/exercises/}
  {../directionalDerivativeAndChainRule/exercises/}
  {../commonCoordinates/exercisesCylindricalCoordinates/}
  {../commonCoordinates/exercisesSphericalCoordinates/}
  {../greensTheorem/exercisesCurlAndLineIntegrals/}
  {../greensTheorem/exercisesDivergenceAndLineIntegrals/}
  {../shapeOfThingsToCome/exercisesDivergenceTheorem/}
  {../greensTheorem/}
  {../shapeOfThingsToCome/}
  {../separableDifferentialEquations/exercises/}
  {vectorFields/}
}

\newcommand{\mooculus}{\textsf{\textbf{MOOC}\textnormal{\textsf{ULUS}}}}

\usepackage{tkz-euclide}\usepackage{tikz}
\usepackage{tikz-cd}
\usetikzlibrary{arrows}
\tikzset{>=stealth,commutative diagrams/.cd,
  arrow style=tikz,diagrams={>=stealth}} %% cool arrow head
\tikzset{shorten <>/.style={ shorten >=#1, shorten <=#1 } } %% allows shorter vectors

\usetikzlibrary{backgrounds} %% for boxes around graphs
\usetikzlibrary{shapes,positioning}  %% Clouds and stars
\usetikzlibrary{matrix} %% for matrix
\usepgfplotslibrary{polar} %% for polar plots
\usepgfplotslibrary{fillbetween} %% to shade area between curves in TikZ
\usetkzobj{all}
\usepackage[makeroom]{cancel} %% for strike outs
%\usepackage{mathtools} %% for pretty underbrace % Breaks Ximera
%\usepackage{multicol}
\usepackage{pgffor} %% required for integral for loops



%% http://tex.stackexchange.com/questions/66490/drawing-a-tikz-arc-specifying-the-center
%% Draws beach ball
\tikzset{pics/carc/.style args={#1:#2:#3}{code={\draw[pic actions] (#1:#3) arc(#1:#2:#3);}}}



\usepackage{array}
\setlength{\extrarowheight}{+.1cm}
\newdimen\digitwidth
\settowidth\digitwidth{9}
\def\divrule#1#2{
\noalign{\moveright#1\digitwidth
\vbox{\hrule width#2\digitwidth}}}





\newcommand{\RR}{\mathbb R}
\newcommand{\R}{\mathbb R}
\newcommand{\N}{\mathbb N}
\newcommand{\Z}{\mathbb Z}

\newcommand{\sagemath}{\textsf{SageMath}}


%\renewcommand{\d}{\,d\!}
\renewcommand{\d}{\mathop{}\!d}
\newcommand{\dd}[2][]{\frac{\d #1}{\d #2}}
\newcommand{\pp}[2][]{\frac{\partial #1}{\partial #2}}
\renewcommand{\l}{\ell}
\newcommand{\ddx}{\frac{d}{\d x}}

\newcommand{\zeroOverZero}{\ensuremath{\boldsymbol{\tfrac{0}{0}}}}
\newcommand{\inftyOverInfty}{\ensuremath{\boldsymbol{\tfrac{\infty}{\infty}}}}
\newcommand{\zeroOverInfty}{\ensuremath{\boldsymbol{\tfrac{0}{\infty}}}}
\newcommand{\zeroTimesInfty}{\ensuremath{\small\boldsymbol{0\cdot \infty}}}
\newcommand{\inftyMinusInfty}{\ensuremath{\small\boldsymbol{\infty - \infty}}}
\newcommand{\oneToInfty}{\ensuremath{\boldsymbol{1^\infty}}}
\newcommand{\zeroToZero}{\ensuremath{\boldsymbol{0^0}}}
\newcommand{\inftyToZero}{\ensuremath{\boldsymbol{\infty^0}}}



\newcommand{\numOverZero}{\ensuremath{\boldsymbol{\tfrac{\#}{0}}}}
\newcommand{\dfn}{\textbf}
%\newcommand{\unit}{\,\mathrm}
\newcommand{\unit}{\mathop{}\!\mathrm}
\newcommand{\eval}[1]{\bigg[ #1 \bigg]}
\newcommand{\seq}[1]{\left( #1 \right)}
\renewcommand{\epsilon}{\varepsilon}
\renewcommand{\phi}{\varphi}


\renewcommand{\iff}{\Leftrightarrow}

\DeclareMathOperator{\arccot}{arccot}
\DeclareMathOperator{\arcsec}{arcsec}
\DeclareMathOperator{\arccsc}{arccsc}
\DeclareMathOperator{\si}{Si}
\DeclareMathOperator{\scal}{scal}
\DeclareMathOperator{\sign}{sign}


%% \newcommand{\tightoverset}[2]{% for arrow vec
%%   \mathop{#2}\limits^{\vbox to -.5ex{\kern-0.75ex\hbox{$#1$}\vss}}}
\newcommand{\arrowvec}[1]{{\overset{\rightharpoonup}{#1}}}
%\renewcommand{\vec}[1]{\arrowvec{\mathbf{#1}}}
\renewcommand{\vec}[1]{{\overset{\boldsymbol{\rightharpoonup}}{\mathbf{#1}}}\hspace{0in}}

\newcommand{\point}[1]{\left(#1\right)} %this allows \vector{ to be changed to \vector{ with a quick find and replace
\newcommand{\pt}[1]{\mathbf{#1}} %this allows \vec{ to be changed to \vec{ with a quick find and replace
\newcommand{\Lim}[2]{\lim_{\point{#1} \to \point{#2}}} %Bart, I changed this to point since I want to use it.  It runs through both of the exercise and exerciseE files in limits section, which is why it was in each document to start with.

\DeclareMathOperator{\proj}{\mathbf{proj}}
\newcommand{\veci}{{\boldsymbol{\hat{\imath}}}}
\newcommand{\vecj}{{\boldsymbol{\hat{\jmath}}}}
\newcommand{\veck}{{\boldsymbol{\hat{k}}}}
\newcommand{\vecl}{\vec{\boldsymbol{\l}}}
\newcommand{\uvec}[1]{\mathbf{\hat{#1}}}
\newcommand{\utan}{\mathbf{\hat{t}}}
\newcommand{\unormal}{\mathbf{\hat{n}}}
\newcommand{\ubinormal}{\mathbf{\hat{b}}}

\newcommand{\dotp}{\bullet}
\newcommand{\cross}{\boldsymbol\times}
\newcommand{\grad}{\boldsymbol\nabla}
\newcommand{\divergence}{\grad\dotp}
\newcommand{\curl}{\grad\cross}
%\DeclareMathOperator{\divergence}{divergence}
%\DeclareMathOperator{\curl}[1]{\grad\cross #1}
\newcommand{\lto}{\mathop{\longrightarrow\,}\limits}

\renewcommand{\bar}{\overline}

\colorlet{textColor}{black}
\colorlet{background}{white}
\colorlet{penColor}{blue!50!black} % Color of a curve in a plot
\colorlet{penColor2}{red!50!black}% Color of a curve in a plot
\colorlet{penColor3}{red!50!blue} % Color of a curve in a plot
\colorlet{penColor4}{green!50!black} % Color of a curve in a plot
\colorlet{penColor5}{orange!80!black} % Color of a curve in a plot
\colorlet{penColor6}{yellow!70!black} % Color of a curve in a plot
\colorlet{fill1}{penColor!20} % Color of fill in a plot
\colorlet{fill2}{penColor2!20} % Color of fill in a plot
\colorlet{fillp}{fill1} % Color of positive area
\colorlet{filln}{penColor2!20} % Color of negative area
\colorlet{fill3}{penColor3!20} % Fill
\colorlet{fill4}{penColor4!20} % Fill
\colorlet{fill5}{penColor5!20} % Fill
\colorlet{gridColor}{gray!50} % Color of grid in a plot

\newcommand{\surfaceColor}{violet}
\newcommand{\surfaceColorTwo}{redyellow}
\newcommand{\sliceColor}{greenyellow}




\pgfmathdeclarefunction{gauss}{2}{% gives gaussian
  \pgfmathparse{1/(#2*sqrt(2*pi))*exp(-((x-#1)^2)/(2*#2^2))}%
}


%%%%%%%%%%%%%
%% Vectors
%%%%%%%%%%%%%

%% Simple horiz vectors
\renewcommand{\vector}[1]{\left\langle #1\right\rangle}


%% %% Complex Horiz Vectors with angle brackets
%% \makeatletter
%% \renewcommand{\vector}[2][ , ]{\left\langle%
%%   \def\nextitem{\def\nextitem{#1}}%
%%   \@for \el:=#2\do{\nextitem\el}\right\rangle%
%% }
%% \makeatother

%% %% Vertical Vectors
%% \def\vector#1{\begin{bmatrix}\vecListA#1,,\end{bmatrix}}
%% \def\vecListA#1,{\if,#1,\else #1\cr \expandafter \vecListA \fi}

%%%%%%%%%%%%%
%% End of vectors
%%%%%%%%%%%%%

%\newcommand{\fullwidth}{}
%\newcommand{\normalwidth}{}



%% makes a snazzy t-chart for evaluating functions
%\newenvironment{tchart}{\rowcolors{2}{}{background!90!textColor}\array}{\endarray}

%%This is to help with formatting on future title pages.
\newenvironment{sectionOutcomes}{}{}



%% Flowchart stuff
%\tikzstyle{startstop} = [rectangle, rounded corners, minimum width=3cm, minimum height=1cm,text centered, draw=black]
%\tikzstyle{question} = [rectangle, minimum width=3cm, minimum height=1cm, text centered, draw=black]
%\tikzstyle{decision} = [trapezium, trapezium left angle=70, trapezium right angle=110, minimum width=3cm, minimum height=1cm, text centered, draw=black]
%\tikzstyle{question} = [rectangle, rounded corners, minimum width=3cm, minimum height=1cm,text centered, draw=black]
%\tikzstyle{process} = [rectangle, minimum width=3cm, minimum height=1cm, text centered, draw=black]
%\tikzstyle{decision} = [trapezium, trapezium left angle=70, trapezium right angle=110, minimum width=3cm, minimum height=1cm, text centered, draw=black]


\author{Jason Miller and Jim Talamo}
\license{Creative Commons 3.0 By-bC}


\outcome{}

\begin{document}
\begin{exercise}

%Two problems that require set up only, but each problem has a region where the inner and outer curve change and a region where there is an inner and outer curve (ex, r=2cos(2theta) and r=1, region 1 is region common to both curves, and region 2 is the region outside of r=1 but inside of r = 2 cos(2theta))

Consider the polar curves $r=3\cos(\theta)$ (shown in blue) and $r=1+\cos(\theta)$ (shown in red). 

\begin{image}  
  \begin{tikzpicture}  
    \begin{axis}[  
        xmin=-1.2,  
        xmax=3.5,  
        ymin=-2,  
        ymax=2,  
        axis lines=center,  
        xlabel=$x$,  
        ylabel=$y$,  
        every axis y label/.style={at=(current axis.above origin),anchor=south},  
        every axis x label/.style={at=(current axis.right of origin),anchor=west},axis on top
      ]  
       \addplot[data cs=polar, very thick, mark=none, fill=fill3,  domain=0:60, samples=180, smooth] (x, {1+cos(x)}) -- (axis cs:0,0)[green];
         \addplot[data cs=polar, very thick, mark=none, fill=fill3,  domain=-90:-60, samples=180, smooth] (x, {3*cos(x)}) -- (axis cs:0,0)[green];
         \addplot[data cs=polar, very thick, mark=none, fill=fill3,  domain=-60:0, samples=180, smooth] (x, {1+cos(x)}) -- (axis cs:0,0)[green];
       \addplot [data cs=polar, very thick, mark=none,fill=green,domain=60:90,samples=180,smooth] (x, {3*cos(x)});
      \addplot[data cs=polar,penColor2,domain=0:360,samples=360,smooth, thick] (x,{1+cos(x)}) ;
      \addplot[data cs=polar, penColor, domain=0:360, samples=360, smooth, thick] (x, {3*cos(x)});      
            \end{axis}  
  \end{tikzpicture}  
\end{image} 

We want to set up the area of the green region common to both polar
curves.  Call the intersection of the green region with the 1st
quadrant $S$. Once we find an integral or sum of integrals to express
this area, we can use symmetry and express the area of the entire
green region by multiplying by $2$.

What is the minimum number of integrals needed to express the area of
the region $S$?

\begin{multipleChoice}
\choice{1}
\choice[correct]{2}
\choice{more than 2}
\end{multipleChoice}


\begin{exercise}

Set up two integrals that gives the area of the region $S$. Use the
smaller $\theta$ values in the left integral.


\[
\int_{0}^{\answer{ \frac{\pi}{3} } } \answer{ \frac{1}{2}(1+\cos(\theta))^2   } \d \theta +  \int_{\answer{\frac{\pi}{3}}}^{\answer{\frac{\pi}{2}}} \answer{ \frac{1}{2}  (3\cos(\theta))^2 } \d \theta
\]

The area of the entire shaded region is obtained by multiplying these
by $2$.


\begin{hint}








We need to have some understanding of how the curves are traced out as $\theta$ varies. 

First we look at $r=3\cos(\theta)$. If we graph this on $r$ and $\theta$ axes we have


\begin{image}  
  \begin{tikzpicture}  
    \begin{axis}[  
        xmin=-.5,  
        xmax=3.5,  
        ymin=-3.5,  
        ymax=3.5,  
        axis lines=center,  
        xlabel=$\theta$,  
        ylabel=$r$,  
        every axis y label/.style={at=(current axis.above origin),anchor=south},  
        every axis x label/.style={at=(current axis.right of origin),anchor=west},  
       xtick={ 1.57, 3.14  },
       xticklabels={ $\frac{\pi}{2}$,  $\pi$ }, 
       ytick={-3, -2, -1, 1, 2, 3}
            ]  
      \addplot [ very thick, mark=none,domain=0:pi,smooth] {3*cos(deg(x))};
            \end{axis}  
  \end{tikzpicture}  
\end{image} 


As $\theta$ varies from $0$ to $\frac{\pi}{2}$, $r$ increases from $0$ to $1$. This means that to obtain the portion in the 1st quadrant we only need the interval $[0, \pi/2]$. 


Let's graph $r=1+\cos(\theta)$ on $r$ and $\theta$ axes. 

\begin{image}  
  \begin{tikzpicture}  
    \begin{axis}[  
        xmin=-.5,  
        xmax=6.5,  
        ymin=-.5,  
        ymax=2.5,  
        axis lines=center,  
        xlabel=$\theta$,  
        ylabel=$r$,  
        every axis y label/.style={at=(current axis.above origin),anchor=south},  
        every axis x label/.style={at=(current axis.right of origin),anchor=west},  
       xtick={  1.57,  3.14 , 4.71, 6.283 },
       xticklabels={ $\frac{\pi}{2}$, $\pi$, $\frac{3\pi}{2}$, $2\pi$ },
            ]  
      \addplot [ very thick, mark=none,domain=0:2*pi,smooth] {1+cos(deg(x))};
            \end{axis}  
  \end{tikzpicture}  
\end{image} 


As $\theta$ goes from $0$ to $\frac{\pi}{2}$ we see that $r$ decreases from $2$ to $1$. This corresponds of the cardioid $r=1+\cos(\theta)$ in the 1st quadrant. 


Now we need to find for which $\theta$ values the $r$ values of the two curves coincide. 

We set $r=3\cos(\theta)$ and $r=1+\cos(\theta)$ equal. 

We need to solve $3\cos(\theta)=1+\cos(\theta)$ for $\theta$. 


We get $\theta=\answer{\frac{\pi}{3}}$ (use $\theta$ values in $[0,\pi/2]$ since we are confining our attention to the 1st quadrant.)


\begin{image}  
  \begin{tikzpicture}  
    \begin{axis}[  
        xmin=-1.2,  
        xmax=3.5,  
        ymin=-2,  
        ymax=2.5,  
        axis lines=center,  
        xlabel=$x$,  
        ylabel=$y$,  
        every axis y label/.style={at=(current axis.above origin),anchor=south},  
        every axis x label/.style={at=(current axis.right of origin),anchor=west},axis on top
      ]  
       \addplot[data cs=polar, very thick, mark=none, fill=fill3,  domain=0:60, samples=180, smooth] (x, {1+cos(x)}) -- (axis cs:0,0)[green];
         \addplot[data cs=polar, very thick, mark=none, fill=fill3,  domain=-90:-60, samples=180, smooth] (x, {3*cos(x)}) -- (axis cs:0,0)[green];
         \addplot[data cs=polar, very thick, mark=none, fill=fill3,  domain=-60:0, samples=180, smooth] (x, {1+cos(x)}) -- (axis cs:0,0)[green];
       \addplot [data cs=polar, very thick, mark=none,fill=green,domain=60:90,samples=180,smooth] (x, {3*cos(x)});
      \addplot[data cs=polar,penColor2,domain=0:360,samples=360,smooth, thick] (x,{1+cos(x)}) ;
      \addplot[data cs=polar, penColor, domain=0:360, samples=360, smooth, thick] (x, {3*cos(x)});      
       \draw[very thick, black] (axis cs:0,0) -- (axis cs:3,5.2) node [pos=.36, above, rotate=60, black] {$\theta=\pi/3$};
        \draw[very thick, purple] (axis cs:0,0) -- (axis cs: .85, 2.45) node [pos=.81, above, rotate=75, purple] {$\theta=\theta_{2}$};
       \draw[very thick, orange] (axis cs:0,0) -- (axis cs: 3, 1) node [pos=.74, above, rotate=28, orange] {$\theta=\theta_{1}$};
            \end{axis}  
  \end{tikzpicture}  
\end{image}


Suppose we take a ray $\theta=\theta_{1}$ between $0$ and $\frac{\pi}{3}$. We see where such an arbitrary ray hits the boundary of our region. So we see that $r_{outer}=1+\cos(\theta)$ ( the blue curve) and $r_{inner}=0$. 

However once we move past $\theta=\frac{\pi}{3}$, the outer radius changes. That is, $r_{outer}=3\cos(\theta)$ (the red curve) while $r_{inner}=0$.

We can summarize in the following picture.

\begin{image}  
  \begin{tikzpicture}  
    \begin{axis}[  
        xmin=-1.2,  
        xmax=3.5,  
        ymin=-2,  
        ymax=2,  
        axis lines=center,  
        xlabel=$x$,  
        ylabel=$y$,  
        every axis y label/.style={at=(current axis.above origin),anchor=south},  
        every axis x label/.style={at=(current axis.right of origin),anchor=west},axis on top
      ]  
       \addplot[data cs=polar, very thick, mark=none, fill=fill3,  domain=0:60, samples=180, smooth] (x, {1+cos(x)}) -- (axis cs:0,0)[penColor2!50];
       \addplot [data cs=polar, very thick, mark=none,fill=penColor!50,domain=60:90,samples=180,smooth] (x, {3*cos(x)});
      \addplot[data cs=polar,penColor2,domain=0:360,samples=360,smooth, thick] (x,{1+cos(x)}) ;
      \addplot[data cs=polar, penColor, domain=0:360, samples=360, smooth, thick] (x, {3*cos(x)});      
            \end{axis}  
  \end{tikzpicture}  
\end{image} 

The red area is given by $\int_{0}^{\answer{\pi/3}} \frac{1}{2}\left(\answer{1+\cos(\theta)}\right)^2 \d \theta$ and the blue area is given by 
$\int_{\answer{\pi/3}}^{\answer{\pi/2}} \frac{1}{2}\left(\answer{3\cos(\theta)}\right)^2 \d \theta$






\end{hint}

\begin{exercise}

Now consider a new shaded region. 

\begin{image}  
  \begin{tikzpicture}  
    \begin{axis}[  
        xmin=-1.2,  
        xmax=3.5,  
        ymin=-2,  
        ymax=2,  
        axis lines=center,  
        xlabel=$x$,  
        ylabel=$y$,  
        every axis y label/.style={at=(current axis.above origin),anchor=south},  
        every axis x label/.style={at=(current axis.right of origin),anchor=west},axis on top
      ]  
       \addplot[data cs=polar, very thick, mark=none, fill=fill3,  domain=60:180, samples=180, smooth] (x, {1+cos(x)}) -- (axis cs:0,0)[green];
         \addplot[data cs=polar, very thick, mark=none, fill=white,  domain=60:90, samples=180, smooth] (x, {3*cos(x)}) -- (axis cs:0,0);
      
      \addplot[data cs=polar,penColor2,domain=0:360,samples=360,smooth, thick] (x,{1+cos(x)}) ;
      \addplot[data cs=polar, penColor, domain=0:360, samples=360, smooth, thick] (x, {3*cos(x)});      
            \end{axis}  
  \end{tikzpicture}  
\end{image} 


Let's set up an integral(s) for the green shaded region. 

What is the minimum number of integrals needed to express the area of the region $S$?

\begin{multipleChoice}
\choice{1}
\choice[correct]{2}
\choice{more than 2}
\end{multipleChoice}

\begin{exercise}

Set up two integrals that gives the area of the green shaded region. Again use smaller $\theta$ values for the left  integral. 


\[
\int_{\answer{\frac{\pi}{3}}}^{\answer{ \frac{\pi}{2} } } \answer{ \frac{1}{2}\left([1+\cos(\theta)]^2 -[3\cos(\theta)]^2\right)  } \d \theta +  \int_{\answer{\frac{\pi}{2}}}^{\answer{\pi}} \answer{ \frac{1}{2} \left([1+\cos(\theta)]^2\right) } \d \theta
\]



\begin{hint}

Now we need to find for which $\theta$ values the $r$ values of the two curves coincide. 


We need to solve $3\cos(\theta)=1+\cos(\theta)$ for $\theta$. 


We get $\theta=\answer{\frac{\pi}{3}}$ as before. 

Suppose we take a ray $\theta=\theta_{1}$ betwen $\frac{\pi}{3}$ and $\frac{\pi}{2}$. Then we see that $r_{outer}=1+\cos(\theta)$ ( the blue curve) and $r_{inner}=3\cos(\theta)$ (the red curve). 


However once we move past $\theta=\frac{\pi}{2}$, the inner radius changes. That is, $r_{outer}=1+\cos(\theta)$ while $r_{inner}=0$. 


\begin{image}  
  \begin{tikzpicture}  
    \begin{axis}[  
        xmin=-.8,  
        xmax=3.5,  
        ymin=-1.5,  
        ymax=2.6,  
        axis lines=center,  
        xlabel=$x$,  
        ylabel=$y$,  
        every axis y label/.style={at=(current axis.above origin),anchor=south},  
        every axis x label/.style={at=(current axis.right of origin),anchor=west},axis on top
      ]  
       \addplot[data cs=polar, very thick, mark=none, fill=fill3,  domain=60:180, samples=180, smooth] (x, {1+cos(x)}) -- (axis cs:0,0)[green];
         \addplot[data cs=polar, very thick, mark=none, fill=white,  domain=60:90, samples=180, smooth] (x, {3*cos(x)}) -- (axis cs:0,0);
      
      \addplot[data cs=polar,penColor2,domain=0:360,samples=360,smooth, thick] (x,{1+cos(x)}) ;
      \addplot[data cs=polar, penColor, domain=0:360, samples=360, smooth, thick] (x, {3*cos(x)});      
   
       \draw[very thick, black] (axis cs:0,0) -- (axis cs:3,5.2) node [pos=.36, above, rotate=60, black] {$\theta=\pi/3$};
          \draw[very thick, orange] (axis cs:0,0) -- (axis cs: .78, 2.45) node [pos=.85, above, rotate=75, orange] {$\theta=\theta_{1}$};
           \draw[very thick, purple] (axis cs:0,0) -- (axis cs: -.44, 2.4) node [pos=.85, below, rotate=-75, purple] {$\theta=\theta_{2}$};
            \end{axis}  
  \end{tikzpicture}  
\end{image} 




\end{hint}

\end{exercise}
\end{exercise}
\end{exercise}
\end{exercise}
\end{document}
