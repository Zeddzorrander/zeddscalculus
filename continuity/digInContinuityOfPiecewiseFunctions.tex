\documentclass{ximera}

%\usepackage{todonotes}
%\usepackage{mathtools} %% Required for wide table Curl and Greens
%\usepackage{cuted} %% Required for wide table Curl and Greens
\newcommand{\todo}{}

\usepackage{esint} % for \oiint
\ifxake%%https://math.meta.stackexchange.com/questions/9973/how-do-you-render-a-closed-surface-double-integral
\renewcommand{\oiint}{{\large\bigcirc}\kern-1.56em\iint}
\fi


\graphicspath{
  {./}
  {ximeraTutorial/}
  {basicPhilosophy/}
  {functionsOfSeveralVariables/}
  {normalVectors/}
  {lagrangeMultipliers/}
  {vectorFields/}
  {greensTheorem/}
  {shapeOfThingsToCome/}
  {dotProducts/}
  {partialDerivativesAndTheGradientVector/}
  {../productAndQuotientRules/exercises/}
  {../motionAndPathsInSpace/exercises/}
  {../normalVectors/exercisesParametricPlots/}
  {../continuityOfFunctionsOfSeveralVariables/exercises/}
  {../partialDerivativesAndTheGradientVector/exercises/}
  {../directionalDerivativeAndChainRule/exercises/}
  {../commonCoordinates/exercisesCylindricalCoordinates/}
  {../commonCoordinates/exercisesSphericalCoordinates/}
  {../greensTheorem/exercisesCurlAndLineIntegrals/}
  {../greensTheorem/exercisesDivergenceAndLineIntegrals/}
  {../shapeOfThingsToCome/exercisesDivergenceTheorem/}
  {../greensTheorem/}
  {../shapeOfThingsToCome/}
  {../separableDifferentialEquations/exercises/}
  {vectorFields/}
}

\newcommand{\mooculus}{\textsf{\textbf{MOOC}\textnormal{\textsf{ULUS}}}}

\usepackage{tkz-euclide}\usepackage{tikz}
\usepackage{tikz-cd}
\usetikzlibrary{arrows}
\tikzset{>=stealth,commutative diagrams/.cd,
  arrow style=tikz,diagrams={>=stealth}} %% cool arrow head
\tikzset{shorten <>/.style={ shorten >=#1, shorten <=#1 } } %% allows shorter vectors

\usetikzlibrary{backgrounds} %% for boxes around graphs
\usetikzlibrary{shapes,positioning}  %% Clouds and stars
\usetikzlibrary{matrix} %% for matrix
\usepgfplotslibrary{polar} %% for polar plots
\usepgfplotslibrary{fillbetween} %% to shade area between curves in TikZ
\usetkzobj{all}
\usepackage[makeroom]{cancel} %% for strike outs
%\usepackage{mathtools} %% for pretty underbrace % Breaks Ximera
%\usepackage{multicol}
\usepackage{pgffor} %% required for integral for loops



%% http://tex.stackexchange.com/questions/66490/drawing-a-tikz-arc-specifying-the-center
%% Draws beach ball
\tikzset{pics/carc/.style args={#1:#2:#3}{code={\draw[pic actions] (#1:#3) arc(#1:#2:#3);}}}



\usepackage{array}
\setlength{\extrarowheight}{+.1cm}
\newdimen\digitwidth
\settowidth\digitwidth{9}
\def\divrule#1#2{
\noalign{\moveright#1\digitwidth
\vbox{\hrule width#2\digitwidth}}}





\newcommand{\RR}{\mathbb R}
\newcommand{\R}{\mathbb R}
\newcommand{\N}{\mathbb N}
\newcommand{\Z}{\mathbb Z}

\newcommand{\sagemath}{\textsf{SageMath}}


%\renewcommand{\d}{\,d\!}
\renewcommand{\d}{\mathop{}\!d}
\newcommand{\dd}[2][]{\frac{\d #1}{\d #2}}
\newcommand{\pp}[2][]{\frac{\partial #1}{\partial #2}}
\renewcommand{\l}{\ell}
\newcommand{\ddx}{\frac{d}{\d x}}

\newcommand{\zeroOverZero}{\ensuremath{\boldsymbol{\tfrac{0}{0}}}}
\newcommand{\inftyOverInfty}{\ensuremath{\boldsymbol{\tfrac{\infty}{\infty}}}}
\newcommand{\zeroOverInfty}{\ensuremath{\boldsymbol{\tfrac{0}{\infty}}}}
\newcommand{\zeroTimesInfty}{\ensuremath{\small\boldsymbol{0\cdot \infty}}}
\newcommand{\inftyMinusInfty}{\ensuremath{\small\boldsymbol{\infty - \infty}}}
\newcommand{\oneToInfty}{\ensuremath{\boldsymbol{1^\infty}}}
\newcommand{\zeroToZero}{\ensuremath{\boldsymbol{0^0}}}
\newcommand{\inftyToZero}{\ensuremath{\boldsymbol{\infty^0}}}



\newcommand{\numOverZero}{\ensuremath{\boldsymbol{\tfrac{\#}{0}}}}
\newcommand{\dfn}{\textbf}
%\newcommand{\unit}{\,\mathrm}
\newcommand{\unit}{\mathop{}\!\mathrm}
\newcommand{\eval}[1]{\bigg[ #1 \bigg]}
\newcommand{\seq}[1]{\left( #1 \right)}
\renewcommand{\epsilon}{\varepsilon}
\renewcommand{\phi}{\varphi}


\renewcommand{\iff}{\Leftrightarrow}

\DeclareMathOperator{\arccot}{arccot}
\DeclareMathOperator{\arcsec}{arcsec}
\DeclareMathOperator{\arccsc}{arccsc}
\DeclareMathOperator{\si}{Si}
\DeclareMathOperator{\scal}{scal}
\DeclareMathOperator{\sign}{sign}


%% \newcommand{\tightoverset}[2]{% for arrow vec
%%   \mathop{#2}\limits^{\vbox to -.5ex{\kern-0.75ex\hbox{$#1$}\vss}}}
\newcommand{\arrowvec}[1]{{\overset{\rightharpoonup}{#1}}}
%\renewcommand{\vec}[1]{\arrowvec{\mathbf{#1}}}
\renewcommand{\vec}[1]{{\overset{\boldsymbol{\rightharpoonup}}{\mathbf{#1}}}\hspace{0in}}

\newcommand{\point}[1]{\left(#1\right)} %this allows \vector{ to be changed to \vector{ with a quick find and replace
\newcommand{\pt}[1]{\mathbf{#1}} %this allows \vec{ to be changed to \vec{ with a quick find and replace
\newcommand{\Lim}[2]{\lim_{\point{#1} \to \point{#2}}} %Bart, I changed this to point since I want to use it.  It runs through both of the exercise and exerciseE files in limits section, which is why it was in each document to start with.

\DeclareMathOperator{\proj}{\mathbf{proj}}
\newcommand{\veci}{{\boldsymbol{\hat{\imath}}}}
\newcommand{\vecj}{{\boldsymbol{\hat{\jmath}}}}
\newcommand{\veck}{{\boldsymbol{\hat{k}}}}
\newcommand{\vecl}{\vec{\boldsymbol{\l}}}
\newcommand{\uvec}[1]{\mathbf{\hat{#1}}}
\newcommand{\utan}{\mathbf{\hat{t}}}
\newcommand{\unormal}{\mathbf{\hat{n}}}
\newcommand{\ubinormal}{\mathbf{\hat{b}}}

\newcommand{\dotp}{\bullet}
\newcommand{\cross}{\boldsymbol\times}
\newcommand{\grad}{\boldsymbol\nabla}
\newcommand{\divergence}{\grad\dotp}
\newcommand{\curl}{\grad\cross}
%\DeclareMathOperator{\divergence}{divergence}
%\DeclareMathOperator{\curl}[1]{\grad\cross #1}
\newcommand{\lto}{\mathop{\longrightarrow\,}\limits}

\renewcommand{\bar}{\overline}

\colorlet{textColor}{black}
\colorlet{background}{white}
\colorlet{penColor}{blue!50!black} % Color of a curve in a plot
\colorlet{penColor2}{red!50!black}% Color of a curve in a plot
\colorlet{penColor3}{red!50!blue} % Color of a curve in a plot
\colorlet{penColor4}{green!50!black} % Color of a curve in a plot
\colorlet{penColor5}{orange!80!black} % Color of a curve in a plot
\colorlet{penColor6}{yellow!70!black} % Color of a curve in a plot
\colorlet{fill1}{penColor!20} % Color of fill in a plot
\colorlet{fill2}{penColor2!20} % Color of fill in a plot
\colorlet{fillp}{fill1} % Color of positive area
\colorlet{filln}{penColor2!20} % Color of negative area
\colorlet{fill3}{penColor3!20} % Fill
\colorlet{fill4}{penColor4!20} % Fill
\colorlet{fill5}{penColor5!20} % Fill
\colorlet{gridColor}{gray!50} % Color of grid in a plot

\newcommand{\surfaceColor}{violet}
\newcommand{\surfaceColorTwo}{redyellow}
\newcommand{\sliceColor}{greenyellow}




\pgfmathdeclarefunction{gauss}{2}{% gives gaussian
  \pgfmathparse{1/(#2*sqrt(2*pi))*exp(-((x-#1)^2)/(2*#2^2))}%
}


%%%%%%%%%%%%%
%% Vectors
%%%%%%%%%%%%%

%% Simple horiz vectors
\renewcommand{\vector}[1]{\left\langle #1\right\rangle}


%% %% Complex Horiz Vectors with angle brackets
%% \makeatletter
%% \renewcommand{\vector}[2][ , ]{\left\langle%
%%   \def\nextitem{\def\nextitem{#1}}%
%%   \@for \el:=#2\do{\nextitem\el}\right\rangle%
%% }
%% \makeatother

%% %% Vertical Vectors
%% \def\vector#1{\begin{bmatrix}\vecListA#1,,\end{bmatrix}}
%% \def\vecListA#1,{\if,#1,\else #1\cr \expandafter \vecListA \fi}

%%%%%%%%%%%%%
%% End of vectors
%%%%%%%%%%%%%

%\newcommand{\fullwidth}{}
%\newcommand{\normalwidth}{}



%% makes a snazzy t-chart for evaluating functions
%\newenvironment{tchart}{\rowcolors{2}{}{background!90!textColor}\array}{\endarray}

%%This is to help with formatting on future title pages.
\newenvironment{sectionOutcomes}{}{}



%% Flowchart stuff
%\tikzstyle{startstop} = [rectangle, rounded corners, minimum width=3cm, minimum height=1cm,text centered, draw=black]
%\tikzstyle{question} = [rectangle, minimum width=3cm, minimum height=1cm, text centered, draw=black]
%\tikzstyle{decision} = [trapezium, trapezium left angle=70, trapezium right angle=110, minimum width=3cm, minimum height=1cm, text centered, draw=black]
%\tikzstyle{question} = [rectangle, rounded corners, minimum width=3cm, minimum height=1cm,text centered, draw=black]
%\tikzstyle{process} = [rectangle, minimum width=3cm, minimum height=1cm, text centered, draw=black]
%\tikzstyle{decision} = [trapezium, trapezium left angle=70, trapezium right angle=110, minimum width=3cm, minimum height=1cm, text centered, draw=black]


\outcome{Identify where a function is, and is not, continuous.}
\outcome{Understand the connection between continuity of a function and
  the value of a limit.}
\outcome{Make a piecewise function continuous.}

\title[Dig-In:]{Continuity of piecewise functions}

\begin{document}
\begin{abstract}
Here we use limits to check whether piecewise functions are continuous.
\end{abstract}
\maketitle
Before we start talking about continuity of piecewise functions, let's remind ourselves of all famous functions that are continuous on their domains.
\begin{theorem}[Continuity of Famous Functions]\index{continuity of famous functions}\label{theorem:continuity}
The following functions are continuous on their natural domains:
\begin{itemize}
\item Constant functions \index{constant function}
\item Polynomials  \index{polynomials}
\item Rational functions  \index{rational function}
\item Power functions \index{power function}
\item Exponential functions \index{exponential function} 
\item Logarithmic functions \index{logarithmic function} 
\item Trigonometric functions\index{trigonometric function}  
\item Inverse trigonometric functions \index{inverse trigonometric functions}    
\end{itemize}
In essence, we are saying that the functions listed above are
continuous wherever they are defined.

We proved continuity of polynomials earlier using the Sum Law, Product
Law and continuity of power functions.


We proved continuity of rational functions earlier using the Quotient
Law and continuity of polynomials.


We can prove continuity of the remaining four trig functions using the
Quotient Law and continuity of sine and cosine functions.


Since a continuous function and its inverse have ``unbroken'' graphs,
it follows that an inverse of a continuous function is continuous on
its domain.


This implies that inverse trig functions are continuous on their domains. 


Using the Limit Laws we can prove that given two functions, both
continuous on the same interval, then their sum, difference, product,
and quotient (where defined) are also continuous on the same interval
(where defined).
\end{theorem}
In this section we will work a couple of examples involving limits,
continuity and piecewise functions.

\begin{example}
Consider the following piecewise defined function
\[
f(x) = 
\begin{cases}
  \frac{x}{x-1} &\text{if $x<0$,}\\
  e^{-x} + c &\text{if $x\ge 0$}.
\end{cases}
\]
Find the constant $c$ so that $f$ is continuous at $x=0$.
\begin{explanation}
  To find $c$ such that $f$ is continuous at $x=0$, we need to find
  $c$ such that
    \[
 \lim_{x\to 0} f(x) = f\Bigl(\answer[given]{0}\Bigr).
  \]
 % \[
 % \lim_{x\to 0^-} f(x) = \lim_{x\to 0^+}f(x) = f(\answer[given]{0}).
 % \]
  In this case, in order to compute the limit, we will have to compute two one-sided limits, since the expression for  $f(x)$ if $x<0$ is  different from the expression for $f(x)$ if $x>0$.
  So,
  \begin{align*}
    \lim_{x\to 0^-} f(x) &= \lim_{x\to 0^-}\answer[given]{\frac{x}{x-1}}\\
    &= \frac{\answer[given]{0}}{-1}\\
    &=\answer[given]{0}.
  \end{align*}
and
  \begin{align*}
    \lim_{x\to 0^+} f(x) &= \lim_{x\to 0^+}\left(e^{-x}+c\right)\\
    &= e^{\answer[given]{0}} + c\\
    &= \answer[given]{1+ c}
  \end{align*}
  In order for $f$ to be continuous, the limit $ \lim_{x\to 0} f(x)$ has to exist. This means that
  \[
  1 + c = 0\qquad\text{so}\qquad c = \answer[given]{-1}.
  \]
  So, if $c=-1$, the limit $ \lim_{x\to 0} f(x)=0$. Now, we have find the value

\[
  f(0) = \answer[given]{0}.
  \]
  Therefore,
   \[
 \lim_{x\to 0} f(x) = f\Bigl(0\Bigr)\checkmark,
  \]
  which proves that $f$ is continuous at $x=0$.
\end{explanation}
\end{example}


Consider the next, more challenging example.

\begin{example}
Consider the following piecewise defined function
\[
f(x) = 
\begin{cases}
  x+4 &\text{if $x<1$,}\\
  ax^2+bx+2 &\text{if $1\le x< 3$,}\\
  6x+a-b &\text{if $x\ge 3$.}
\end{cases}
\]
Find the constants $a$ and $b$ so that $f$ is continuous at both $x=1$ and $x=3$.
\begin{explanation}
This problem is more challenging because we have more
unknowns. However, be brave intrepid mathematician.  To find $a$ and
$b$ that make $f$ continuous at $x=1$, we need to find $a$ and $b$
such that
\[
\lim_{x\to 1} f(x) =f(\answer[given]{1}).
\]
%\[
%\lim_{x\to 1^-} f(x) = \lim_{x\to 1^+}f(x)=f(\answer[given]{1}).
%\]
Since $f(1)=a+b+2$, it follows that 
\[
\lim_{x\to 1} f(x) =a+b+2.
\]
We have to compute two one-sided limits, since the expression for  $f(x)$ if $x<1$ is  different from the expression for $f(x)$ if $x>1$.
Looking at the limit from the left, we have
\begin{align*}
  \lim_{x\to 1^-} f(x) &= \lim_{x\to 1^-} \left(\answer[given]{x+4}\right) \\
  &=\answer[given]{5}.
\end{align*}
Looking at the limit from the right, we have
\begin{align*}
  \lim_{x\to 1^+} f(x) &= \lim_{x\to 1^+} \left(ax^2+bx+2\right) \\
  &= \answer[given]{a+b+2}.
\end{align*}
Hence, for the limit $\lim_{x\to 1} f(x)$ to exist, we must have that 
\begin{align*}
  5 &= a+b+2\\ or\\
  a+b &= \answer[given]{3}.
\end{align*}
Also, 
Hmmmm. More work needs to be done.

To find $a$ and $b$ that make $f$ is continuous at $x=3$, we need to
find $a$ and $b$ such that
\[
\lim_{x\to 3} f(x) =f(3).
\]
Since $f(3)=18+a-b$, it follows that 
\[
\lim_{x\to 3} f(x) =18+a-b.
\]
Looking at the limit from the left, we have
\begin{align*}
  \lim_{x\to 3^-} f(x) &= \lim_{x\to 3^-} \left(ax^2+bx+2\right) \\
  &=a\cdot 9 + b\cdot 3 + 2.
\end{align*}
Looking at the limit from the right, we have
\begin{align*}
  \lim_{x\to 3^+} f(x) &= \lim_{x\to 3^+} \left(6x+a-b\right) \\
  &= \answer[given]{18+a-b}.
\end{align*}
Hence 
\begin{align*}
  9a + 3b + 2 &= 18+a-b\\
  8a + 4b -16 &= 0\\
  2a + b -4 &= 0
\end{align*}
So now we have two equations and two unknowns:
\[
 a+b=3 \qquad\text{and}\qquad 2a + b  = 4.
 \]
 Set $b = 3-a$ and write
 \begin{align*}
   2a+3-a&=4 \\
  a &= 1,
 \end{align*}
 hence
 \[
 a=\answer[given]{1}\qquad\text{and so}\qquad b =\answer[given]{2}.
 \]
 Let's check, so now plugging in values for both $a$ and $b$ we find
 \[
 f(x) = 
 \begin{cases}
   x+4 &\text{if $x<1$,}\\
   \answer[given]{x^2+2x+2} &\text{if $1\le x< 3$,}\\
   \answer[given]{6x-1} &\text{if $x\ge 3$.}
\end{cases}
 \]
 Now
 \[
 \lim_{x\to 1^-} f(x) =\lim_{x\to 1^+}f(x) =f(1) =  5,
 \]
 and
 \[
 \lim_{x\to 3^-} f(x) =\lim_{x\to 3^+}f(x) =f(3) = 17.  
 \]
 So setting $a= \answer[given]{1}$ and $b=\answer[given]{2}$ makes $f$ continuous at $x=1$ and $x=3$.
 \begin{onlineOnly}
   We can confirm our results by looking at the graph of $y=f(x)$:
   \[
   \graph[xmin=-10,xmax=10,ymin=0,ymax=30]{y=x+4\left\{x<1\right\},y=x^2+2x+2\left\{1\leq x<3\right\},y=6x-1\left\{3 \leq x\right\}} 
   \]
 \end{onlineOnly}
\end{explanation}
\end{example}



\end{document}
