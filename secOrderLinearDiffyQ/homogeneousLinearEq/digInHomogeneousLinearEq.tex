\documentclass{ximera}

%\usepackage{todonotes}
%\usepackage{mathtools} %% Required for wide table Curl and Greens
%\usepackage{cuted} %% Required for wide table Curl and Greens
\newcommand{\todo}{}

\usepackage{esint} % for \oiint
\ifxake%%https://math.meta.stackexchange.com/questions/9973/how-do-you-render-a-closed-surface-double-integral
\renewcommand{\oiint}{{\large\bigcirc}\kern-1.56em\iint}
\fi


\graphicspath{
  {./}
  {ximeraTutorial/}
  {basicPhilosophy/}
  {functionsOfSeveralVariables/}
  {normalVectors/}
  {lagrangeMultipliers/}
  {vectorFields/}
  {greensTheorem/}
  {shapeOfThingsToCome/}
  {dotProducts/}
  {partialDerivativesAndTheGradientVector/}
  {../productAndQuotientRules/exercises/}
  {../motionAndPathsInSpace/exercises/}
  {../normalVectors/exercisesParametricPlots/}
  {../continuityOfFunctionsOfSeveralVariables/exercises/}
  {../partialDerivativesAndTheGradientVector/exercises/}
  {../directionalDerivativeAndChainRule/exercises/}
  {../commonCoordinates/exercisesCylindricalCoordinates/}
  {../commonCoordinates/exercisesSphericalCoordinates/}
  {../greensTheorem/exercisesCurlAndLineIntegrals/}
  {../greensTheorem/exercisesDivergenceAndLineIntegrals/}
  {../shapeOfThingsToCome/exercisesDivergenceTheorem/}
  {../greensTheorem/}
  {../shapeOfThingsToCome/}
  {../separableDifferentialEquations/exercises/}
  {vectorFields/}
}

\newcommand{\mooculus}{\textsf{\textbf{MOOC}\textnormal{\textsf{ULUS}}}}

\usepackage{tkz-euclide}\usepackage{tikz}
\usepackage{tikz-cd}
\usetikzlibrary{arrows}
\tikzset{>=stealth,commutative diagrams/.cd,
  arrow style=tikz,diagrams={>=stealth}} %% cool arrow head
\tikzset{shorten <>/.style={ shorten >=#1, shorten <=#1 } } %% allows shorter vectors

\usetikzlibrary{backgrounds} %% for boxes around graphs
\usetikzlibrary{shapes,positioning}  %% Clouds and stars
\usetikzlibrary{matrix} %% for matrix
\usepgfplotslibrary{polar} %% for polar plots
\usepgfplotslibrary{fillbetween} %% to shade area between curves in TikZ
\usetkzobj{all}
\usepackage[makeroom]{cancel} %% for strike outs
%\usepackage{mathtools} %% for pretty underbrace % Breaks Ximera
%\usepackage{multicol}
\usepackage{pgffor} %% required for integral for loops



%% http://tex.stackexchange.com/questions/66490/drawing-a-tikz-arc-specifying-the-center
%% Draws beach ball
\tikzset{pics/carc/.style args={#1:#2:#3}{code={\draw[pic actions] (#1:#3) arc(#1:#2:#3);}}}



\usepackage{array}
\setlength{\extrarowheight}{+.1cm}
\newdimen\digitwidth
\settowidth\digitwidth{9}
\def\divrule#1#2{
\noalign{\moveright#1\digitwidth
\vbox{\hrule width#2\digitwidth}}}





\newcommand{\RR}{\mathbb R}
\newcommand{\R}{\mathbb R}
\newcommand{\N}{\mathbb N}
\newcommand{\Z}{\mathbb Z}

\newcommand{\sagemath}{\textsf{SageMath}}


%\renewcommand{\d}{\,d\!}
\renewcommand{\d}{\mathop{}\!d}
\newcommand{\dd}[2][]{\frac{\d #1}{\d #2}}
\newcommand{\pp}[2][]{\frac{\partial #1}{\partial #2}}
\renewcommand{\l}{\ell}
\newcommand{\ddx}{\frac{d}{\d x}}

\newcommand{\zeroOverZero}{\ensuremath{\boldsymbol{\tfrac{0}{0}}}}
\newcommand{\inftyOverInfty}{\ensuremath{\boldsymbol{\tfrac{\infty}{\infty}}}}
\newcommand{\zeroOverInfty}{\ensuremath{\boldsymbol{\tfrac{0}{\infty}}}}
\newcommand{\zeroTimesInfty}{\ensuremath{\small\boldsymbol{0\cdot \infty}}}
\newcommand{\inftyMinusInfty}{\ensuremath{\small\boldsymbol{\infty - \infty}}}
\newcommand{\oneToInfty}{\ensuremath{\boldsymbol{1^\infty}}}
\newcommand{\zeroToZero}{\ensuremath{\boldsymbol{0^0}}}
\newcommand{\inftyToZero}{\ensuremath{\boldsymbol{\infty^0}}}



\newcommand{\numOverZero}{\ensuremath{\boldsymbol{\tfrac{\#}{0}}}}
\newcommand{\dfn}{\textbf}
%\newcommand{\unit}{\,\mathrm}
\newcommand{\unit}{\mathop{}\!\mathrm}
\newcommand{\eval}[1]{\bigg[ #1 \bigg]}
\newcommand{\seq}[1]{\left( #1 \right)}
\renewcommand{\epsilon}{\varepsilon}
\renewcommand{\phi}{\varphi}


\renewcommand{\iff}{\Leftrightarrow}

\DeclareMathOperator{\arccot}{arccot}
\DeclareMathOperator{\arcsec}{arcsec}
\DeclareMathOperator{\arccsc}{arccsc}
\DeclareMathOperator{\si}{Si}
\DeclareMathOperator{\scal}{scal}
\DeclareMathOperator{\sign}{sign}


%% \newcommand{\tightoverset}[2]{% for arrow vec
%%   \mathop{#2}\limits^{\vbox to -.5ex{\kern-0.75ex\hbox{$#1$}\vss}}}
\newcommand{\arrowvec}[1]{{\overset{\rightharpoonup}{#1}}}
%\renewcommand{\vec}[1]{\arrowvec{\mathbf{#1}}}
\renewcommand{\vec}[1]{{\overset{\boldsymbol{\rightharpoonup}}{\mathbf{#1}}}\hspace{0in}}

\newcommand{\point}[1]{\left(#1\right)} %this allows \vector{ to be changed to \vector{ with a quick find and replace
\newcommand{\pt}[1]{\mathbf{#1}} %this allows \vec{ to be changed to \vec{ with a quick find and replace
\newcommand{\Lim}[2]{\lim_{\point{#1} \to \point{#2}}} %Bart, I changed this to point since I want to use it.  It runs through both of the exercise and exerciseE files in limits section, which is why it was in each document to start with.

\DeclareMathOperator{\proj}{\mathbf{proj}}
\newcommand{\veci}{{\boldsymbol{\hat{\imath}}}}
\newcommand{\vecj}{{\boldsymbol{\hat{\jmath}}}}
\newcommand{\veck}{{\boldsymbol{\hat{k}}}}
\newcommand{\vecl}{\vec{\boldsymbol{\l}}}
\newcommand{\uvec}[1]{\mathbf{\hat{#1}}}
\newcommand{\utan}{\mathbf{\hat{t}}}
\newcommand{\unormal}{\mathbf{\hat{n}}}
\newcommand{\ubinormal}{\mathbf{\hat{b}}}

\newcommand{\dotp}{\bullet}
\newcommand{\cross}{\boldsymbol\times}
\newcommand{\grad}{\boldsymbol\nabla}
\newcommand{\divergence}{\grad\dotp}
\newcommand{\curl}{\grad\cross}
%\DeclareMathOperator{\divergence}{divergence}
%\DeclareMathOperator{\curl}[1]{\grad\cross #1}
\newcommand{\lto}{\mathop{\longrightarrow\,}\limits}

\renewcommand{\bar}{\overline}

\colorlet{textColor}{black}
\colorlet{background}{white}
\colorlet{penColor}{blue!50!black} % Color of a curve in a plot
\colorlet{penColor2}{red!50!black}% Color of a curve in a plot
\colorlet{penColor3}{red!50!blue} % Color of a curve in a plot
\colorlet{penColor4}{green!50!black} % Color of a curve in a plot
\colorlet{penColor5}{orange!80!black} % Color of a curve in a plot
\colorlet{penColor6}{yellow!70!black} % Color of a curve in a plot
\colorlet{fill1}{penColor!20} % Color of fill in a plot
\colorlet{fill2}{penColor2!20} % Color of fill in a plot
\colorlet{fillp}{fill1} % Color of positive area
\colorlet{filln}{penColor2!20} % Color of negative area
\colorlet{fill3}{penColor3!20} % Fill
\colorlet{fill4}{penColor4!20} % Fill
\colorlet{fill5}{penColor5!20} % Fill
\colorlet{gridColor}{gray!50} % Color of grid in a plot

\newcommand{\surfaceColor}{violet}
\newcommand{\surfaceColorTwo}{redyellow}
\newcommand{\sliceColor}{greenyellow}




\pgfmathdeclarefunction{gauss}{2}{% gives gaussian
  \pgfmathparse{1/(#2*sqrt(2*pi))*exp(-((x-#1)^2)/(2*#2^2))}%
}


%%%%%%%%%%%%%
%% Vectors
%%%%%%%%%%%%%

%% Simple horiz vectors
\renewcommand{\vector}[1]{\left\langle #1\right\rangle}


%% %% Complex Horiz Vectors with angle brackets
%% \makeatletter
%% \renewcommand{\vector}[2][ , ]{\left\langle%
%%   \def\nextitem{\def\nextitem{#1}}%
%%   \@for \el:=#2\do{\nextitem\el}\right\rangle%
%% }
%% \makeatother

%% %% Vertical Vectors
%% \def\vector#1{\begin{bmatrix}\vecListA#1,,\end{bmatrix}}
%% \def\vecListA#1,{\if,#1,\else #1\cr \expandafter \vecListA \fi}

%%%%%%%%%%%%%
%% End of vectors
%%%%%%%%%%%%%

%\newcommand{\fullwidth}{}
%\newcommand{\normalwidth}{}



%% makes a snazzy t-chart for evaluating functions
%\newenvironment{tchart}{\rowcolors{2}{}{background!90!textColor}\array}{\endarray}

%%This is to help with formatting on future title pages.
\newenvironment{sectionOutcomes}{}{}



%% Flowchart stuff
%\tikzstyle{startstop} = [rectangle, rounded corners, minimum width=3cm, minimum height=1cm,text centered, draw=black]
%\tikzstyle{question} = [rectangle, minimum width=3cm, minimum height=1cm, text centered, draw=black]
%\tikzstyle{decision} = [trapezium, trapezium left angle=70, trapezium right angle=110, minimum width=3cm, minimum height=1cm, text centered, draw=black]
%\tikzstyle{question} = [rectangle, rounded corners, minimum width=3cm, minimum height=1cm,text centered, draw=black]
%\tikzstyle{process} = [rectangle, minimum width=3cm, minimum height=1cm, text centered, draw=black]
%\tikzstyle{decision} = [trapezium, trapezium left angle=70, trapezium right angle=110, minimum width=3cm, minimum height=1cm, text centered, draw=black]



\outcome{}


\title[Dig-In:]{Homogeneous Linear Equations}

\begin{document}
	\begin{abstract}
	  We cover the general theory behind finding the general solutions to second order homogeneous and non-homogeneous equations.
	\end{abstract}
	\maketitle

	Second order differential equations are an important branch of mathematics with applications to physics, engineering, chemistry, biology, economics, and many other disciplines.  We will focus our study on a subset of these equations.
	
	\begin{definition}[Second Order Linear Equations]
		A second order differential equation is said to be \dfn{linear} if it can be written in the form:
		\[
			y''+p(x)y'+q(x)y = f(x)
		\]
		If it is not in this form, we say it is \dfn{nonlinear}.  
	\end{definition}
	
	In other words, the equataion contains no powers, products or quotients of $ y, y', $ or $ y'' $.
	
	\begin{definition}[Homogeneous Equations]
		A second order linear equation is said to be \dfn{homogeneous} when the term $ f(x)=0 $ for all $ x $.  Otherwise, the equation is called \dfn{nonhomogeneous}.
	\end{definition}
	
	\begin{example}
		Which of the following differential equations are second order linear equations?
		\begin{selectAll}
			\choice[correct] $ x^2y''-5xy'+\ln(x)y = \sqrt{x} $
			\choice $ y''-\sqrt{xy'}+x^2y=0 $
			\choice $ y \dfrac{dy}{dx}-5x \dfrac{d^2y}{dx^2} = 3x $
			\choice[correct] $ e^x y''=y \sin(x) $
			\choice $ y''+3y'-x \cos(y) = 0 $
			\choice[correct] $ \dfrac{y''}{y'}-\ln(x) = \sqrt{x} $
		\end{selectAll}
		\begin{feedback}
			Remember, linear equations are linear in the dependent variable, $ y $, not in the independent variable, $ x $.  That means that $ p(x) $ and $ q(x) $ can be any function.  Also, we can use algebra to rewrite the equations in equivalent forms.  As long as one of these forms matches that in the definition of a linear equation, the equation is linear.
		\end{feedback}
	\end{example}
	
	\begin{question}
		Consider the differential equation $ y''-4y=0 $.  Which of the following functions are solutions to this differential equation?
		\begin{selectAll}
			\choice[correct] $ y = e^{2x} $
			\choice $ y=\sin(2x) $
			\choice $ y=\cos(2x) $
			\choice[correct] $ y = e^{-2x} $
			\choice[correct] $ y = 3e^{2x} $
		\end{selectAll}
		\begin{hint}
			Compute the first and second derivatives of each option and plug them into the equation.  A solution is a function that satisfies the equation.
		\end{hint}
		\begin{feedback}
			Notice that the differential equation is stating that the second derivative of the function is a multiple of the original function.  Exponential functions and the sine and consine functions are natural choices as solutions to this type of equation.
		\end{feedback}
	\end{question}
	
	With homogeneous equations, if we know two solutions, we can know many more soltuions as any constant multiple of a solution will produce a new solution.  This idea is called the principal of superposition.  It can be expressed in the following theorem.
	
	\begin{theorem}[Principal of Superposition]
		If $ y_1 $ and $ y_2 $ are two solutions to the homogeneous equation 
		\[
			y'' +p(x)y'+q(x)y = 0
		\]
		on the interval $ (a,b) $, then any linear combination
		\[
			y = c_1y_1 + c_2y_2
		\]
		of $ y_1 $ amd $ y_2 $ is also a solution on $ (a,b) $.
	\end{theorem}
	\begin{explanation}
		Since we know $ y-1 $ and $ y_2 $ are solutions to the equation, we know that
		\[
			y_1''+p(x)y_1'+q(x)y_1 = 0 \mbox{ and } y_2''+p(x)y_2'+q(x)y_2 = 0
		\]
		Now let's check that the linear combination of these two solutions, $ y=c_1y_1 + c_2y_2 $ is also a solution.  We have $ y' = c_1y_1' + c_2y_2' $ and $ y'' = c_1y_1'' + c_2y_2'' $.
		\begin{align*}
			y'' + p(x)y' + q(x)y &= 0 \\
			(c_1y_1'' + c_2y_2'') + p(x)(c_1y_1' + c_2y_2') + q(x)(c_1y_1 + c_2y_2) &= 0 \\
			c_1y_1'' + c_2y_2'' + c_1p(x)y_1' + c_2p(x)y_2' + c_1q(x)y_1 + c_2q(x)y_2 &= 0 \\
			c_1y_1'' + c_1p(x)y_1' + c_1q(x)y_1 + c_2y_2'' + c_2p(x)y_2' + c_2q(x)y_2 &= 0 \\
			c_1(y_1'' + p(x)y_1' + q(x)y_1) + c_2(y_2'' + p(x)y_2' + q(x)y_2) &= 0 \\
			c_1(0) + c_2(0) &= 0
		\end{align*}
	\end{explanation}
	
	This theorem shows that second order linear homogeneous equations can have infinitely many solutions.  We call the linear combination of solutions a family of solutions.  If we wanted to know a particular set of solutions from this family, we would need a set of initial conditions.  We would need to know a point the function passes through and the instantaneous rate of change at that point.  That is, we would need to know $ y(x_0) = k_0 $ and $ y'(x_0) = k_1. $  The differential equation together with the initial conditions is called an initial value problem.
	
	Unlike first order linear equations, it can be quite difficult to solve a general second order linear equation.  However, mathematicians have discovered conditions under which we can guarantee the existence of solutions to second order linear equations.
	\begin{theorem}[Uniqueness and Existence]
		content...
	\end{theorem}
\end{document}
