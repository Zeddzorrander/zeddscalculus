\documentclass{ximera}

%\usepackage{todonotes}
%\usepackage{mathtools} %% Required for wide table Curl and Greens
%\usepackage{cuted} %% Required for wide table Curl and Greens
\newcommand{\todo}{}

\usepackage{esint} % for \oiint
\ifxake%%https://math.meta.stackexchange.com/questions/9973/how-do-you-render-a-closed-surface-double-integral
\renewcommand{\oiint}{{\large\bigcirc}\kern-1.56em\iint}
\fi


\graphicspath{
  {./}
  {ximeraTutorial/}
  {basicPhilosophy/}
  {functionsOfSeveralVariables/}
  {normalVectors/}
  {lagrangeMultipliers/}
  {vectorFields/}
  {greensTheorem/}
  {shapeOfThingsToCome/}
  {dotProducts/}
  {partialDerivativesAndTheGradientVector/}
  {../productAndQuotientRules/exercises/}
  {../motionAndPathsInSpace/exercises/}
  {../normalVectors/exercisesParametricPlots/}
  {../continuityOfFunctionsOfSeveralVariables/exercises/}
  {../partialDerivativesAndTheGradientVector/exercises/}
  {../directionalDerivativeAndChainRule/exercises/}
  {../commonCoordinates/exercisesCylindricalCoordinates/}
  {../commonCoordinates/exercisesSphericalCoordinates/}
  {../greensTheorem/exercisesCurlAndLineIntegrals/}
  {../greensTheorem/exercisesDivergenceAndLineIntegrals/}
  {../shapeOfThingsToCome/exercisesDivergenceTheorem/}
  {../greensTheorem/}
  {../shapeOfThingsToCome/}
  {../separableDifferentialEquations/exercises/}
  {vectorFields/}
}

\newcommand{\mooculus}{\textsf{\textbf{MOOC}\textnormal{\textsf{ULUS}}}}

\usepackage{tkz-euclide}\usepackage{tikz}
\usepackage{tikz-cd}
\usetikzlibrary{arrows}
\tikzset{>=stealth,commutative diagrams/.cd,
  arrow style=tikz,diagrams={>=stealth}} %% cool arrow head
\tikzset{shorten <>/.style={ shorten >=#1, shorten <=#1 } } %% allows shorter vectors

\usetikzlibrary{backgrounds} %% for boxes around graphs
\usetikzlibrary{shapes,positioning}  %% Clouds and stars
\usetikzlibrary{matrix} %% for matrix
\usepgfplotslibrary{polar} %% for polar plots
\usepgfplotslibrary{fillbetween} %% to shade area between curves in TikZ
\usetkzobj{all}
\usepackage[makeroom]{cancel} %% for strike outs
%\usepackage{mathtools} %% for pretty underbrace % Breaks Ximera
%\usepackage{multicol}
\usepackage{pgffor} %% required for integral for loops



%% http://tex.stackexchange.com/questions/66490/drawing-a-tikz-arc-specifying-the-center
%% Draws beach ball
\tikzset{pics/carc/.style args={#1:#2:#3}{code={\draw[pic actions] (#1:#3) arc(#1:#2:#3);}}}



\usepackage{array}
\setlength{\extrarowheight}{+.1cm}
\newdimen\digitwidth
\settowidth\digitwidth{9}
\def\divrule#1#2{
\noalign{\moveright#1\digitwidth
\vbox{\hrule width#2\digitwidth}}}





\newcommand{\RR}{\mathbb R}
\newcommand{\R}{\mathbb R}
\newcommand{\N}{\mathbb N}
\newcommand{\Z}{\mathbb Z}

\newcommand{\sagemath}{\textsf{SageMath}}


%\renewcommand{\d}{\,d\!}
\renewcommand{\d}{\mathop{}\!d}
\newcommand{\dd}[2][]{\frac{\d #1}{\d #2}}
\newcommand{\pp}[2][]{\frac{\partial #1}{\partial #2}}
\renewcommand{\l}{\ell}
\newcommand{\ddx}{\frac{d}{\d x}}

\newcommand{\zeroOverZero}{\ensuremath{\boldsymbol{\tfrac{0}{0}}}}
\newcommand{\inftyOverInfty}{\ensuremath{\boldsymbol{\tfrac{\infty}{\infty}}}}
\newcommand{\zeroOverInfty}{\ensuremath{\boldsymbol{\tfrac{0}{\infty}}}}
\newcommand{\zeroTimesInfty}{\ensuremath{\small\boldsymbol{0\cdot \infty}}}
\newcommand{\inftyMinusInfty}{\ensuremath{\small\boldsymbol{\infty - \infty}}}
\newcommand{\oneToInfty}{\ensuremath{\boldsymbol{1^\infty}}}
\newcommand{\zeroToZero}{\ensuremath{\boldsymbol{0^0}}}
\newcommand{\inftyToZero}{\ensuremath{\boldsymbol{\infty^0}}}



\newcommand{\numOverZero}{\ensuremath{\boldsymbol{\tfrac{\#}{0}}}}
\newcommand{\dfn}{\textbf}
%\newcommand{\unit}{\,\mathrm}
\newcommand{\unit}{\mathop{}\!\mathrm}
\newcommand{\eval}[1]{\bigg[ #1 \bigg]}
\newcommand{\seq}[1]{\left( #1 \right)}
\renewcommand{\epsilon}{\varepsilon}
\renewcommand{\phi}{\varphi}


\renewcommand{\iff}{\Leftrightarrow}

\DeclareMathOperator{\arccot}{arccot}
\DeclareMathOperator{\arcsec}{arcsec}
\DeclareMathOperator{\arccsc}{arccsc}
\DeclareMathOperator{\si}{Si}
\DeclareMathOperator{\scal}{scal}
\DeclareMathOperator{\sign}{sign}


%% \newcommand{\tightoverset}[2]{% for arrow vec
%%   \mathop{#2}\limits^{\vbox to -.5ex{\kern-0.75ex\hbox{$#1$}\vss}}}
\newcommand{\arrowvec}[1]{{\overset{\rightharpoonup}{#1}}}
%\renewcommand{\vec}[1]{\arrowvec{\mathbf{#1}}}
\renewcommand{\vec}[1]{{\overset{\boldsymbol{\rightharpoonup}}{\mathbf{#1}}}\hspace{0in}}

\newcommand{\point}[1]{\left(#1\right)} %this allows \vector{ to be changed to \vector{ with a quick find and replace
\newcommand{\pt}[1]{\mathbf{#1}} %this allows \vec{ to be changed to \vec{ with a quick find and replace
\newcommand{\Lim}[2]{\lim_{\point{#1} \to \point{#2}}} %Bart, I changed this to point since I want to use it.  It runs through both of the exercise and exerciseE files in limits section, which is why it was in each document to start with.

\DeclareMathOperator{\proj}{\mathbf{proj}}
\newcommand{\veci}{{\boldsymbol{\hat{\imath}}}}
\newcommand{\vecj}{{\boldsymbol{\hat{\jmath}}}}
\newcommand{\veck}{{\boldsymbol{\hat{k}}}}
\newcommand{\vecl}{\vec{\boldsymbol{\l}}}
\newcommand{\uvec}[1]{\mathbf{\hat{#1}}}
\newcommand{\utan}{\mathbf{\hat{t}}}
\newcommand{\unormal}{\mathbf{\hat{n}}}
\newcommand{\ubinormal}{\mathbf{\hat{b}}}

\newcommand{\dotp}{\bullet}
\newcommand{\cross}{\boldsymbol\times}
\newcommand{\grad}{\boldsymbol\nabla}
\newcommand{\divergence}{\grad\dotp}
\newcommand{\curl}{\grad\cross}
%\DeclareMathOperator{\divergence}{divergence}
%\DeclareMathOperator{\curl}[1]{\grad\cross #1}
\newcommand{\lto}{\mathop{\longrightarrow\,}\limits}

\renewcommand{\bar}{\overline}

\colorlet{textColor}{black}
\colorlet{background}{white}
\colorlet{penColor}{blue!50!black} % Color of a curve in a plot
\colorlet{penColor2}{red!50!black}% Color of a curve in a plot
\colorlet{penColor3}{red!50!blue} % Color of a curve in a plot
\colorlet{penColor4}{green!50!black} % Color of a curve in a plot
\colorlet{penColor5}{orange!80!black} % Color of a curve in a plot
\colorlet{penColor6}{yellow!70!black} % Color of a curve in a plot
\colorlet{fill1}{penColor!20} % Color of fill in a plot
\colorlet{fill2}{penColor2!20} % Color of fill in a plot
\colorlet{fillp}{fill1} % Color of positive area
\colorlet{filln}{penColor2!20} % Color of negative area
\colorlet{fill3}{penColor3!20} % Fill
\colorlet{fill4}{penColor4!20} % Fill
\colorlet{fill5}{penColor5!20} % Fill
\colorlet{gridColor}{gray!50} % Color of grid in a plot

\newcommand{\surfaceColor}{violet}
\newcommand{\surfaceColorTwo}{redyellow}
\newcommand{\sliceColor}{greenyellow}




\pgfmathdeclarefunction{gauss}{2}{% gives gaussian
  \pgfmathparse{1/(#2*sqrt(2*pi))*exp(-((x-#1)^2)/(2*#2^2))}%
}


%%%%%%%%%%%%%
%% Vectors
%%%%%%%%%%%%%

%% Simple horiz vectors
\renewcommand{\vector}[1]{\left\langle #1\right\rangle}


%% %% Complex Horiz Vectors with angle brackets
%% \makeatletter
%% \renewcommand{\vector}[2][ , ]{\left\langle%
%%   \def\nextitem{\def\nextitem{#1}}%
%%   \@for \el:=#2\do{\nextitem\el}\right\rangle%
%% }
%% \makeatother

%% %% Vertical Vectors
%% \def\vector#1{\begin{bmatrix}\vecListA#1,,\end{bmatrix}}
%% \def\vecListA#1,{\if,#1,\else #1\cr \expandafter \vecListA \fi}

%%%%%%%%%%%%%
%% End of vectors
%%%%%%%%%%%%%

%\newcommand{\fullwidth}{}
%\newcommand{\normalwidth}{}



%% makes a snazzy t-chart for evaluating functions
%\newenvironment{tchart}{\rowcolors{2}{}{background!90!textColor}\array}{\endarray}

%%This is to help with formatting on future title pages.
\newenvironment{sectionOutcomes}{}{}



%% Flowchart stuff
%\tikzstyle{startstop} = [rectangle, rounded corners, minimum width=3cm, minimum height=1cm,text centered, draw=black]
%\tikzstyle{question} = [rectangle, minimum width=3cm, minimum height=1cm, text centered, draw=black]
%\tikzstyle{decision} = [trapezium, trapezium left angle=70, trapezium right angle=110, minimum width=3cm, minimum height=1cm, text centered, draw=black]
%\tikzstyle{question} = [rectangle, rounded corners, minimum width=3cm, minimum height=1cm,text centered, draw=black]
%\tikzstyle{process} = [rectangle, minimum width=3cm, minimum height=1cm, text centered, draw=black]
%\tikzstyle{decision} = [trapezium, trapezium left angle=70, trapezium right angle=110, minimum width=3cm, minimum height=1cm, text centered, draw=black]


\author{Jason Miller}
\license{Creative Commons 3.0 By-bC}


\outcome{}


\begin{document}
\begin{exercise}
Consider the parametric equations 
\begin{align*}
x(t) &= t^2-1\\
y(t) &= t^3-t
\end{align*}

For $-\infty < t < \infty$.

First note that the point $(0,0)$ belongs to the curve. For what value of $t$ does the curve pass through the origin?

Solving $0=t^2-1$ and $0=t^3-t$ we obtain $t=1$ and $t=-1$. If we think of the parameter $t$ as time, this means that as our 
curve is being traced out in time, it actually goes through the origin twice. Thus our curve intersects itself at the origin. 

Below is a portion of the curve traced out for certain values of the parameter $t$.

\begin{image}  
  \begin{tikzpicture}  
    \begin{axis}[  
        xmin=-1.4,  
        xmax=1.4,  
        ymin=-1.2,  
        ymax=1.2,  
        axis lines=center,  
        xlabel=$x$,  
        ylabel=$y$,  
        every axis y label/.style={at=(current axis.above origin),anchor=south},  
        every axis x label/.style={at=(current axis.right of origin),anchor=west},  
      ]  
      \addplot [ultra thick, penColor, smooth, domain=(-1.3:1.3)] ({x^2-1},{x^3-x});
    \end{axis}  
  \end{tikzpicture}  
\end{image} 



Because of this, it doesn't make sense to ask what the tangent line is at the origin because there is no well defined tangent line at $(0,0)$ if we simply regard our curve as just a set of points.That is, if we zoom in on the origin, our curve does not look approximately like a line. However if we use the parametric description of the curve then we can make sense of a tangent line at two different points in time. 

Let's calculate the equation of the tangent line to the curve when $t=1$. 

We calculate $\dd[y]{x}=\frac{y'(t)}{x'(t)}=\answer{ \frac{3t^2-1}{2t}}$. The slope when $t=1$ is $\answer{1}$. Thus the tangent line to the curve when $t=1$ is
$y-\answer{0}=\answer{1}(x-\answer{0})$. 

We see the tangent line to the curve when $t=1$ below in orange

\begin{image}  
  \begin{tikzpicture}  
    \begin{axis}[  
        xmin=-1.4,  
        xmax=1.4,  
        ymin=-1.2,  
        ymax=1.2,  
        axis lines=center,  
        xlabel=$x$,  
        ylabel=$y$,  
        every axis y label/.style={at=(current axis.above origin),anchor=south},  
        every axis x label/.style={at=(current axis.right of origin),anchor=west},  
      ]  
      \addplot [ultra thick, penColor, smooth, domain=(-1.3:1.3)] ({x^2-1},{x^3-x});
      \addplot [ultra thick, penColor5, smooth,domain=-1:1] {x};  
    \end{axis}  
  \end{tikzpicture}  
\end{image} 

We can also find the tangent line to the curve when $t=-1$ using the same method as above. The tangent line is $y-\answer{0}=\answer{-1}(x-\answer{0})$. This line is shown below in purple. 

\begin{image}  
  \begin{tikzpicture}  
    \begin{axis}[  
        xmin=-1.4,  
        xmax=1.4,  
        ymin=-1.2,  
        ymax=1.2,  
        axis lines=center,  
        xlabel=$x$,  
        ylabel=$y$,  
        every axis y label/.style={at=(current axis.above origin),anchor=south},  
        every axis x label/.style={at=(current axis.right of origin),anchor=west},  
      ]  
      \addplot [ultra thick, penColor, smooth, domain=(-1.3:1.3)] ({x^2-1},{x^3-x});
      \addplot [ultra thick, penColor3, smooth,domain=-1:1] {-x};  
    \end{axis}  
  \end{tikzpicture}  
\end{image} 

This shows one advantage to using a parametric description of a curve. Even thoug the curve as a static set of points has no well defined tangent line at the origin, we can still make sense of the tangent line if we use a parametric description of the curve that presents the curve as being traced out in time. 

\begin{exercise}
Find all points where the curve has a vertical tangent line. 

We have already found that $\dd[y]{x}=\frac{3t^2-1}{2t}$. This gives the slope of the tangent line at the point associated to $t$. A vertical tangent line is obtained when the denominator vanishes and the numerator does not. This essentially corresponds to line with infinite slope. 

This means that we have a vertical tangent line when $t=\answer{0}$. This corresponds to the point $(\answer{-1}, \answer{0})$.  The equation of this tangent line is $x=\answer{-1}$. 

\begin{image}  
  \begin{tikzpicture}  
    \begin{axis}[  
        xmin=-1.4,  
        xmax=1.4,  
        ymin=-1.2,  
        ymax=1.2,  
        axis lines=center,  
        xlabel=$x$,  
        ylabel=$y$,  
        every axis y label/.style={at=(current axis.above origin),anchor=south},  
        every axis x label/.style={at=(current axis.right of origin),anchor=west},  
      ]  
      \addplot [ultra thick, penColor, smooth, domain=(-1.3:1.3)] ({x^2-1},{x^3-x});
      \addplot +[ultra thick, mark=none] coordinates {(-1, -1) (-1, 1)};  
    \end{axis}  
  \end{tikzpicture}  
\end{image} 

\begin{exercise}

Find all points where the curve has a horizontal tangent line. 

We use $\dd[y]{x}=\frac{3t^2-1}{2t}$. A horizontal tangent line has slope $0$ so we need the numerator to be zero while the denominator is nonzero. Thus we see that 
we have a horizontal tangent line when $t=\answer{\frac{1}{\sqrt{3}}}$ and $t=\answer{-\frac{1}{\sqrt{3}}}$. 

The positive $t$ value corresponds to the point $\left( \answer{ \left( \frac{1}{\sqrt{3}} \right)^2-1 } ,  \answer{ \left( \frac{1}{\sqrt{3}} \right)^3 -   \left( \frac{1}{\sqrt{3}} \right)   } \right)$. The equation of the tangent line to this point is $y=\answer{ \left( \frac{1}{\sqrt{3}} \right)^3 -   \left( \frac{1}{\sqrt{3}} \right) }$. This line is show in green below. 


The negative $t$ value corresponds to the point $\left( \answer{ \left( \frac{-1}{\sqrt{3}} \right)^2-1 } ,  \answer{ \left( \frac{-1}{\sqrt{3}} \right)^3 -   \left( \frac{-1}{\sqrt{3}} \right)   } \right)$. 
The equation of the tangent line to this point is $y=\answer{ \left( \frac{-1}{\sqrt{3}} \right)^3 -   \left( \frac{-1}{\sqrt{3}} \right) }$. This line is shown in orange below. 






\begin{image}  
  \begin{tikzpicture}  
    \begin{axis}[  
        xmin=-1.4,  
        xmax=1.4,  
        ymin=-1.2,  
        ymax=1.2,  
        axis lines=center,  
        xlabel=$x$,  
        ylabel=$y$,  
        every axis y label/.style={at=(current axis.above origin),anchor=south},  
        every axis x label/.style={at=(current axis.right of origin),anchor=west},  
      ]  
      \addplot [ultra thick, penColor, smooth, domain=(-1.3:1.3)] ({x^2-1},{x^3-x});
      \addplot [ultra thick, penColor4, smooth] { -.385};  
      \addplot [ultra thick, penColor5, smooth] {.385};
    \end{axis}  
  \end{tikzpicture}  
\end{image} 











\end{exercise}
\end{exercise}
\end{exercise}
\end{document}
