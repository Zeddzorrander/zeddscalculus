\documentclass{ximera}

%\usepackage{todonotes}
%\usepackage{mathtools} %% Required for wide table Curl and Greens
%\usepackage{cuted} %% Required for wide table Curl and Greens
\newcommand{\todo}{}

\usepackage{esint} % for \oiint
\ifxake%%https://math.meta.stackexchange.com/questions/9973/how-do-you-render-a-closed-surface-double-integral
\renewcommand{\oiint}{{\large\bigcirc}\kern-1.56em\iint}
\fi


\graphicspath{
  {./}
  {ximeraTutorial/}
  {basicPhilosophy/}
  {functionsOfSeveralVariables/}
  {normalVectors/}
  {lagrangeMultipliers/}
  {vectorFields/}
  {greensTheorem/}
  {shapeOfThingsToCome/}
  {dotProducts/}
  {partialDerivativesAndTheGradientVector/}
  {../productAndQuotientRules/exercises/}
  {../motionAndPathsInSpace/exercises/}
  {../normalVectors/exercisesParametricPlots/}
  {../continuityOfFunctionsOfSeveralVariables/exercises/}
  {../partialDerivativesAndTheGradientVector/exercises/}
  {../directionalDerivativeAndChainRule/exercises/}
  {../commonCoordinates/exercisesCylindricalCoordinates/}
  {../commonCoordinates/exercisesSphericalCoordinates/}
  {../greensTheorem/exercisesCurlAndLineIntegrals/}
  {../greensTheorem/exercisesDivergenceAndLineIntegrals/}
  {../shapeOfThingsToCome/exercisesDivergenceTheorem/}
  {../greensTheorem/}
  {../shapeOfThingsToCome/}
  {../separableDifferentialEquations/exercises/}
  {vectorFields/}
}

\newcommand{\mooculus}{\textsf{\textbf{MOOC}\textnormal{\textsf{ULUS}}}}

\usepackage{tkz-euclide}\usepackage{tikz}
\usepackage{tikz-cd}
\usetikzlibrary{arrows}
\tikzset{>=stealth,commutative diagrams/.cd,
  arrow style=tikz,diagrams={>=stealth}} %% cool arrow head
\tikzset{shorten <>/.style={ shorten >=#1, shorten <=#1 } } %% allows shorter vectors

\usetikzlibrary{backgrounds} %% for boxes around graphs
\usetikzlibrary{shapes,positioning}  %% Clouds and stars
\usetikzlibrary{matrix} %% for matrix
\usepgfplotslibrary{polar} %% for polar plots
\usepgfplotslibrary{fillbetween} %% to shade area between curves in TikZ
\usetkzobj{all}
\usepackage[makeroom]{cancel} %% for strike outs
%\usepackage{mathtools} %% for pretty underbrace % Breaks Ximera
%\usepackage{multicol}
\usepackage{pgffor} %% required for integral for loops



%% http://tex.stackexchange.com/questions/66490/drawing-a-tikz-arc-specifying-the-center
%% Draws beach ball
\tikzset{pics/carc/.style args={#1:#2:#3}{code={\draw[pic actions] (#1:#3) arc(#1:#2:#3);}}}



\usepackage{array}
\setlength{\extrarowheight}{+.1cm}
\newdimen\digitwidth
\settowidth\digitwidth{9}
\def\divrule#1#2{
\noalign{\moveright#1\digitwidth
\vbox{\hrule width#2\digitwidth}}}





\newcommand{\RR}{\mathbb R}
\newcommand{\R}{\mathbb R}
\newcommand{\N}{\mathbb N}
\newcommand{\Z}{\mathbb Z}

\newcommand{\sagemath}{\textsf{SageMath}}


%\renewcommand{\d}{\,d\!}
\renewcommand{\d}{\mathop{}\!d}
\newcommand{\dd}[2][]{\frac{\d #1}{\d #2}}
\newcommand{\pp}[2][]{\frac{\partial #1}{\partial #2}}
\renewcommand{\l}{\ell}
\newcommand{\ddx}{\frac{d}{\d x}}

\newcommand{\zeroOverZero}{\ensuremath{\boldsymbol{\tfrac{0}{0}}}}
\newcommand{\inftyOverInfty}{\ensuremath{\boldsymbol{\tfrac{\infty}{\infty}}}}
\newcommand{\zeroOverInfty}{\ensuremath{\boldsymbol{\tfrac{0}{\infty}}}}
\newcommand{\zeroTimesInfty}{\ensuremath{\small\boldsymbol{0\cdot \infty}}}
\newcommand{\inftyMinusInfty}{\ensuremath{\small\boldsymbol{\infty - \infty}}}
\newcommand{\oneToInfty}{\ensuremath{\boldsymbol{1^\infty}}}
\newcommand{\zeroToZero}{\ensuremath{\boldsymbol{0^0}}}
\newcommand{\inftyToZero}{\ensuremath{\boldsymbol{\infty^0}}}



\newcommand{\numOverZero}{\ensuremath{\boldsymbol{\tfrac{\#}{0}}}}
\newcommand{\dfn}{\textbf}
%\newcommand{\unit}{\,\mathrm}
\newcommand{\unit}{\mathop{}\!\mathrm}
\newcommand{\eval}[1]{\bigg[ #1 \bigg]}
\newcommand{\seq}[1]{\left( #1 \right)}
\renewcommand{\epsilon}{\varepsilon}
\renewcommand{\phi}{\varphi}


\renewcommand{\iff}{\Leftrightarrow}

\DeclareMathOperator{\arccot}{arccot}
\DeclareMathOperator{\arcsec}{arcsec}
\DeclareMathOperator{\arccsc}{arccsc}
\DeclareMathOperator{\si}{Si}
\DeclareMathOperator{\scal}{scal}
\DeclareMathOperator{\sign}{sign}


%% \newcommand{\tightoverset}[2]{% for arrow vec
%%   \mathop{#2}\limits^{\vbox to -.5ex{\kern-0.75ex\hbox{$#1$}\vss}}}
\newcommand{\arrowvec}[1]{{\overset{\rightharpoonup}{#1}}}
%\renewcommand{\vec}[1]{\arrowvec{\mathbf{#1}}}
\renewcommand{\vec}[1]{{\overset{\boldsymbol{\rightharpoonup}}{\mathbf{#1}}}\hspace{0in}}

\newcommand{\point}[1]{\left(#1\right)} %this allows \vector{ to be changed to \vector{ with a quick find and replace
\newcommand{\pt}[1]{\mathbf{#1}} %this allows \vec{ to be changed to \vec{ with a quick find and replace
\newcommand{\Lim}[2]{\lim_{\point{#1} \to \point{#2}}} %Bart, I changed this to point since I want to use it.  It runs through both of the exercise and exerciseE files in limits section, which is why it was in each document to start with.

\DeclareMathOperator{\proj}{\mathbf{proj}}
\newcommand{\veci}{{\boldsymbol{\hat{\imath}}}}
\newcommand{\vecj}{{\boldsymbol{\hat{\jmath}}}}
\newcommand{\veck}{{\boldsymbol{\hat{k}}}}
\newcommand{\vecl}{\vec{\boldsymbol{\l}}}
\newcommand{\uvec}[1]{\mathbf{\hat{#1}}}
\newcommand{\utan}{\mathbf{\hat{t}}}
\newcommand{\unormal}{\mathbf{\hat{n}}}
\newcommand{\ubinormal}{\mathbf{\hat{b}}}

\newcommand{\dotp}{\bullet}
\newcommand{\cross}{\boldsymbol\times}
\newcommand{\grad}{\boldsymbol\nabla}
\newcommand{\divergence}{\grad\dotp}
\newcommand{\curl}{\grad\cross}
%\DeclareMathOperator{\divergence}{divergence}
%\DeclareMathOperator{\curl}[1]{\grad\cross #1}
\newcommand{\lto}{\mathop{\longrightarrow\,}\limits}

\renewcommand{\bar}{\overline}

\colorlet{textColor}{black}
\colorlet{background}{white}
\colorlet{penColor}{blue!50!black} % Color of a curve in a plot
\colorlet{penColor2}{red!50!black}% Color of a curve in a plot
\colorlet{penColor3}{red!50!blue} % Color of a curve in a plot
\colorlet{penColor4}{green!50!black} % Color of a curve in a plot
\colorlet{penColor5}{orange!80!black} % Color of a curve in a plot
\colorlet{penColor6}{yellow!70!black} % Color of a curve in a plot
\colorlet{fill1}{penColor!20} % Color of fill in a plot
\colorlet{fill2}{penColor2!20} % Color of fill in a plot
\colorlet{fillp}{fill1} % Color of positive area
\colorlet{filln}{penColor2!20} % Color of negative area
\colorlet{fill3}{penColor3!20} % Fill
\colorlet{fill4}{penColor4!20} % Fill
\colorlet{fill5}{penColor5!20} % Fill
\colorlet{gridColor}{gray!50} % Color of grid in a plot

\newcommand{\surfaceColor}{violet}
\newcommand{\surfaceColorTwo}{redyellow}
\newcommand{\sliceColor}{greenyellow}




\pgfmathdeclarefunction{gauss}{2}{% gives gaussian
  \pgfmathparse{1/(#2*sqrt(2*pi))*exp(-((x-#1)^2)/(2*#2^2))}%
}


%%%%%%%%%%%%%
%% Vectors
%%%%%%%%%%%%%

%% Simple horiz vectors
\renewcommand{\vector}[1]{\left\langle #1\right\rangle}


%% %% Complex Horiz Vectors with angle brackets
%% \makeatletter
%% \renewcommand{\vector}[2][ , ]{\left\langle%
%%   \def\nextitem{\def\nextitem{#1}}%
%%   \@for \el:=#2\do{\nextitem\el}\right\rangle%
%% }
%% \makeatother

%% %% Vertical Vectors
%% \def\vector#1{\begin{bmatrix}\vecListA#1,,\end{bmatrix}}
%% \def\vecListA#1,{\if,#1,\else #1\cr \expandafter \vecListA \fi}

%%%%%%%%%%%%%
%% End of vectors
%%%%%%%%%%%%%

%\newcommand{\fullwidth}{}
%\newcommand{\normalwidth}{}



%% makes a snazzy t-chart for evaluating functions
%\newenvironment{tchart}{\rowcolors{2}{}{background!90!textColor}\array}{\endarray}

%%This is to help with formatting on future title pages.
\newenvironment{sectionOutcomes}{}{}



%% Flowchart stuff
%\tikzstyle{startstop} = [rectangle, rounded corners, minimum width=3cm, minimum height=1cm,text centered, draw=black]
%\tikzstyle{question} = [rectangle, minimum width=3cm, minimum height=1cm, text centered, draw=black]
%\tikzstyle{decision} = [trapezium, trapezium left angle=70, trapezium right angle=110, minimum width=3cm, minimum height=1cm, text centered, draw=black]
%\tikzstyle{question} = [rectangle, rounded corners, minimum width=3cm, minimum height=1cm,text centered, draw=black]
%\tikzstyle{process} = [rectangle, minimum width=3cm, minimum height=1cm, text centered, draw=black]
%\tikzstyle{decision} = [trapezium, trapezium left angle=70, trapezium right angle=110, minimum width=3cm, minimum height=1cm, text centered, draw=black]


\outcome{Identify products of functions.}
\outcome{Use the product rule to calculate derivatives.}
\outcome{Identify quotients of functions.}
\outcome{Use the quotient rule to calculate derivatives.}
\outcome{Combine derivative rules to take derivatives of more complicated functions.}
\outcome{Explain the signs of the terms in the numerator of the quotient rule.}
\outcome{Multiply tangent lines to justify the product rule.}
\outcome{Use the product and quotient rule to calculate derivatives from a table of values.}

\title[Dig-In:]{The Product rule and quotient rule}

\begin{document}
\begin{abstract}
Here we compute derivatives of products and quotients of functions
\end{abstract}
\maketitle


\section{The product rule}


Consider the product of two simple functions, say
\[
f(x)\cdot g(x)
\]
where $f(x)=x^2+1$ and $g(x)=x^3-3x$. An obvious guess for the
derivative of $f(x)g(x)$ is the product of the derivatives:
\begin{align*}
f'(x)g'(x) &= (2x)(3x^2-3)\\
&= 6x^3-6x.
\end{align*}
Is this guess correct? We can check by rewriting $f$ and $g$ and doing
the calculation in a way that is known to work. Write with me
\begin{align*}
f(x)g(x) &= (x^2+1)(x^3-3x)\\
&=x^5-3x^3+x^3-3x\\
&=x^5-2x^3-3x.
\end{align*} 
Hence
\[
\ddx f(x) g(x) = \ddx(x^5 - 2x^3 - 3x) = 5x^4-6x^2-3, 
\]
so we see that 
\[
\ddx f(x) g(x) \ne  f'(x)g'(x).
\]
So the derivative of $f(x)g(x)$ is \textbf{not} as simple as
$f'(x)g'(x)$. Never fear, we have a rule for exactly this
situation.
\begin{theorem}[The product rule]\index{derivative rules!product}\index{product rule}\label{theorem:product-rule}
If $f$ and $g$ are differentiable, then
\[
\ddx f(x)g(x) = f(x)g'(x)+f'(x)g(x).
\]
\end{theorem}

%% \begin{image}
%% \begin{tikzpicture}
%% 	\begin{axis}[
%%             clip=false,
%%             domain=1.5:6, 
%%             ytickmin=1,ytickmax=0,
%%             xtick={4},ytickmin=1,ytickmax=0,
%%             xticklabels={$a$},
%%             ymin=0, ymax=6,
%%             xlabel=$x$, ylabel=$y$,
%%             axis lines=center,
%%             every axis y label/.style={at=(current axis.above origin),anchor=south},
%%             every axis x label/.style={at=(current axis.right of origin),anchor=west},
%%             axis on top,
%%           ]          
%%           %\addplot [dashed, textColor] plot coordinates {(4,0) (4,3.08)};
%%           %\node at (axis cs:4,0) [anchor=north] {$x$};

%%           \addplot [penColor5,very thick] plot coordinates {(5,1.4) (5,1.9)};
%%           \addplot [dashed, very thick, textColor] plot coordinates {(4,1.4) (5,1.4)};

%%           \addplot [penColor4,very thick] plot coordinates {(5,2.2) (5,2.6)};
%%           \addplot [dashed, very thick, textColor] plot coordinates {(4,2.2) (5,2.2)};

%%           \addplot [very thick, penColor5!50!penColor2] plot coordinates {(5,3.08) (5,4.18)};
%%           \addplot [very thick, penColor4!50!penColor] plot coordinates {(5,4.18) (5,4.74)};
%%           \addplot [dashed, very thick, textColor] plot coordinates {(4,3.08) (5,3.08)};
        
%%           \addplot [very thick,penColor,smooth] {-.6+.5*x};
%%           \addplot [very thick,penColor2,smooth] {.6+.4*x};         
%%           \addplot [very thick,penColor3,smooth,domain=1:5.5] {-.36+.06*x+.2*x^2};      
%%           \addplot [penColor3!70!background,smooth, domain=1.6:5.6] {-3.56+1.66*x};      
          
%%           \addplot [dashed, textColor] plot coordinates {(4,0) (4,3.08)};

%%           \node at (axis cs:3.5,1.1) [anchor=north,penColor] {$f(x)$};
%%           \node at (axis cs:2,1.95) [anchor=north,penColor2] {$g(x)$};
%%           \node at (axis cs:4.7,4.3) [anchor=east,penColor3] {$f(x)g(x)$};

%%           \node at (axis cs:5,3.91) [anchor=west] {${\color{penColor}f(a)}{\color{penColor4}g'(a)h}+{\color{penColor5}f'(a)h}{\color{penColor2}g(a)} + {\color{penColor5}f'(a)h}{\color{penColor4}g'(a)h}$};
%%           \node at (axis cs:5,1.65) [anchor=west,penColor5] {$f'(a)h$};
%%           \node at (axis cs:5,2.4) [anchor=west,penColor4] {$g'(a)h$};
%%           \node at (axis cs:4.5,1.5) [anchor=north] {$\underbrace{\hspace{.40in}}_{h}$};

%%           \addplot[color=penColor,fill=penColor,only marks,mark=*] coordinates{(4,1.4)};  %% closed hole          
%%           \addplot[color=penColor2,fill=penColor2,only marks,mark=*] coordinates{(4,2.2)};  %% closed hole          
%%           \addplot[color=penColor3,fill=penColor3,only marks,mark=*] coordinates{(4,3.08)};  %% closed hole          
%%         \end{axis}
%% \end{tikzpicture}
%% %% \caption[A geometric interpretation of the product rule.]{A geometric interpretation of the product rule. Since every
%% %%   point on $f(x)g(x)$ is the product of the corresponding points on
%% %%   $f(x)$ and $g(x)$, increasing $a$ by a ``small amount'' $h$,
%% %%   increases $f(a)g(a)$ by the sum of $f(a)g'(a)h$ and
%% %%   $f'(a)hg(a)$. Hence,
%% %% \begin{align*}
%% %% \frac{\Delta y}{\Delta x} &\approx \frac{f(a)g'(a)h+f'(a)g(a)h + f'(a)g'(a)h^2}{h}\\
%% %% &\approx f(a)g'(a) + f'(a)g(a).
%% %% \end{align*}}
%% \end{image}

%% \begin{explanation}
%% From the limit definition of the derivative, write
%% \[
%% \ddx (f(x)g(x)) = \lim_{h \to0} \frac{f(\answer[given]{x+h})g(\answer[given]{x+h}) - f(\answer[given]{x})g(\answer[given]{x})}{\answer[given]{h}}
%% \]
%% Now we use a trick, we add $0 = -f(x+h)g(x) + f(x+h)g(x)$:
%% \begin{align*}
%% &=\lim_{h \to0} \frac{f(x+h)g(x+h){\color{penColor2}-f(x+h)g(x) + f(x+h)g(x)}- f(x)g(x)}{h} \\ 
%% &=\lim_{h \to0} \frac{f(x+h)g(x+h)-f(x+h)g(x)}{h} + \lim_{h \to0} \frac{f(x+h)g(x)- f(x)g(x)}{h}.
%% \end{align*}
%% Since both $f(x)$ and $g(x)$ are differentiable, they are
%% continuous, see Theorem~\ref{theorem:diff-cont}. Hence
%% \begin{align*}
%% &=\lim_{h \to0} \left(f(x+h)\frac{g(x+h)-g(x)}{h}\right) + \lim_{h \to0} \left(\frac{f(x+h)- f(x)}{h}g(x)\right) \\ 
%% &=\lim_{h \to0} f(x+h)\lim_{h \to0}\cdot \frac{g(x+h)-g(x)}{h} + \lim_{h \to0} \frac{f(x+h)- f(x)}{h} \cdot \lim_{h \to0}g(x) \\ 
%% &=f(x)g'(x) + f'(x)g(x).
%% \end{align*}
%% \end{explanation}



Let's return to the example with which we started.
\begin{example} 
Let $f(x)=(x^2+1)$ and $g(x)=(x^3-3x)$. Compute:
\[
\ddx f(x)g(x)
\]

\begin{explanation}
Write with me
\begin{align*}
\ddx f(x)g(x) &= f(x)g'(x) + f'(x)g(x)\\
&=(x^2+1)(\answer[given]{3x^2-3}) + (\answer[given]{2x})(x^3-3x).
\end{align*}
We could stop here, but we should show that expanding this out recovers
our previous result. Write with me
\begin{align*}
(x^2+1)&(3x^2-3) + 2x(x^3-3x)\\
  &= 3x^4-3x^2 +3x^2 -3 + 2x^4-6x^2\\
&=\answer[given]{5x^4-6x^2-3},
\end{align*}
which is precisely what we obtained before.
\end{explanation}
\end{example}

Now that we are pros, let's try one more example.

\begin{example} 
Compute:
\[
\ddx(xe^x-e^x)
\]
\begin{explanation}
Using the sum rule and the product rule, write with me
\begin{align*}
  \ddx \left(xe^x-e^x \right) &=\ddx \left(xe^x\right)-\ddx e^x\\
  &=(xe^x+e^x)-e^x\\
  &=\answer[given]{x e^x}.
\end{align*}
\end{explanation}
\end{example}


\section{The quotient rule}


We'd like to have a formula to compute
\[
\ddx \frac{f(x)}{g(x)}
\]
 This brings us to our next derivative rule.

\begin{theorem}[The quotient rule]\index{derivative rules!quotient}\index{quotient rule}\label{theorem:quotient-rule}
If $f$ and $g$ are differentiable, then
\[
\ddx \frac{f(x)}{g(x)} = \frac{f'(x)g(x)-f(x)g'(x)}{g(x)^2}.
\]
\end{theorem}
%% \begin{explanation}
%% First note that if we knew how to compute
%% \[
%% \ddx \frac{1}{g(x)}
%% \]
%% then we could use the product rule to complete our explanation.  Write
%% \begin{align*}
%% \ddx\frac{1}{g(x)}&=\lim_{h\to0} \frac{\frac{1}{g(x+h)}-\frac{1}{g(x)}}{h} \\
%% &=\lim_{h\to0} \frac{\frac{g(\answer[given]{x})-g(\answer[given]{x+h})}{g(x+h)g(x)}}{h} \\
%% &=\lim_{h\to0} \frac{g(x)-g(x+h)}{g(x+h)g(x)h} \\
%% &=\lim_{h\to0} -\frac{g(x+h)-g(x)}{h} \frac{1}{g(x+h)g(x)} \\
%% &=-\frac{g'(x)}{g(x)^2}.
%% \end{align*}
%% Now we can put this together with the product rule:
%% \begin{align*}
%% \ddx\frac{f(x)}{g(x)} &=f(x)\frac{-g'(x)}{g(x)^2}+f'(x)\frac{1}{g(x)}\\
%% &=\frac{-f(x)g'(x)+f'(x)g(x)}{g(x)^2}\\
%% &=\frac{f'(x)g(x)-f(x)g'(x)}{g(x)^2}.
%% \end{align*}

%% \end{explanation}


\begin{example}
Compute:
\[
\ddx \frac{x^2+1}{x^3-3x}
\]

\begin{explanation}
Write with me
\begin{align*}
\ddx \frac{x^2+1}{x^3-3x} &= \frac{2x(\answer[given]{x^3-3x})-(\answer[given]{x^2+1})(3x^2-3)}{(\answer[given]{x^3-3x})^2}\\
&=\frac{-x^4-6x^2+3}{(\answer[given]{x^3-3x})^2}.
\end{align*}
\end{explanation}
\end{example}
\begin{example}
Compute:
\[
\ddx \frac{\sin{x}}{x}
\]

\begin{explanation}
Write with me
\begin{align*}
\ddx \frac{\sin{x}}{x} &=  \frac{\cos{x}\cdot x-\sin{x}\cdot1}{x^2}
\end{align*}
\end{explanation}
\end{example}
It is often possible to calculate derivatives in more than one way, as
we have already seen. Since every quotient can be written as a
product, it is always possible to use the product rule to compute the
derivative, though it is not always simpler.

\begin{example}
Compute: 
\[
\ddx \frac{625-x^2}{\sqrt{x}}
\] 
in two ways. First using the quotient rule and then using the product
rule.
\begin{explanation}
First, we'll compute the derivative using the quotient rule. Write with me
\[
\ddx \frac{625-x^2}{\sqrt{x}} = \frac{\left(-2x\right)\left(\answer[given]{\sqrt{x}}\right) - (\answer[given]{625-x^2})\left(\frac{1}{2}x^{-1/2}\right)}{\answer[given]{x}}.
\]

Second, we'll compute the derivative using the product rule:
\begin{align*}
\ddx \frac{625-x^2}{\sqrt{x}} &= \ddx \left(625-x^2\right)x^{-1/2}\\
=\left(625-x^2\right)&\left(\answer[given]{\frac{-x^{-3/2}}{2}}\right)+ (\answer[given]{-2x})\left(x^{-1/2}\right).
\end{align*}
With a bit of algebra, both of these simplify to
\[
-\frac{3x^2+625}{2x^{3/2}}.
\]
\end{explanation}
\end{example}
\begin{example}
  Suppose we have two functions, $f$, and $g$, and we know that $f(4) = 3$, $f'(4) = 5$, $g(4) = -2$, and $g'(4) = 2$.
  What is the slope of the tangent line to the curve $y=\frac{f(x)}{g(x)}$ at the point where $x = 4$?
  \begin{explanation}
  	 The slope of the tangent line to the curve  $y=\frac{f(x)}{g(x)}$ at $x = 4$ is given by $ \eval{\ddx\frac{f(x)}{g(x)}}_{x=4}$.
	 
	  By the QuotientRule, this derivative is given by 
	 \begin{align*}
\eval{\ddx\frac{f(x)}{g(x)}}_{x=4} &=\eval{\frac{f'(x)g(x)-f(x)g'(x)}{\left(g(x)\right)^2}}_{x=4}\\
&=\frac{f'(4)g(4)-f(4)g'(4)}{\left(g(4)\right)^2}\\
&=\frac{5(-2)-3\cdot2}{\left(-2\right)^2}\\
&=\frac{-10-6}{4}\\
&=4
\end{align*}
	   \end{explanation}
\end{example} 
\begin{example}
  Suppose we have two functions, $f$, and $g$, and we know that $f(4) = 3$, $f'(4) = 5$, $g(4) = -2$, and $g'(4) = 2$.
  What is the slope of the tangent line to the curve $y=\frac{xf(x)}{g(x)}$ at the point where $x = 4$?
  \begin{explanation}
  	 The slope of the tangent line to the curve  $y=\frac{xf(x)}{g(x)}$ at $x = 4$ is given by $ \eval{\ddx\frac{xf(x)}{g(x)}}_{x=4}$.
	 
	  By the QuotientRule and the Product Rule, this derivative is given by 
	 \begin{align*}
\eval{\ddx\frac{xf(x)}{g(x)}}_{x=4} &=\eval{\frac{(xf'(x)+f(x))g(x)-xf(x)g'(x)}{\left(g(x)\right)^2}}_{x=4}\\
&=\frac{(4f'(4)+f(4))g(4)-4f(4)g'(4)}{\left(g(4)\right)^2}\\
&=\frac{(4\cdot5+3)(-2)-4\cdot3\cdot2}{\left(-2\right)^2}\\
&=\frac{(4\cdot5+3)(-2)-4\cdot3\cdot2}{\left(-2\right)^2}\\
&=\frac{-35}{2}\\
\end{align*}
	   \end{explanation}
\end{example} 


\end{document}
