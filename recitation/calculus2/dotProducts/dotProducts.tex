\documentclass[noauthor, handout]{ximera}
%handout:  for handout version with no solutions or instructor notes
%handout,instructornotes:  for instructor version with just problems and notes, no solutions
%noinstructornotes:  shows only problem and solutions

%% handout
%% space
%% newpage
%% numbers
%% nooutcomes

%I added the commands here so that I would't have to keep looking them up
%\newcommand{\RR}{\mathbb R}
%\renewcommand{\d}{\,d}
%\newcommand{\dd}[2][]{\frac{d #1}{d #2}}
%\renewcommand{\l}{\ell}
%\newcommand{\ddx}{\frac{d}{dx}}
%\everymath{\displaystyle}
%\newcommand{\dfn}{\textbf}
%\newcommand{\eval}[1]{\bigg[ #1 \bigg]}

%\begin{image}
%\includegraphics[trim= 170 420 250 180]{Figure1.pdf}
%\end{image}

%add a ``.'' below when used in a specific directory.

%%\usepackage{todonotes}
%\usepackage{mathtools} %% Required for wide table Curl and Greens
%\usepackage{cuted} %% Required for wide table Curl and Greens
\newcommand{\todo}{}

\usepackage{esint} % for \oiint
\ifxake%%https://math.meta.stackexchange.com/questions/9973/how-do-you-render-a-closed-surface-double-integral
\renewcommand{\oiint}{{\large\bigcirc}\kern-1.56em\iint}
\fi


\graphicspath{
  {./}
  {ximeraTutorial/}
  {basicPhilosophy/}
  {functionsOfSeveralVariables/}
  {normalVectors/}
  {lagrangeMultipliers/}
  {vectorFields/}
  {greensTheorem/}
  {shapeOfThingsToCome/}
  {dotProducts/}
  {partialDerivativesAndTheGradientVector/}
  {../productAndQuotientRules/exercises/}
  {../motionAndPathsInSpace/exercises/}
  {../normalVectors/exercisesParametricPlots/}
  {../continuityOfFunctionsOfSeveralVariables/exercises/}
  {../partialDerivativesAndTheGradientVector/exercises/}
  {../directionalDerivativeAndChainRule/exercises/}
  {../commonCoordinates/exercisesCylindricalCoordinates/}
  {../commonCoordinates/exercisesSphericalCoordinates/}
  {../greensTheorem/exercisesCurlAndLineIntegrals/}
  {../greensTheorem/exercisesDivergenceAndLineIntegrals/}
  {../shapeOfThingsToCome/exercisesDivergenceTheorem/}
  {../greensTheorem/}
  {../shapeOfThingsToCome/}
  {../separableDifferentialEquations/exercises/}
  {vectorFields/}
}

\newcommand{\mooculus}{\textsf{\textbf{MOOC}\textnormal{\textsf{ULUS}}}}

\usepackage{tkz-euclide}\usepackage{tikz}
\usepackage{tikz-cd}
\usetikzlibrary{arrows}
\tikzset{>=stealth,commutative diagrams/.cd,
  arrow style=tikz,diagrams={>=stealth}} %% cool arrow head
\tikzset{shorten <>/.style={ shorten >=#1, shorten <=#1 } } %% allows shorter vectors

\usetikzlibrary{backgrounds} %% for boxes around graphs
\usetikzlibrary{shapes,positioning}  %% Clouds and stars
\usetikzlibrary{matrix} %% for matrix
\usepgfplotslibrary{polar} %% for polar plots
\usepgfplotslibrary{fillbetween} %% to shade area between curves in TikZ
\usetkzobj{all}
\usepackage[makeroom]{cancel} %% for strike outs
%\usepackage{mathtools} %% for pretty underbrace % Breaks Ximera
%\usepackage{multicol}
\usepackage{pgffor} %% required for integral for loops



%% http://tex.stackexchange.com/questions/66490/drawing-a-tikz-arc-specifying-the-center
%% Draws beach ball
\tikzset{pics/carc/.style args={#1:#2:#3}{code={\draw[pic actions] (#1:#3) arc(#1:#2:#3);}}}



\usepackage{array}
\setlength{\extrarowheight}{+.1cm}
\newdimen\digitwidth
\settowidth\digitwidth{9}
\def\divrule#1#2{
\noalign{\moveright#1\digitwidth
\vbox{\hrule width#2\digitwidth}}}





\newcommand{\RR}{\mathbb R}
\newcommand{\R}{\mathbb R}
\newcommand{\N}{\mathbb N}
\newcommand{\Z}{\mathbb Z}

\newcommand{\sagemath}{\textsf{SageMath}}


%\renewcommand{\d}{\,d\!}
\renewcommand{\d}{\mathop{}\!d}
\newcommand{\dd}[2][]{\frac{\d #1}{\d #2}}
\newcommand{\pp}[2][]{\frac{\partial #1}{\partial #2}}
\renewcommand{\l}{\ell}
\newcommand{\ddx}{\frac{d}{\d x}}

\newcommand{\zeroOverZero}{\ensuremath{\boldsymbol{\tfrac{0}{0}}}}
\newcommand{\inftyOverInfty}{\ensuremath{\boldsymbol{\tfrac{\infty}{\infty}}}}
\newcommand{\zeroOverInfty}{\ensuremath{\boldsymbol{\tfrac{0}{\infty}}}}
\newcommand{\zeroTimesInfty}{\ensuremath{\small\boldsymbol{0\cdot \infty}}}
\newcommand{\inftyMinusInfty}{\ensuremath{\small\boldsymbol{\infty - \infty}}}
\newcommand{\oneToInfty}{\ensuremath{\boldsymbol{1^\infty}}}
\newcommand{\zeroToZero}{\ensuremath{\boldsymbol{0^0}}}
\newcommand{\inftyToZero}{\ensuremath{\boldsymbol{\infty^0}}}



\newcommand{\numOverZero}{\ensuremath{\boldsymbol{\tfrac{\#}{0}}}}
\newcommand{\dfn}{\textbf}
%\newcommand{\unit}{\,\mathrm}
\newcommand{\unit}{\mathop{}\!\mathrm}
\newcommand{\eval}[1]{\bigg[ #1 \bigg]}
\newcommand{\seq}[1]{\left( #1 \right)}
\renewcommand{\epsilon}{\varepsilon}
\renewcommand{\phi}{\varphi}


\renewcommand{\iff}{\Leftrightarrow}

\DeclareMathOperator{\arccot}{arccot}
\DeclareMathOperator{\arcsec}{arcsec}
\DeclareMathOperator{\arccsc}{arccsc}
\DeclareMathOperator{\si}{Si}
\DeclareMathOperator{\scal}{scal}
\DeclareMathOperator{\sign}{sign}


%% \newcommand{\tightoverset}[2]{% for arrow vec
%%   \mathop{#2}\limits^{\vbox to -.5ex{\kern-0.75ex\hbox{$#1$}\vss}}}
\newcommand{\arrowvec}[1]{{\overset{\rightharpoonup}{#1}}}
%\renewcommand{\vec}[1]{\arrowvec{\mathbf{#1}}}
\renewcommand{\vec}[1]{{\overset{\boldsymbol{\rightharpoonup}}{\mathbf{#1}}}\hspace{0in}}

\newcommand{\point}[1]{\left(#1\right)} %this allows \vector{ to be changed to \vector{ with a quick find and replace
\newcommand{\pt}[1]{\mathbf{#1}} %this allows \vec{ to be changed to \vec{ with a quick find and replace
\newcommand{\Lim}[2]{\lim_{\point{#1} \to \point{#2}}} %Bart, I changed this to point since I want to use it.  It runs through both of the exercise and exerciseE files in limits section, which is why it was in each document to start with.

\DeclareMathOperator{\proj}{\mathbf{proj}}
\newcommand{\veci}{{\boldsymbol{\hat{\imath}}}}
\newcommand{\vecj}{{\boldsymbol{\hat{\jmath}}}}
\newcommand{\veck}{{\boldsymbol{\hat{k}}}}
\newcommand{\vecl}{\vec{\boldsymbol{\l}}}
\newcommand{\uvec}[1]{\mathbf{\hat{#1}}}
\newcommand{\utan}{\mathbf{\hat{t}}}
\newcommand{\unormal}{\mathbf{\hat{n}}}
\newcommand{\ubinormal}{\mathbf{\hat{b}}}

\newcommand{\dotp}{\bullet}
\newcommand{\cross}{\boldsymbol\times}
\newcommand{\grad}{\boldsymbol\nabla}
\newcommand{\divergence}{\grad\dotp}
\newcommand{\curl}{\grad\cross}
%\DeclareMathOperator{\divergence}{divergence}
%\DeclareMathOperator{\curl}[1]{\grad\cross #1}
\newcommand{\lto}{\mathop{\longrightarrow\,}\limits}

\renewcommand{\bar}{\overline}

\colorlet{textColor}{black}
\colorlet{background}{white}
\colorlet{penColor}{blue!50!black} % Color of a curve in a plot
\colorlet{penColor2}{red!50!black}% Color of a curve in a plot
\colorlet{penColor3}{red!50!blue} % Color of a curve in a plot
\colorlet{penColor4}{green!50!black} % Color of a curve in a plot
\colorlet{penColor5}{orange!80!black} % Color of a curve in a plot
\colorlet{penColor6}{yellow!70!black} % Color of a curve in a plot
\colorlet{fill1}{penColor!20} % Color of fill in a plot
\colorlet{fill2}{penColor2!20} % Color of fill in a plot
\colorlet{fillp}{fill1} % Color of positive area
\colorlet{filln}{penColor2!20} % Color of negative area
\colorlet{fill3}{penColor3!20} % Fill
\colorlet{fill4}{penColor4!20} % Fill
\colorlet{fill5}{penColor5!20} % Fill
\colorlet{gridColor}{gray!50} % Color of grid in a plot

\newcommand{\surfaceColor}{violet}
\newcommand{\surfaceColorTwo}{redyellow}
\newcommand{\sliceColor}{greenyellow}




\pgfmathdeclarefunction{gauss}{2}{% gives gaussian
  \pgfmathparse{1/(#2*sqrt(2*pi))*exp(-((x-#1)^2)/(2*#2^2))}%
}


%%%%%%%%%%%%%
%% Vectors
%%%%%%%%%%%%%

%% Simple horiz vectors
\renewcommand{\vector}[1]{\left\langle #1\right\rangle}


%% %% Complex Horiz Vectors with angle brackets
%% \makeatletter
%% \renewcommand{\vector}[2][ , ]{\left\langle%
%%   \def\nextitem{\def\nextitem{#1}}%
%%   \@for \el:=#2\do{\nextitem\el}\right\rangle%
%% }
%% \makeatother

%% %% Vertical Vectors
%% \def\vector#1{\begin{bmatrix}\vecListA#1,,\end{bmatrix}}
%% \def\vecListA#1,{\if,#1,\else #1\cr \expandafter \vecListA \fi}

%%%%%%%%%%%%%
%% End of vectors
%%%%%%%%%%%%%

%\newcommand{\fullwidth}{}
%\newcommand{\normalwidth}{}



%% makes a snazzy t-chart for evaluating functions
%\newenvironment{tchart}{\rowcolors{2}{}{background!90!textColor}\array}{\endarray}

%%This is to help with formatting on future title pages.
\newenvironment{sectionOutcomes}{}{}



%% Flowchart stuff
%\tikzstyle{startstop} = [rectangle, rounded corners, minimum width=3cm, minimum height=1cm,text centered, draw=black]
%\tikzstyle{question} = [rectangle, minimum width=3cm, minimum height=1cm, text centered, draw=black]
%\tikzstyle{decision} = [trapezium, trapezium left angle=70, trapezium right angle=110, minimum width=3cm, minimum height=1cm, text centered, draw=black]
%\tikzstyle{question} = [rectangle, rounded corners, minimum width=3cm, minimum height=1cm,text centered, draw=black]
%\tikzstyle{process} = [rectangle, minimum width=3cm, minimum height=1cm, text centered, draw=black]
%\tikzstyle{decision} = [trapezium, trapezium left angle=70, trapezium right angle=110, minimum width=3cm, minimum height=1cm, text centered, draw=black]

%\usepackage{todonotes}
%\usepackage{mathtools} %% Required for wide table Curl and Greens
%\usepackage{cuted} %% Required for wide table Curl and Greens
\newcommand{\todo}{}

\usepackage{esint} % for \oiint
\ifxake%%https://math.meta.stackexchange.com/questions/9973/how-do-you-render-a-closed-surface-double-integral
\renewcommand{\oiint}{{\large\bigcirc}\kern-1.56em\iint}
\fi


\graphicspath{
  {./}
  {ximeraTutorial/}
  {basicPhilosophy/}
  {functionsOfSeveralVariables/}
  {normalVectors/}
  {lagrangeMultipliers/}
  {vectorFields/}
  {greensTheorem/}
  {shapeOfThingsToCome/}
  {dotProducts/}
  {partialDerivativesAndTheGradientVector/}
  {../productAndQuotientRules/exercises/}
  {../motionAndPathsInSpace/exercises/}
  {../normalVectors/exercisesParametricPlots/}
  {../continuityOfFunctionsOfSeveralVariables/exercises/}
  {../partialDerivativesAndTheGradientVector/exercises/}
  {../directionalDerivativeAndChainRule/exercises/}
  {../commonCoordinates/exercisesCylindricalCoordinates/}
  {../commonCoordinates/exercisesSphericalCoordinates/}
  {../greensTheorem/exercisesCurlAndLineIntegrals/}
  {../greensTheorem/exercisesDivergenceAndLineIntegrals/}
  {../shapeOfThingsToCome/exercisesDivergenceTheorem/}
  {../greensTheorem/}
  {../shapeOfThingsToCome/}
  {../separableDifferentialEquations/exercises/}
  {vectorFields/}
}

\newcommand{\mooculus}{\textsf{\textbf{MOOC}\textnormal{\textsf{ULUS}}}}

\usepackage{tkz-euclide}\usepackage{tikz}
\usepackage{tikz-cd}
\usetikzlibrary{arrows}
\tikzset{>=stealth,commutative diagrams/.cd,
  arrow style=tikz,diagrams={>=stealth}} %% cool arrow head
\tikzset{shorten <>/.style={ shorten >=#1, shorten <=#1 } } %% allows shorter vectors

\usetikzlibrary{backgrounds} %% for boxes around graphs
\usetikzlibrary{shapes,positioning}  %% Clouds and stars
\usetikzlibrary{matrix} %% for matrix
\usepgfplotslibrary{polar} %% for polar plots
\usepgfplotslibrary{fillbetween} %% to shade area between curves in TikZ
\usetkzobj{all}
\usepackage[makeroom]{cancel} %% for strike outs
%\usepackage{mathtools} %% for pretty underbrace % Breaks Ximera
%\usepackage{multicol}
\usepackage{pgffor} %% required for integral for loops



%% http://tex.stackexchange.com/questions/66490/drawing-a-tikz-arc-specifying-the-center
%% Draws beach ball
\tikzset{pics/carc/.style args={#1:#2:#3}{code={\draw[pic actions] (#1:#3) arc(#1:#2:#3);}}}



\usepackage{array}
\setlength{\extrarowheight}{+.1cm}
\newdimen\digitwidth
\settowidth\digitwidth{9}
\def\divrule#1#2{
\noalign{\moveright#1\digitwidth
\vbox{\hrule width#2\digitwidth}}}





\newcommand{\RR}{\mathbb R}
\newcommand{\R}{\mathbb R}
\newcommand{\N}{\mathbb N}
\newcommand{\Z}{\mathbb Z}

\newcommand{\sagemath}{\textsf{SageMath}}


%\renewcommand{\d}{\,d\!}
\renewcommand{\d}{\mathop{}\!d}
\newcommand{\dd}[2][]{\frac{\d #1}{\d #2}}
\newcommand{\pp}[2][]{\frac{\partial #1}{\partial #2}}
\renewcommand{\l}{\ell}
\newcommand{\ddx}{\frac{d}{\d x}}

\newcommand{\zeroOverZero}{\ensuremath{\boldsymbol{\tfrac{0}{0}}}}
\newcommand{\inftyOverInfty}{\ensuremath{\boldsymbol{\tfrac{\infty}{\infty}}}}
\newcommand{\zeroOverInfty}{\ensuremath{\boldsymbol{\tfrac{0}{\infty}}}}
\newcommand{\zeroTimesInfty}{\ensuremath{\small\boldsymbol{0\cdot \infty}}}
\newcommand{\inftyMinusInfty}{\ensuremath{\small\boldsymbol{\infty - \infty}}}
\newcommand{\oneToInfty}{\ensuremath{\boldsymbol{1^\infty}}}
\newcommand{\zeroToZero}{\ensuremath{\boldsymbol{0^0}}}
\newcommand{\inftyToZero}{\ensuremath{\boldsymbol{\infty^0}}}



\newcommand{\numOverZero}{\ensuremath{\boldsymbol{\tfrac{\#}{0}}}}
\newcommand{\dfn}{\textbf}
%\newcommand{\unit}{\,\mathrm}
\newcommand{\unit}{\mathop{}\!\mathrm}
\newcommand{\eval}[1]{\bigg[ #1 \bigg]}
\newcommand{\seq}[1]{\left( #1 \right)}
\renewcommand{\epsilon}{\varepsilon}
\renewcommand{\phi}{\varphi}


\renewcommand{\iff}{\Leftrightarrow}

\DeclareMathOperator{\arccot}{arccot}
\DeclareMathOperator{\arcsec}{arcsec}
\DeclareMathOperator{\arccsc}{arccsc}
\DeclareMathOperator{\si}{Si}
\DeclareMathOperator{\scal}{scal}
\DeclareMathOperator{\sign}{sign}


%% \newcommand{\tightoverset}[2]{% for arrow vec
%%   \mathop{#2}\limits^{\vbox to -.5ex{\kern-0.75ex\hbox{$#1$}\vss}}}
\newcommand{\arrowvec}[1]{{\overset{\rightharpoonup}{#1}}}
%\renewcommand{\vec}[1]{\arrowvec{\mathbf{#1}}}
\renewcommand{\vec}[1]{{\overset{\boldsymbol{\rightharpoonup}}{\mathbf{#1}}}\hspace{0in}}

\newcommand{\point}[1]{\left(#1\right)} %this allows \vector{ to be changed to \vector{ with a quick find and replace
\newcommand{\pt}[1]{\mathbf{#1}} %this allows \vec{ to be changed to \vec{ with a quick find and replace
\newcommand{\Lim}[2]{\lim_{\point{#1} \to \point{#2}}} %Bart, I changed this to point since I want to use it.  It runs through both of the exercise and exerciseE files in limits section, which is why it was in each document to start with.

\DeclareMathOperator{\proj}{\mathbf{proj}}
\newcommand{\veci}{{\boldsymbol{\hat{\imath}}}}
\newcommand{\vecj}{{\boldsymbol{\hat{\jmath}}}}
\newcommand{\veck}{{\boldsymbol{\hat{k}}}}
\newcommand{\vecl}{\vec{\boldsymbol{\l}}}
\newcommand{\uvec}[1]{\mathbf{\hat{#1}}}
\newcommand{\utan}{\mathbf{\hat{t}}}
\newcommand{\unormal}{\mathbf{\hat{n}}}
\newcommand{\ubinormal}{\mathbf{\hat{b}}}

\newcommand{\dotp}{\bullet}
\newcommand{\cross}{\boldsymbol\times}
\newcommand{\grad}{\boldsymbol\nabla}
\newcommand{\divergence}{\grad\dotp}
\newcommand{\curl}{\grad\cross}
%\DeclareMathOperator{\divergence}{divergence}
%\DeclareMathOperator{\curl}[1]{\grad\cross #1}
\newcommand{\lto}{\mathop{\longrightarrow\,}\limits}

\renewcommand{\bar}{\overline}

\colorlet{textColor}{black}
\colorlet{background}{white}
\colorlet{penColor}{blue!50!black} % Color of a curve in a plot
\colorlet{penColor2}{red!50!black}% Color of a curve in a plot
\colorlet{penColor3}{red!50!blue} % Color of a curve in a plot
\colorlet{penColor4}{green!50!black} % Color of a curve in a plot
\colorlet{penColor5}{orange!80!black} % Color of a curve in a plot
\colorlet{penColor6}{yellow!70!black} % Color of a curve in a plot
\colorlet{fill1}{penColor!20} % Color of fill in a plot
\colorlet{fill2}{penColor2!20} % Color of fill in a plot
\colorlet{fillp}{fill1} % Color of positive area
\colorlet{filln}{penColor2!20} % Color of negative area
\colorlet{fill3}{penColor3!20} % Fill
\colorlet{fill4}{penColor4!20} % Fill
\colorlet{fill5}{penColor5!20} % Fill
\colorlet{gridColor}{gray!50} % Color of grid in a plot

\newcommand{\surfaceColor}{violet}
\newcommand{\surfaceColorTwo}{redyellow}
\newcommand{\sliceColor}{greenyellow}




\pgfmathdeclarefunction{gauss}{2}{% gives gaussian
  \pgfmathparse{1/(#2*sqrt(2*pi))*exp(-((x-#1)^2)/(2*#2^2))}%
}


%%%%%%%%%%%%%
%% Vectors
%%%%%%%%%%%%%

%% Simple horiz vectors
\renewcommand{\vector}[1]{\left\langle #1\right\rangle}


%% %% Complex Horiz Vectors with angle brackets
%% \makeatletter
%% \renewcommand{\vector}[2][ , ]{\left\langle%
%%   \def\nextitem{\def\nextitem{#1}}%
%%   \@for \el:=#2\do{\nextitem\el}\right\rangle%
%% }
%% \makeatother

%% %% Vertical Vectors
%% \def\vector#1{\begin{bmatrix}\vecListA#1,,\end{bmatrix}}
%% \def\vecListA#1,{\if,#1,\else #1\cr \expandafter \vecListA \fi}

%%%%%%%%%%%%%
%% End of vectors
%%%%%%%%%%%%%

%\newcommand{\fullwidth}{}
%\newcommand{\normalwidth}{}



%% makes a snazzy t-chart for evaluating functions
%\newenvironment{tchart}{\rowcolors{2}{}{background!90!textColor}\array}{\endarray}

%%This is to help with formatting on future title pages.
\newenvironment{sectionOutcomes}{}{}



%% Flowchart stuff
%\tikzstyle{startstop} = [rectangle, rounded corners, minimum width=3cm, minimum height=1cm,text centered, draw=black]
%\tikzstyle{question} = [rectangle, minimum width=3cm, minimum height=1cm, text centered, draw=black]
%\tikzstyle{decision} = [trapezium, trapezium left angle=70, trapezium right angle=110, minimum width=3cm, minimum height=1cm, text centered, draw=black]
%\tikzstyle{question} = [rectangle, rounded corners, minimum width=3cm, minimum height=1cm,text centered, draw=black]
%\tikzstyle{process} = [rectangle, minimum width=3cm, minimum height=1cm, text centered, draw=black]
%\tikzstyle{decision} = [trapezium, trapezium left angle=70, trapezium right angle=110, minimum width=3cm, minimum height=1cm, text centered, draw=black]





\author{Jim Talamo and Tom Needham}

\outcome{Compute dot products between $2$ and $3$-dimensional vectors.}
\outcome{Answer conceptual questions about dot products.}
\outcome{Find orthogonal decompositions of vectors.}

\title[Collaborate:]{Dot Products}

\begin{document}
\begin{abstract}
\end{abstract}
\maketitle

\section{Discussion Questions}

\begin{problem}
Let $\vec{u}$ and $\vec{v}$ be nonzero $3$-dimensional vectors. Determine whether the following statements are true or false. 

\begin{enumerate}[label=(\alph*)]
\item $\mathrm{scal}_{\vec{v}} \vec{u} \leq \left|\vec{u}\right|$.
\item If $\vec{w}$ is parallel to $\vec{v}$, then $\proj_\vec{v} \vec{u} = \proj_\vec{w} \vec{u}$.
\item $(\vec{u} - \proj_\vec{v} \vec{u}) \cdot \vec{v} = 0$ .
\item $\mathrm{scal}_\vec{u} \vec{u} = 0$ .
\item If $\proj_\vec{v} \vec{u} = \proj_\vec{u} \vec{v}$, then $\vec{u} = \vec{v}$ .
\item If $\mathrm{scal}_\vec{v} \vec{u} = \mathrm{scal}_\vec{u} \vec{v}$, then $\vec{u} = \vec{v}$ .
\end{enumerate}
\begin{freeResponse}
\begin{enumerate}[label=(\alph*)]
\item This statement is true. We have 
$$
\mathrm{scal}_{\vec{v}} \vec{u} = \frac{\vec{u} \cdot \vec{v}}{\left|\vec{v}\right|} = \left|\vec{u}\right| \cos \theta,
$$
where $\theta$ is the angle between $\vec{u}$ and $\vec{v}$. Since the range of the cosine function is $[-1,1]$, the conclusion follows.
\item This statement is true. Assume that $\vec{v} = a \vec{w}$ for some scalar $a$. Then
$$
\proj_\vec{v} \vec{u} = \frac{\vec{u} \cdot \vec{v}}{\left|\vec{v}\right|^2} \vec{v} = \frac{\vec{u} \cdot a \vec{w}}{\left|a \vec{w}\right|^2} a \vec{w} = \frac{a^2}{a^2} \frac{\vec{u} \cdot \vec{w}}{\left|\vec{w}\right|^2} \vec{w} = \proj_\vec{w} \vec{u}.
$$
\item This statement is true. We have
$$
(\vec{u} - \proj_\vec{v} \vec{u}) \cdot \vec{v} = \vec{u} \cdot \vec{v} - \frac{\vec{u} \cdot \vec{v}}{\left|\vec{v}\right|^2} \vec{v} \cdot \vec{v} = \vec{u} \cdot \vec{v} - \frac{\vec{u} \cdot \vec{v}}{\left|\vec{v}\right|^2} \left|\vec{v}\right|^2 = \vec{u} \cdot \vec{v} - \vec{u} \cdot \vec{v} = 0.
$$
\item This statement is false. For any nonzero $\vec{u}$, we have
$$
\mathrm{scal}_\vec{u} \vec{u} = \frac{\vec{u} \cdot \vec{u}}{\left|\vec{u}\right|} = \frac{\left|\vec{u}\right|^2}{\left|\vec{u}\right|} = \left|\vec{u}\right| \neq 0.
$$
\item This statement is false; a counterexample is given by any pair of nonzero, orthogonal vectors. 
\item This statement is false, with the same counterexample used in the previous statement. 
\end{enumerate}
\end{freeResponse}
\end{problem}

%%%%%%%%%%%%%%%%%%%%%%%%%%%%%%%%%%%%%%%%%%%%%%%%%%%%%

\section{Group Work}

\begin{problem}
Let $\vec{u} = \left<1,0,2\right>$, $\vec{v} = \left<-1,1,2\right>$ and $\vec{w} = \left<2,2,0\right>$. 
\begin{itemize}
\item[I.] Which pairs of vectors listed above are orthogonal?
\item[II.] Which pairs of vectors have interior angle between them less than $\frac{\pi}{2}$?
\end{itemize}
\begin{freeResponse}
Calculating dot products explicitly, we have
\begin{align*}
\vec{u} \cdot \vec{v} &=  1 \cdot -1 + 0 \cdot 1 + 2 \cdot 2 = 3 \neq 0\\
\vec{u} \cdot \vec{w} &= 1 \cdot 2 + 0 \cdot 2 + 2 \cdot 0 = 2 \neq 0 \\
\vec{v} \cdot \vec{w} &= -1 \cdot 2 + 1 \cdot 2 + 2 \cdot 0 = 0.
\end{align*}
It follows that $\vec{v}$ and $\vec{w}$ are the unique pair which are orthogonal.

II. A pair of vectors has interior angle less than $\frac{\pi}{2}$ if and only if their dot product is positive. Therefore the pairs $\vec{u}$ and $\vec{v}$ and $\vec{u}$ and $\vec{w}$ have this property.
\end{freeResponse}
\end{problem}

%%%%%%%%%%%%%%%%%%%%%%%%%%%%%%%%%%%%%%%%%%%%%%%%%%%%%

\begin{problem}
Let $\vec{u} = \left<2,a,6\right>$ and $\vec{v} = \left<-1,2,a\right>$ for some number $a$. 
\begin{itemize}
\item[I.] Find $a$ so that $\vec{u}$ and $\vec{v}$ are parallel or explain why no such $a$-value exists.
\item[II.] Find $a$ so that $\vec{u}$ and $\vec{v}$ are orthogonal or explain why no such $a$-value exists.
\end{itemize}

\begin{freeResponse}
I. Since the first coordinate of $\vec{u}$ is $2$ and the first coordinate of $\vec{v}$ is $-1$, the vectors could only be parallel if $\vec{u} = -2 \cdot \vec{v}$. Considering the second and third coordinates of the vectors, this condition would force $-2 \cdot 2 = a$ and $-2 \cdot a = 6$. Since these equations cannot be satisfied simultaneously, it must be that there is no number $a$ which would make the vectors $\vec{u}$ and $\vec{v}$ parallel.

II. We have
$$
\vec{u} \cdot \vec{v} = 2 \cdot (-1) + a \cdot 2 + 6 \cdot a = -2 + 8a.
$$
For the vectors to be orthogonal, we need
$$
-2 + 8 a = 0.
$$
The number $a = \frac{1}{4}$ produces orthogonal vectors $\vec{u}$ and $\vec{v}$. 
\end{freeResponse}
\end{problem}

%%%%%%%%%%%%%%%%%%%%%%%%%%%%%%%%%%%%%%%%%%%%%%%%%%%%%

\begin{problem}
Let $\vec{u} = \left<-2,2\right>$ and $\vec{v} = \left<1,4\right>$. 
\begin{itemize}
\item[I.] Sketch $\vec{u}$, $\vec{v}$ and $\proj_{\vec{v}} \vec{u}$ on the axes below.  Then, compute $\proj_\vec{v} \vec{u}$. 

\resizebox {6cm} {!} {   \begin{tikzpicture}  
    \begin{axis}[  
        xmin=-5,  
        xmax=5,  
        ymin=-5,  
        ymax=5,  
        axis lines=center,  
        xlabel=$x$,  
        ylabel=$y$,  
        every axis y label/.style={at=(current axis.above origin),anchor=south},  axis on top
        every axis x label/.style={at=(current axis.right of origin),anchor=west},  axis on top
      ]  
      
            \end{axis}  
  \end{tikzpicture}  }



\item[II.] Find a vector $\vec{p}$ parallel to $\vec{v}$ and a vector $\vec{n}$ orthogonal to $\vec{v}$ so that $\vec{u} = \vec{p} + \vec{n}$ and sketch $\vec{u}$, $\vec{v}$, $\vec{p}$ and $\vec{n}$ on the axes below.


\resizebox {6cm} {!} {   \begin{tikzpicture}  
    \begin{axis}[  
        xmin=-5,  
        xmax=5,  
        ymin=-5,  
        ymax=5,  
        axis lines=center,  
        xlabel=$x$,  
        ylabel=$y$,  
        every axis y label/.style={at=(current axis.above origin),anchor=south},  axis on top
        every axis x label/.style={at=(current axis.right of origin),anchor=west},  axis on top
      ]  
      
            \end{axis}  
  \end{tikzpicture}  }

\end{itemize}

\begin{freeResponse}
I. The projection vector is 
$$
\proj_\vec{v} \vec{u} = \frac{\left<-2,2\right> \cdot \left<1,4\right>}{\left|\left<1,4\right>\right|^2} \left<1,4\right> = \frac{-2 + 8}{1^2 + 4^2} \left<1,4\right>  = \frac{6}{17}\left<1,4\right>.
$$

II. Let $\vec{p} = \proj_\vec{v} \vec{u}$, as calculated above, and let 
$$
\vec{n} = \vec{u} - \vec{p} = \left<-2,2\right> - \frac{6}{17} \left<1,4\right> = \left<-\frac{40}{17},\frac{10}{17}\right>.
$$
One can check by a calculation that $\vec{p}$ is parallel to $\vec{v}$ and $\vec{n}$ is perpendicular to $\vec{v}$. This also follows by general principles: $\vec{p}$ is parallel to $\vec{v}$ by definition, and $\vec{n}$ is perpendicular to $\vec{v}$ by Problem 1.
\end{freeResponse}
\end{problem}

\begin{problem}
Determine necessary and sufficient conditions on vectors $\vec{u}$ and $\vec{v}$ for $\proj_\vec{v} \vec{u} = \proj_\vec{u} \vec{v}$. (Hint: consider the case where $\vec{u}$ and $\vec{v}$ are orthogonal separately.)

\begin{freeResponse}
If $\vec{u}$ and $\vec{v}$ are orthogonal, then
$$
\proj_\vec{v} \vec{u} = \proj_\vec{u} \vec{v} = \vec{0}.
$$
In fact, orthogonality of $\vec{u}$ and $\vec{v}$ is both a necessary and sufficient for one (and hence both) projections to be $\vec{0}$. 

Now suppose that $\vec{u}$ and $\vec{v}$ are not orthogonal. Then, note that $\proj_\vec{v} \vec{u}$ is parallel to $\vec{v}$, while $\proj_\vec{u} \vec{v}$ is parallel to $\vec{u}$.  Thus, $\proj_\vec{v} \vec{u}=\proj_\vec{u} \vec{v}$ holds if and only if $\vec{u}$ and $\vec{v}$ are parallel.

Since $\vec{u}$ and $\vec{v}$ are parallel and nonzero, there must be a nonzero constant $c$ so $\vec{u} = c \vec{v}$.  We can rewrite both projections using this fact.

\begin{itemize}
\item Notice that for $\proj_{\vec{v} } \vec{u}$, we have:

\[
\proj_{\vec{v} } \vec{u} = \proj_{\vec{v} } \left(c \vec{v}\right) = \left[\frac{\vec{v} \dotp c \vec{v}}{\vec{v} \dotp \vec{v} }\right] \vec{v} = c \cdot \left[\frac{\cancel{\vec{v} \dotp  \vec{v}}}{\cancel{\vec{v} \dotp \vec{v} }}\right] \vec{v} = c\vec{v}.
\]
\item Notice that for $\proj_{\vec{u} } \vec{v}$, we have:

\[
\proj_{\vec{u} } \vec{v} = \proj_{c\vec{v} } \vec{v} = \left[\frac{c \vec{v} \dotp \vec{v}}{c \vec{v} \dotp c \vec{v} }\right] \cdot c \vec{v} = \frac{c^2}{c^2} \cdot \left[\frac{\cancel{\vec{v} \dotp  \vec{v}}}{\cancel{\vec{v} \dotp \vec{v} }}\right] \vec{v} = \vec{v}.
\]
\end{itemize}


Thus, when $\vec{u}$ and $\vec{v}$ are not orthogonal, $\proj_\vec{v} \vec{u}=\proj_\vec{u} \vec{v}$ holds if and only if $c\vec{v} = \vec{v}$, so $c=1$.  Since $\vec{u} = c\vec{v}$ and $c=1$, we have that $\vec{u} = \vec{v}$.

\end{freeResponse}
\end{problem}

\end{document}
