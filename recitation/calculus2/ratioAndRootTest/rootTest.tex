\documentclass[noauthor,handout]{ximera}
%handout:  for handout version with no solutions or instructor notes
%handout,instructornotes:  for instructor version with just problems and notes, no solutions
%noinstructornotes:  shows only problem and solutions

%% handout
%% space
%% newpage
%% numbers
%% nooutcomes

%I added the commands here so that I would't have to keep looking them up
%\newcommand{\RR}{\mathbb R}
%\renewcommand{\d}{\,d}
%\newcommand{\dd}[2][]{\frac{d #1}{d #2}}
%\renewcommand{\l}{\ell}
%\newcommand{\ddx}{\frac{d}{dx}}
%\everymath{\displaystyle}
%\newcommand{\dfn}{\textbf}
%\newcommand{\eval}[1]{\bigg[ #1 \bigg]}

%\begin{image}
%\includegraphics[trim= 170 420 250 180]{Figure1.pdf}
%\end{image}

%add a ``.'' below when used in a specific directory.

%\usepackage{todonotes}
%\usepackage{mathtools} %% Required for wide table Curl and Greens
%\usepackage{cuted} %% Required for wide table Curl and Greens
\newcommand{\todo}{}

\usepackage{esint} % for \oiint
\ifxake%%https://math.meta.stackexchange.com/questions/9973/how-do-you-render-a-closed-surface-double-integral
\renewcommand{\oiint}{{\large\bigcirc}\kern-1.56em\iint}
\fi


\graphicspath{
  {./}
  {ximeraTutorial/}
  {basicPhilosophy/}
  {functionsOfSeveralVariables/}
  {normalVectors/}
  {lagrangeMultipliers/}
  {vectorFields/}
  {greensTheorem/}
  {shapeOfThingsToCome/}
  {dotProducts/}
  {partialDerivativesAndTheGradientVector/}
  {../productAndQuotientRules/exercises/}
  {../motionAndPathsInSpace/exercises/}
  {../normalVectors/exercisesParametricPlots/}
  {../continuityOfFunctionsOfSeveralVariables/exercises/}
  {../partialDerivativesAndTheGradientVector/exercises/}
  {../directionalDerivativeAndChainRule/exercises/}
  {../commonCoordinates/exercisesCylindricalCoordinates/}
  {../commonCoordinates/exercisesSphericalCoordinates/}
  {../greensTheorem/exercisesCurlAndLineIntegrals/}
  {../greensTheorem/exercisesDivergenceAndLineIntegrals/}
  {../shapeOfThingsToCome/exercisesDivergenceTheorem/}
  {../greensTheorem/}
  {../shapeOfThingsToCome/}
  {../separableDifferentialEquations/exercises/}
  {vectorFields/}
}

\newcommand{\mooculus}{\textsf{\textbf{MOOC}\textnormal{\textsf{ULUS}}}}

\usepackage{tkz-euclide}\usepackage{tikz}
\usepackage{tikz-cd}
\usetikzlibrary{arrows}
\tikzset{>=stealth,commutative diagrams/.cd,
  arrow style=tikz,diagrams={>=stealth}} %% cool arrow head
\tikzset{shorten <>/.style={ shorten >=#1, shorten <=#1 } } %% allows shorter vectors

\usetikzlibrary{backgrounds} %% for boxes around graphs
\usetikzlibrary{shapes,positioning}  %% Clouds and stars
\usetikzlibrary{matrix} %% for matrix
\usepgfplotslibrary{polar} %% for polar plots
\usepgfplotslibrary{fillbetween} %% to shade area between curves in TikZ
\usetkzobj{all}
\usepackage[makeroom]{cancel} %% for strike outs
%\usepackage{mathtools} %% for pretty underbrace % Breaks Ximera
%\usepackage{multicol}
\usepackage{pgffor} %% required for integral for loops



%% http://tex.stackexchange.com/questions/66490/drawing-a-tikz-arc-specifying-the-center
%% Draws beach ball
\tikzset{pics/carc/.style args={#1:#2:#3}{code={\draw[pic actions] (#1:#3) arc(#1:#2:#3);}}}



\usepackage{array}
\setlength{\extrarowheight}{+.1cm}
\newdimen\digitwidth
\settowidth\digitwidth{9}
\def\divrule#1#2{
\noalign{\moveright#1\digitwidth
\vbox{\hrule width#2\digitwidth}}}





\newcommand{\RR}{\mathbb R}
\newcommand{\R}{\mathbb R}
\newcommand{\N}{\mathbb N}
\newcommand{\Z}{\mathbb Z}

\newcommand{\sagemath}{\textsf{SageMath}}


%\renewcommand{\d}{\,d\!}
\renewcommand{\d}{\mathop{}\!d}
\newcommand{\dd}[2][]{\frac{\d #1}{\d #2}}
\newcommand{\pp}[2][]{\frac{\partial #1}{\partial #2}}
\renewcommand{\l}{\ell}
\newcommand{\ddx}{\frac{d}{\d x}}

\newcommand{\zeroOverZero}{\ensuremath{\boldsymbol{\tfrac{0}{0}}}}
\newcommand{\inftyOverInfty}{\ensuremath{\boldsymbol{\tfrac{\infty}{\infty}}}}
\newcommand{\zeroOverInfty}{\ensuremath{\boldsymbol{\tfrac{0}{\infty}}}}
\newcommand{\zeroTimesInfty}{\ensuremath{\small\boldsymbol{0\cdot \infty}}}
\newcommand{\inftyMinusInfty}{\ensuremath{\small\boldsymbol{\infty - \infty}}}
\newcommand{\oneToInfty}{\ensuremath{\boldsymbol{1^\infty}}}
\newcommand{\zeroToZero}{\ensuremath{\boldsymbol{0^0}}}
\newcommand{\inftyToZero}{\ensuremath{\boldsymbol{\infty^0}}}



\newcommand{\numOverZero}{\ensuremath{\boldsymbol{\tfrac{\#}{0}}}}
\newcommand{\dfn}{\textbf}
%\newcommand{\unit}{\,\mathrm}
\newcommand{\unit}{\mathop{}\!\mathrm}
\newcommand{\eval}[1]{\bigg[ #1 \bigg]}
\newcommand{\seq}[1]{\left( #1 \right)}
\renewcommand{\epsilon}{\varepsilon}
\renewcommand{\phi}{\varphi}


\renewcommand{\iff}{\Leftrightarrow}

\DeclareMathOperator{\arccot}{arccot}
\DeclareMathOperator{\arcsec}{arcsec}
\DeclareMathOperator{\arccsc}{arccsc}
\DeclareMathOperator{\si}{Si}
\DeclareMathOperator{\scal}{scal}
\DeclareMathOperator{\sign}{sign}


%% \newcommand{\tightoverset}[2]{% for arrow vec
%%   \mathop{#2}\limits^{\vbox to -.5ex{\kern-0.75ex\hbox{$#1$}\vss}}}
\newcommand{\arrowvec}[1]{{\overset{\rightharpoonup}{#1}}}
%\renewcommand{\vec}[1]{\arrowvec{\mathbf{#1}}}
\renewcommand{\vec}[1]{{\overset{\boldsymbol{\rightharpoonup}}{\mathbf{#1}}}\hspace{0in}}

\newcommand{\point}[1]{\left(#1\right)} %this allows \vector{ to be changed to \vector{ with a quick find and replace
\newcommand{\pt}[1]{\mathbf{#1}} %this allows \vec{ to be changed to \vec{ with a quick find and replace
\newcommand{\Lim}[2]{\lim_{\point{#1} \to \point{#2}}} %Bart, I changed this to point since I want to use it.  It runs through both of the exercise and exerciseE files in limits section, which is why it was in each document to start with.

\DeclareMathOperator{\proj}{\mathbf{proj}}
\newcommand{\veci}{{\boldsymbol{\hat{\imath}}}}
\newcommand{\vecj}{{\boldsymbol{\hat{\jmath}}}}
\newcommand{\veck}{{\boldsymbol{\hat{k}}}}
\newcommand{\vecl}{\vec{\boldsymbol{\l}}}
\newcommand{\uvec}[1]{\mathbf{\hat{#1}}}
\newcommand{\utan}{\mathbf{\hat{t}}}
\newcommand{\unormal}{\mathbf{\hat{n}}}
\newcommand{\ubinormal}{\mathbf{\hat{b}}}

\newcommand{\dotp}{\bullet}
\newcommand{\cross}{\boldsymbol\times}
\newcommand{\grad}{\boldsymbol\nabla}
\newcommand{\divergence}{\grad\dotp}
\newcommand{\curl}{\grad\cross}
%\DeclareMathOperator{\divergence}{divergence}
%\DeclareMathOperator{\curl}[1]{\grad\cross #1}
\newcommand{\lto}{\mathop{\longrightarrow\,}\limits}

\renewcommand{\bar}{\overline}

\colorlet{textColor}{black}
\colorlet{background}{white}
\colorlet{penColor}{blue!50!black} % Color of a curve in a plot
\colorlet{penColor2}{red!50!black}% Color of a curve in a plot
\colorlet{penColor3}{red!50!blue} % Color of a curve in a plot
\colorlet{penColor4}{green!50!black} % Color of a curve in a plot
\colorlet{penColor5}{orange!80!black} % Color of a curve in a plot
\colorlet{penColor6}{yellow!70!black} % Color of a curve in a plot
\colorlet{fill1}{penColor!20} % Color of fill in a plot
\colorlet{fill2}{penColor2!20} % Color of fill in a plot
\colorlet{fillp}{fill1} % Color of positive area
\colorlet{filln}{penColor2!20} % Color of negative area
\colorlet{fill3}{penColor3!20} % Fill
\colorlet{fill4}{penColor4!20} % Fill
\colorlet{fill5}{penColor5!20} % Fill
\colorlet{gridColor}{gray!50} % Color of grid in a plot

\newcommand{\surfaceColor}{violet}
\newcommand{\surfaceColorTwo}{redyellow}
\newcommand{\sliceColor}{greenyellow}




\pgfmathdeclarefunction{gauss}{2}{% gives gaussian
  \pgfmathparse{1/(#2*sqrt(2*pi))*exp(-((x-#1)^2)/(2*#2^2))}%
}


%%%%%%%%%%%%%
%% Vectors
%%%%%%%%%%%%%

%% Simple horiz vectors
\renewcommand{\vector}[1]{\left\langle #1\right\rangle}


%% %% Complex Horiz Vectors with angle brackets
%% \makeatletter
%% \renewcommand{\vector}[2][ , ]{\left\langle%
%%   \def\nextitem{\def\nextitem{#1}}%
%%   \@for \el:=#2\do{\nextitem\el}\right\rangle%
%% }
%% \makeatother

%% %% Vertical Vectors
%% \def\vector#1{\begin{bmatrix}\vecListA#1,,\end{bmatrix}}
%% \def\vecListA#1,{\if,#1,\else #1\cr \expandafter \vecListA \fi}

%%%%%%%%%%%%%
%% End of vectors
%%%%%%%%%%%%%

%\newcommand{\fullwidth}{}
%\newcommand{\normalwidth}{}



%% makes a snazzy t-chart for evaluating functions
%\newenvironment{tchart}{\rowcolors{2}{}{background!90!textColor}\array}{\endarray}

%%This is to help with formatting on future title pages.
\newenvironment{sectionOutcomes}{}{}



%% Flowchart stuff
%\tikzstyle{startstop} = [rectangle, rounded corners, minimum width=3cm, minimum height=1cm,text centered, draw=black]
%\tikzstyle{question} = [rectangle, minimum width=3cm, minimum height=1cm, text centered, draw=black]
%\tikzstyle{decision} = [trapezium, trapezium left angle=70, trapezium right angle=110, minimum width=3cm, minimum height=1cm, text centered, draw=black]
%\tikzstyle{question} = [rectangle, rounded corners, minimum width=3cm, minimum height=1cm,text centered, draw=black]
%\tikzstyle{process} = [rectangle, minimum width=3cm, minimum height=1cm, text centered, draw=black]
%\tikzstyle{decision} = [trapezium, trapezium left angle=70, trapezium right angle=110, minimum width=3cm, minimum height=1cm, text centered, draw=black]




\author{Jim Talamo}

\outcome{Use the root test to determine whether a series converges or diverges.}
\outcome{Use the root test to determine whether a series converges or diverges.}

\title[]{The Root Test}

\begin{document}
\begin{abstract}
\end{abstract}
\maketitle

\vspace{-0.5in}

\section{Discussion Questions}

\begin{problem}
Suppose $\{a_n\}_{n=1}$ is a sequence of positive terms and $\lim_{n \to \infty} \sqrt[n]{a_n} = \frac{3}{4}$ and let $p>0$.  
\begin{itemize}
\item[I.] Determine whether $\sum_{k=1}^{\infty} k^p a_k$ converges or diverges.  
\item[II.] Suppose that $\{a_n\}_{n=1}$ is any series for which $\sum_{k=1}^{\infty} a_k$ converges by the root test.  Does multiplying $a_k$ by any polynomial in $k$ affect convergence?  That is, if $p(x)$ is a polynomial, does $\sum_{k=1}^{\infty} p(k)a_k$ still converge?
\item[III.] Suppose that we want to determine whether $\sum_{k=1}^\infty \frac{k^2+6k}{5k^3 + 2k^2 + 1}$ converges.  Would the root test be conclusive here?
\end{itemize}

\begin{freeResponse}
I. We use root test to determine if $\sum_{k=1}^{\infty} k^p a_k$ converges.  Note that by setting $b_n = n^p a_n$, we find

\[
L = \lim_{n \to \infty}\sqrt[n]{ \left|b_n\right|} = \lim_{n \to \infty} \sqrt[n]{n^p \cdot a_n}= \lim_{n \to \infty} \sqrt[n]{n^p} \cdot \sqrt[n]{a_n}
\]

Note the following.

\begin{itemize}
\item $\lim_{n \to \infty} \sqrt[n]{n^p} = 1$ (you are asked to verify this later).
\item  $\lim_{n \to \infty} \sqrt[n]{a_n} = \frac{3}{4}$ is given.
\end{itemize}

Thus, we find that

\[
L= \lim_{n \to \infty} \sqrt[n]{n^p} \cdot \sqrt[n]{a_n} = \frac{3}{4}.
\]

The series $\sum_{k=1}^{\infty} k^p a_k$ thus converges by root test.  Note that the polynomial term $n^p$ does not affect the limit $L$.

II. The presence of the polynomial will not affect the convergence; that is if  $\sum_{k=1}^{\infty} a_k$ converges by the root test, so too will $\sum_{k=1}^{\infty} p(k) a_k$.  To verify this, we would have to compute 

\[
L = \lim_{n \to \infty}\sqrt[n]{ \left|p(n) \cdot a_n\right|} .
\]

Since we have seen that for the purpose of limits, polynomials can be treated by only considering the highest degree term, this limit will be equivalent to the limit we just computed by letting $p$ be the power of the highest degree term in the polynomial.

III. The root test would not be conclusive here; since we only have polynomial terms, we know without doing any computation that $\lim_{n \to \infty}\sqrt[n]{ \left|a_n\right|}$ will be $1$.  You may verify this by explicitly computing the limit if you would like!

\end{freeResponse}
\end{problem}


\begin{problem}
Suppose $a_n>0$ for all $n \geq 1$ and $\lim_{k \rightarrow \infty}\sqrt[k]{a_k} = \frac{1}{2}.$

\begin{itemize}
\item[I.] Does $\sum_{k=1}^\infty a_k$ converge?
\item[II.] Does $\sum_{k=1}^\infty\sqrt[k]{a_k}$ converge?
\item[III.] A student claims that $\sum_{k=1}^\infty a_k$ converges to $\frac{1}{2}$ by root test. Is the student correct?
\end{itemize}

\begin{freeResponse}
A lot of information is being stored in the notation used in these problems.  A good strategy before tackling a problem is to take a moment and parse what information is being given and what you are being asked to do with it.

I. Since $\lim_{k \rightarrow \infty}\sqrt[k]{a_k} = \frac{1}{2}<1$, the root test tells us that $\sum_{k=1}^\infty a_k$ converges.

\begin{remark}
Since the limit of the \emph{sequence} $\left\{\sqrt[n]{a_n} \right\}$ is $\frac{1}{2}$, the root test tells us that the \emph{series} $\sum_{k=1}^\infty a_k$ converges.  Note that we are using the limit of a completely different sequence than $\{a_n\}$ in order to determine if we can sum all of the terms in $\{a_n\}$.
\end{remark}


II. Since $\lim_{k \rightarrow \infty}\sqrt[k]{a_k} \neq 0$, the divergence test tells us that $\sum_{k=1}^\infty\sqrt[k]{a_k}$ diverges.


\begin{remark}
Note that the divergence test tells us that if the terms in a \emph{sequence} do not tend to $0$, then the sum of all infinitely many terms in the sequence will diverge.  Here, we have information about the limit of the sequence $\left\{\sqrt[n]{a_n} \right\}$, so divergence test applies to the sum of the terms of this sequence.
\end{remark}

III. The student is not correct; the root test can only tell us that a series converges, not the value to which it converges. 

To give a concrete counterexample to the student's claim, consider the series 
$$
\sum_{k=1}^\infty \frac{1}{2^k}.
$$
The terms $a_k = \frac{1}{2^k}$ of this series satisfy $\lim_{k \rightarrow \infty}\sqrt[k]{a_k} = \frac{1}{2}$, but in this case we can determine the value of the series exactly since

$$
\sum_{k=1}^\infty \frac{1}{2^k} = \sum_{k=0}^\infty \frac{1}{2^k} - 1 = \frac{1}{1-1/2} - 1 = 1 \neq \frac{1}{2}.
$$


\end{freeResponse}
\end{problem}

%%%%%%%%%%%%%%%%%%%%%%%%%%%%%%%%%%%%%%%
\section{Group Work}

\begin{problem}
Suppose that $p>0$.  

\begin{itemize}
\item[I.] Explicitly verify that $\lim_{n \to \infty} \sqrt[n]{n^p}=1$.  
\item[II.] Argue why if $p(x)$ is any polynomial, then $\lim_{n \to \infty} \sqrt[n]{|p(n)|} =1$.
\end{itemize}

\begin{freeResponse}
I Note that trying to evaluate the limit directly gives the indeterminate form $\infty^0$.  Setting $L= \lim_{n \to \infty} \sqrt[n]{n^p}$ and noting that 

\[
\sqrt[n]{n^p} = \left(n^p\right)^{1/n} = n^{p/n},
\]

we take the natural logarithm of both sides, and proceed.

\begin{align*}
\ln(L) & = \ln\left( \lim_{n \to \infty} n^{p/n} \right) \\
\ln(L) & =  \lim_{n \to \infty} \ln\left(n^{p/n} \right) \\
\ln(L) & =  \lim_{n \to \infty} \frac{p}{n}\ln(n) \\
\ln(L) & =  \lim_{n \to \infty} \frac{p\ln(n)}{n} \\
\end{align*}
This is now in a form where L'Hopital's Rule applies, so we use it then continue.

\begin{align*}
\ln(L) & =  \lim_{n \to \infty} \frac{p\cdot{1/n}}{1} \\
\ln(L) & = 0 \\
\end{align*}

Since $\ln(L) = 0$, we have $L=e^0 = 1$.


So, we have established that $\lim_{n \to \infty} \sqrt[n]{n^p}=1$.  

II. Now, suppose that $p(x)$ is \emph{any} polynomial of degree $m$.  We may write

\[
p(x) = a_mx^m+a_{m-1}x^{m-1}+ \ldots + a_0.
\]  

Now, the limit to compute can be exhibited and evaluated.

\begin{align*}
\lim_{n \to \infty} \sqrt[n]{|p(n)|} &= \lim_{n \to \infty} \sqrt[n]{\left|a_mn^m+a_{m-1}n^{m-1}+ \ldots + a_0\right|} \\
&= \lim_{n \to \infty} \sqrt[n]{\left|n^m\right| \cdot \left|a_m+\frac{a_{m-1}}{n}+ \ldots + \frac{a_0}{n^m}\right|} \\
&= \lim_{n \to \infty} \sqrt[n]{\left|n^m\right|} \cdot  \sqrt[n]{\left|a_m+\frac{a_{m-1}}{n}+ \ldots + \frac{a_0}{n^m}\right|} \\
\end{align*}

Note the following.
\begin{itemize}
\item $\lim_{n \to \infty} \sqrt[n]{\left|n^m\right|}$ by the computation that we just did in the previous part.
\item As $n \to \infty$, all of the terms with an $n$ in the denominator vanish, so 

\[
\lim_{n \to \infty} \sqrt[n]{\left|a_m+\frac{a_{m-1}}{n}+ \ldots + \frac{a_0}{n^m}\right|}  = \lim_{n \to \infty} \sqrt[n]{\left|a_m\right|} = 1.
\]

Thus, $\lim_{n \to \infty} \sqrt[n]{|p(n)|}  \lim_{n \to \infty} \sqrt[n]{\left|n^m\right|} \cdot  \sqrt[n]{\left|a_m+\frac{a_{m-1}}{n}+ \ldots + \frac{a_0}{n^m}\right|} = 1 \cdot 1 =1$.

This establishes the facts in the first problem on this handout explicitly.
\end{itemize}
\end{freeResponse}
\end{problem}

%\begin{problem}
%Suppose that the sequence $\{a_n\}_{n=1}^\infty$ has sequence of partial sums $\{s_n\}_{n=1}^\infty$ given by the formula
%$$
%s_n = \frac{3^n}{n!}.
%$$
%\begin{itemize}
%\item[I.] Does $\sum_{k=1}^\infty a_k$ converge?
%\item[II.] Does $\sum_{k=1}^\infty s_k$ converge?
%\end{itemize}
%
%\begin{freeResponse}
%I. By definition,
%$$
%\sum_{k=1}^\infty a_k = \lim_{n\rightarrow \infty} s_n = \lim_{n\rightarrow \infty} \frac{3^n}{n!} = 0,
%$$
%where the limit is computed by comparing growth rates. 
%
%II. The series $\sum_{n=1}^{\infty} s_n$ converges by the root test, since
%\begin{align*}
%\lim_{n \rightarrow \infty} \left|\frac{s_{n+1}}{s_n} \right| &= \lim_{n \rightarrow \infty} \frac{3^{n+1}}{(n+1)!} \cdot \frac{n!} {3^n} \\
%&= \lim_{n \rightarrow \infty} \frac{3^{n+1}}{3^n} \frac{n!}{(n+1)!} \\
%&= \lim_{n \rightarrow \infty} \frac{\cancel{3^n} \cdot 3^1}{\cancel{3^n}} \frac{\cancel{n!}}{(n+1) \cdot\cancel{n!}} \\
%&= \lim_{n \rightarrow \infty} \frac{3}{n+1} \\
%&= 0.
%\end{align*}
%The series $\sum_{n=1}^{\infty} s_n$ converges, but we don't have the tools necessary yet to determine its value. 
%\end{freeResponse}
%
%
%\end{problem}


\begin{problem}
Consider the series $\sum_{k=1}^\infty \left(\frac{2k+3}{3k+1}\right)^{3k}$.  

\begin{itemize}
\item[I.] Explain why the root test is preferable to the ratio test, then use the root test to determine if the series converges.
\item[II.] Let $s_n = \sum_{k=1}^n \left(\frac{2k+3}{3k+1}\right)^{3k}$.  Is $\{s_n\}_{n=1}$ bounded?  Is it monotonic?
\end{itemize}

\begin{freeResponse}

I. Because the terms in the series have a $k$ in the exponent, the root test will give a limit that is more efficient to compute.  Set $a_k = \left(\frac{2k+3}{3k+1}\right)^{3k}$.

\begin{align*}
L = \lim_{n \rightarrow \infty} \left|\sqrt[n]{a_n} \right| &= \lim_{n \rightarrow \infty}\sqrt[n]{\left(\frac{2n+3}{3n+1}\right)^{3n}} \\
&= \lim_{n \rightarrow \infty} \left[\left(\frac{2n+3}{3n+1}\right)^{3\cancel{n}}\right]^{\frac{1}{\cancel{n}}} \\
&= \lim_{n \rightarrow \infty} \left(\frac{2n+3}{3n+1}\right)^3 \\
& = \left(\frac{2}{3}\right)^3 \\
&= \frac{8}{27}
\end{align*}
Since the limit above is less than $1$, it follows that the series $\sum_{k=1}^\infty \left(\frac{2k+3}{3k+1}\right)^{3k}$ converges by the root test.

II. Note that $\left(\frac{2k+3}{3k+1}\right)^{3k}>0$ for all $k$ so

\[
s_{n+1} = s_n + \left(\frac{2n+3}{3n+1}\right)^{3n} >s_n+0.
\]
Hence, $s_{n+1}>s_n$ for all $n$, so $s_n$ is monotone increasing.

Also, since $\sum_{k=1}^\infty \left(\frac{2k+3}{3k+1}\right)^{3k}$ converges, $\lim_{n \to \infty} s_n$ exists.  Since $\lim_{n \to \infty} s_n$ exists, $\{s_n\}_{n=1}$ is bounded.
\end{freeResponse}
\end{problem}

\begin{problem}
Determine whether the series $\sum_{k=1}^\infty 3^{2k} k^{2-3k}$ converges or diverges.

\begin{freeResponse}
I.  We can try to apply root test, but a little preliminary algebra is a good first step.

\[
3^{2k} \cdot k^{2-3k} = k^2\cdot k^{-3k} \cdot \left(3^2\right)^k= k^2 \cdot 9^k \left(k^{-3}\right)^k = k^2\cdot 9^k \cdot \left(\frac{1}{k^3}\right)^k = k^2\cdot \left(\frac{9}{k^3}\right)^k
\]

We now apply root test.

\begin{align*}
L &= \lim_{n \to \infty} \sqrt[n]{ n^2\cdot \left(\frac{9}{n^3}\right)^n} \\
&= \lim_{n \to \infty} \sqrt[n]{ n^2} \cdot \sqrt[n]{\left(\frac{9}{n^3}\right)^n} \\
&= \lim_{n \to \infty} \sqrt[n]{ n^2} \cdot \frac{9}{n^3} \\
&= 1 \cdot 0
\end{align*}

Hence, $\sum_{k=1}^\infty 3^{2k}k^{2-3k}$ converges by the root test.

\end{freeResponse}
\end{problem}


\begin{problem}
This problem explores a ``fringe case" of geometric series.  Suppose that $a_n = \left(1-\frac{1}{n}\right)^n$ and $b_n = \left(1-\frac{1}{n}\right)^{n^2}$.  

You may take as a fact that $\lim_{n \to \infty} \left(1+\frac{a}{n}\right)^n = e^a$.

\begin{itemize}
\item[I.] Note that the limit of the base in each case is $1$, but the exponent $k^2$ in $b_k$ grows more quickly than the exponent $k$ in $a_k$.  What effect, if any do you think this will have on $\lim_{n \to \infty} b_n$?
\item[II.] Determine whether $\sum_{k=1}^{\infty} a_k$ and $\sum_{k=1}^{\infty} b_k$ will converge or diverge.  Does the rate at which the exponent grows affect the outcome?
\end{itemize}

\begin{freeResponse}
I. Note that the exponential indeterminate form in the sequences is $1^{\infty}$.  This is indeterminate because if we think about the limits separately, the base wants the limit to be $1$, but since the base is always less than $1$, the exponent wants the limit to be $0$. 

Since $\lim_{n \to \infty} \left(1+\frac{a}{n}\right)^n = e^a$, setting $a=-1$ gives that $\lim_{n \to \infty} \left(1-\frac{1}{n}\right)^n = \frac{1}{e}.$  In this sense, the base and the exponent ``compromise'' on a value between $0$ and $1$.  

For $b_k$, the exponent now grows faster, so if anything, the exponent should play a more prevalent role in determining the limit.  Indeed, note that by setting $L = \lim_{n \to \infty} \left(1-\frac{1}{n}\right)^{n^2}$, we have

\begin{align*}
\ln(L) &= \ln\left(  \lim_{n \to \infty} \left(1-\frac{1}{n}\right)^{n^2} \right) \\
\ln(L) &= \lim_{n \to \infty} \ln\left(\left(1-\frac{1}{n}\right)^{n^2} \right) \\
\ln(L) &= \lim_{n \to \infty} n^2 \ln \left(1-\frac{1}{n}\right) \\
\ln(L) &= \lim_{n \to \infty} \frac{\ln \left(1-\frac{1}{n}\right)}{1/n^2} \\
\ln(L) &= \lim_{n \to \infty} \frac{\frac{1}{1-1/n} \cdot \frac{1}{n^2}}{-2/n^3} \textrm{ by L'Hopital's Rule. } \\
\ln(L) &= \lim_{n \to \infty} -\frac{1}{1-1/n} \cdot \frac{1}{n^2} \cdot \frac{n^3}{2}  \\
\ln(L) &= \lim_{n \to \infty} -\frac{n}{2-2/n} \\
\end{align*}
So, $\ln(L) \to -\infty$ as $n \to \infty$, so $L \to e^{-\infty}=0$ .  Hence, we conclude that

\[
 \lim_{n \to \infty} \left(1-\frac{1}{n}\right)^{n^2} = 0.
\] 
The exponent now wins!

II. The above limits are relevant in the determination of the series $\sum_{k=1}^{\infty} a_k$ and $\sum_{k=1}^{\infty} b_k$.  

\begin{itemize} 
\item Note that since $\lim_{n \to \infty} \left(1-\frac{1}{n}\right)^n = \frac{1}{e} \neq 0$, we have that $\sum_{k=1}^{\infty} a_k = \sum_{k=1}^{\infty} \left(1-\frac{1}{k}\right)^k$ diverges by the divergence test.

\item Since $ \lim_{n \to \infty} \left(1-\frac{1}{n}\right)^{n^2} = 0$, we cannot conclude anything by the divergence test.  However, we may use the root test. Set $L = \lim_{n \to \infty} \sqrt[n]{\left(1-\frac{1}{n}\right)^{n^2} } $ and notice that

\begin{align*}
L &= \lim_{n \to \infty} \left[\left(1-\frac{1}{n}\right)^{n^2}\right]^{\frac{1}{n}}\\
&= \lim_{n \to \infty} \left(1-\frac{1}{n}\right)^{n^2\cdot\frac{1}{n}}\\
&= \lim_{n \to \infty} \left(1-\frac{1}{n}\right)^{n}\\
&= \frac{1}{e} \textrm{ by the earlier result. }
\end{align*}

Since $L = \frac{1}{e} < 1$, the series $\sum_{k=1}^{\infty} b_k = \sum_{k=1}^{\infty} \left(1-\frac{1}{k}\right)^{k^2}$ converges by the root test.
\end{itemize}

\end{freeResponse}
\end{problem}
\end{document}
