\documentclass[]{ximera}
%handout:  for handout version with no solutions or instructor notes
%handout,instructornotes:  for instructor version with just problems and notes, no solutions
%noinstructornotes:  shows only problem and solutions

%% handout
%% space
%% newpage
%% numbers
%% nooutcomes

%I added the commands here so that I would't have to keep looking them up
%\newcommand{\RR}{\mathbb R}
%\renewcommand{\d}{\,d}
%\newcommand{\dd}[2][]{\frac{d #1}{d #2}}
%\renewcommand{\l}{\ell}
%\newcommand{\ddx}{\frac{d}{dx}}
%\everymath{\displaystyle}
%\newcommand{\dfn}{\textbf}
%\newcommand{\eval}[1]{\bigg[ #1 \bigg]}

%\begin{image}
%\includegraphics[trim= 170 420 250 180]{Figure1.pdf}
%\end{image}

%add a ``.'' below when used in a specific directory.

%\usepackage{todonotes}
%\usepackage{mathtools} %% Required for wide table Curl and Greens
%\usepackage{cuted} %% Required for wide table Curl and Greens
\newcommand{\todo}{}

\usepackage{esint} % for \oiint
\ifxake%%https://math.meta.stackexchange.com/questions/9973/how-do-you-render-a-closed-surface-double-integral
\renewcommand{\oiint}{{\large\bigcirc}\kern-1.56em\iint}
\fi


\graphicspath{
  {./}
  {ximeraTutorial/}
  {basicPhilosophy/}
  {functionsOfSeveralVariables/}
  {normalVectors/}
  {lagrangeMultipliers/}
  {vectorFields/}
  {greensTheorem/}
  {shapeOfThingsToCome/}
  {dotProducts/}
  {partialDerivativesAndTheGradientVector/}
  {../productAndQuotientRules/exercises/}
  {../motionAndPathsInSpace/exercises/}
  {../normalVectors/exercisesParametricPlots/}
  {../continuityOfFunctionsOfSeveralVariables/exercises/}
  {../partialDerivativesAndTheGradientVector/exercises/}
  {../directionalDerivativeAndChainRule/exercises/}
  {../commonCoordinates/exercisesCylindricalCoordinates/}
  {../commonCoordinates/exercisesSphericalCoordinates/}
  {../greensTheorem/exercisesCurlAndLineIntegrals/}
  {../greensTheorem/exercisesDivergenceAndLineIntegrals/}
  {../shapeOfThingsToCome/exercisesDivergenceTheorem/}
  {../greensTheorem/}
  {../shapeOfThingsToCome/}
  {../separableDifferentialEquations/exercises/}
  {vectorFields/}
}

\newcommand{\mooculus}{\textsf{\textbf{MOOC}\textnormal{\textsf{ULUS}}}}

\usepackage{tkz-euclide}\usepackage{tikz}
\usepackage{tikz-cd}
\usetikzlibrary{arrows}
\tikzset{>=stealth,commutative diagrams/.cd,
  arrow style=tikz,diagrams={>=stealth}} %% cool arrow head
\tikzset{shorten <>/.style={ shorten >=#1, shorten <=#1 } } %% allows shorter vectors

\usetikzlibrary{backgrounds} %% for boxes around graphs
\usetikzlibrary{shapes,positioning}  %% Clouds and stars
\usetikzlibrary{matrix} %% for matrix
\usepgfplotslibrary{polar} %% for polar plots
\usepgfplotslibrary{fillbetween} %% to shade area between curves in TikZ
\usetkzobj{all}
\usepackage[makeroom]{cancel} %% for strike outs
%\usepackage{mathtools} %% for pretty underbrace % Breaks Ximera
%\usepackage{multicol}
\usepackage{pgffor} %% required for integral for loops



%% http://tex.stackexchange.com/questions/66490/drawing-a-tikz-arc-specifying-the-center
%% Draws beach ball
\tikzset{pics/carc/.style args={#1:#2:#3}{code={\draw[pic actions] (#1:#3) arc(#1:#2:#3);}}}



\usepackage{array}
\setlength{\extrarowheight}{+.1cm}
\newdimen\digitwidth
\settowidth\digitwidth{9}
\def\divrule#1#2{
\noalign{\moveright#1\digitwidth
\vbox{\hrule width#2\digitwidth}}}





\newcommand{\RR}{\mathbb R}
\newcommand{\R}{\mathbb R}
\newcommand{\N}{\mathbb N}
\newcommand{\Z}{\mathbb Z}

\newcommand{\sagemath}{\textsf{SageMath}}


%\renewcommand{\d}{\,d\!}
\renewcommand{\d}{\mathop{}\!d}
\newcommand{\dd}[2][]{\frac{\d #1}{\d #2}}
\newcommand{\pp}[2][]{\frac{\partial #1}{\partial #2}}
\renewcommand{\l}{\ell}
\newcommand{\ddx}{\frac{d}{\d x}}

\newcommand{\zeroOverZero}{\ensuremath{\boldsymbol{\tfrac{0}{0}}}}
\newcommand{\inftyOverInfty}{\ensuremath{\boldsymbol{\tfrac{\infty}{\infty}}}}
\newcommand{\zeroOverInfty}{\ensuremath{\boldsymbol{\tfrac{0}{\infty}}}}
\newcommand{\zeroTimesInfty}{\ensuremath{\small\boldsymbol{0\cdot \infty}}}
\newcommand{\inftyMinusInfty}{\ensuremath{\small\boldsymbol{\infty - \infty}}}
\newcommand{\oneToInfty}{\ensuremath{\boldsymbol{1^\infty}}}
\newcommand{\zeroToZero}{\ensuremath{\boldsymbol{0^0}}}
\newcommand{\inftyToZero}{\ensuremath{\boldsymbol{\infty^0}}}



\newcommand{\numOverZero}{\ensuremath{\boldsymbol{\tfrac{\#}{0}}}}
\newcommand{\dfn}{\textbf}
%\newcommand{\unit}{\,\mathrm}
\newcommand{\unit}{\mathop{}\!\mathrm}
\newcommand{\eval}[1]{\bigg[ #1 \bigg]}
\newcommand{\seq}[1]{\left( #1 \right)}
\renewcommand{\epsilon}{\varepsilon}
\renewcommand{\phi}{\varphi}


\renewcommand{\iff}{\Leftrightarrow}

\DeclareMathOperator{\arccot}{arccot}
\DeclareMathOperator{\arcsec}{arcsec}
\DeclareMathOperator{\arccsc}{arccsc}
\DeclareMathOperator{\si}{Si}
\DeclareMathOperator{\scal}{scal}
\DeclareMathOperator{\sign}{sign}


%% \newcommand{\tightoverset}[2]{% for arrow vec
%%   \mathop{#2}\limits^{\vbox to -.5ex{\kern-0.75ex\hbox{$#1$}\vss}}}
\newcommand{\arrowvec}[1]{{\overset{\rightharpoonup}{#1}}}
%\renewcommand{\vec}[1]{\arrowvec{\mathbf{#1}}}
\renewcommand{\vec}[1]{{\overset{\boldsymbol{\rightharpoonup}}{\mathbf{#1}}}\hspace{0in}}

\newcommand{\point}[1]{\left(#1\right)} %this allows \vector{ to be changed to \vector{ with a quick find and replace
\newcommand{\pt}[1]{\mathbf{#1}} %this allows \vec{ to be changed to \vec{ with a quick find and replace
\newcommand{\Lim}[2]{\lim_{\point{#1} \to \point{#2}}} %Bart, I changed this to point since I want to use it.  It runs through both of the exercise and exerciseE files in limits section, which is why it was in each document to start with.

\DeclareMathOperator{\proj}{\mathbf{proj}}
\newcommand{\veci}{{\boldsymbol{\hat{\imath}}}}
\newcommand{\vecj}{{\boldsymbol{\hat{\jmath}}}}
\newcommand{\veck}{{\boldsymbol{\hat{k}}}}
\newcommand{\vecl}{\vec{\boldsymbol{\l}}}
\newcommand{\uvec}[1]{\mathbf{\hat{#1}}}
\newcommand{\utan}{\mathbf{\hat{t}}}
\newcommand{\unormal}{\mathbf{\hat{n}}}
\newcommand{\ubinormal}{\mathbf{\hat{b}}}

\newcommand{\dotp}{\bullet}
\newcommand{\cross}{\boldsymbol\times}
\newcommand{\grad}{\boldsymbol\nabla}
\newcommand{\divergence}{\grad\dotp}
\newcommand{\curl}{\grad\cross}
%\DeclareMathOperator{\divergence}{divergence}
%\DeclareMathOperator{\curl}[1]{\grad\cross #1}
\newcommand{\lto}{\mathop{\longrightarrow\,}\limits}

\renewcommand{\bar}{\overline}

\colorlet{textColor}{black}
\colorlet{background}{white}
\colorlet{penColor}{blue!50!black} % Color of a curve in a plot
\colorlet{penColor2}{red!50!black}% Color of a curve in a plot
\colorlet{penColor3}{red!50!blue} % Color of a curve in a plot
\colorlet{penColor4}{green!50!black} % Color of a curve in a plot
\colorlet{penColor5}{orange!80!black} % Color of a curve in a plot
\colorlet{penColor6}{yellow!70!black} % Color of a curve in a plot
\colorlet{fill1}{penColor!20} % Color of fill in a plot
\colorlet{fill2}{penColor2!20} % Color of fill in a plot
\colorlet{fillp}{fill1} % Color of positive area
\colorlet{filln}{penColor2!20} % Color of negative area
\colorlet{fill3}{penColor3!20} % Fill
\colorlet{fill4}{penColor4!20} % Fill
\colorlet{fill5}{penColor5!20} % Fill
\colorlet{gridColor}{gray!50} % Color of grid in a plot

\newcommand{\surfaceColor}{violet}
\newcommand{\surfaceColorTwo}{redyellow}
\newcommand{\sliceColor}{greenyellow}




\pgfmathdeclarefunction{gauss}{2}{% gives gaussian
  \pgfmathparse{1/(#2*sqrt(2*pi))*exp(-((x-#1)^2)/(2*#2^2))}%
}


%%%%%%%%%%%%%
%% Vectors
%%%%%%%%%%%%%

%% Simple horiz vectors
\renewcommand{\vector}[1]{\left\langle #1\right\rangle}


%% %% Complex Horiz Vectors with angle brackets
%% \makeatletter
%% \renewcommand{\vector}[2][ , ]{\left\langle%
%%   \def\nextitem{\def\nextitem{#1}}%
%%   \@for \el:=#2\do{\nextitem\el}\right\rangle%
%% }
%% \makeatother

%% %% Vertical Vectors
%% \def\vector#1{\begin{bmatrix}\vecListA#1,,\end{bmatrix}}
%% \def\vecListA#1,{\if,#1,\else #1\cr \expandafter \vecListA \fi}

%%%%%%%%%%%%%
%% End of vectors
%%%%%%%%%%%%%

%\newcommand{\fullwidth}{}
%\newcommand{\normalwidth}{}



%% makes a snazzy t-chart for evaluating functions
%\newenvironment{tchart}{\rowcolors{2}{}{background!90!textColor}\array}{\endarray}

%%This is to help with formatting on future title pages.
\newenvironment{sectionOutcomes}{}{}



%% Flowchart stuff
%\tikzstyle{startstop} = [rectangle, rounded corners, minimum width=3cm, minimum height=1cm,text centered, draw=black]
%\tikzstyle{question} = [rectangle, minimum width=3cm, minimum height=1cm, text centered, draw=black]
%\tikzstyle{decision} = [trapezium, trapezium left angle=70, trapezium right angle=110, minimum width=3cm, minimum height=1cm, text centered, draw=black]
%\tikzstyle{question} = [rectangle, rounded corners, minimum width=3cm, minimum height=1cm,text centered, draw=black]
%\tikzstyle{process} = [rectangle, minimum width=3cm, minimum height=1cm, text centered, draw=black]
%\tikzstyle{decision} = [trapezium, trapezium left angle=70, trapezium right angle=110, minimum width=3cm, minimum height=1cm, text centered, draw=black]




\author{Jim Talamo}

\outcome{Find Taylor Polynomials.}
\outcome{Understand the relationship between the derivatives of a function and the coefficients of its Taylor Polynomial.}

\title[handout]{Radius and Interval of Convergence}

\begin{document}
\begin{abstract}
\end{abstract}
\maketitle

\vspace{-0.9in}

\section{Discussion Questions}

\begin{problem} 
Suppose that $f(x) = \sum_{k=0}^{\infty} a_k x^k$ and it is known that $\sum_{k=0}^{\infty} 2^k a_k $ converges but $ \sum_{k=0}^{\infty} (-1)^k 2^k a_k $ diverges.

\begin{itemize}
\item[I.] What is the radius of convergence of the power series?
%Add something that discusses why diverging at x=-2 does not mean that the ROC is less than 2
\item[II.] What is the interval of convergence of the power series?
\end{itemize}

\begin{freeResponse}
I. Since $\sum 2^k a_k = f(2)$ converges, the radius of convergence is at least $2$. Since $\sum (-1)^k 2^k a_k = f(-2)$, the radius of convergence is at most $2$. We conclude that the radius of convergence is exactly $2$.

II. Since the radius of convergence is $2$, with convergence at the right endpoint and divergence at the left endpoint, the interval of convergence is $(-2,2]$. 
\end{freeResponse}
\end{problem}


\begin{problem} 
Suppose that $f(x) = \sum_{k=0}^{\infty} a_k (x+3)^k$ and it is known that $\sum_{k=0}^{\infty} 5^k a_k$ converges and the series represented by $f(-10)$ diverges.

\begin{itemize}
\item[I.] Does the series represented by $f(5)$ converge, diverge, or is more information needed to determine this?
\item[II.] Does the series $\sum_{k=1}^{\infty} a_k$ converge, diverge, or is more information needed?
\item[III.] What is the smallest and largest possible radius and interval of convergence for the series represented by $f(x)$?
\end{itemize}


\begin{freeResponse}
I. The power series is centered at $x=-3$.  Since $5$ is farther from the center than $-10$, and $f(-10)$ diverges, we conclude that the series represented by $f(5)$ must diverge.

II. The series $\sum a_k$ is equal to $f(-2)$. Since $-2$ is closer to the center of series than $2$ and  $f(2) = \sum 5^k a_k$ converges, the series represented by $f(-2)$ converges.

III. That the series $\sum 5^k a_k = f(2)$ converges implies that the radius of convergence of the series is at least $5$. Since $f(-10)$ diverges, the radius of convergence is at most $7$.
\end{freeResponse}
\end{problem}


\section{Group Work}
%%%%%%%%%%%%%%%%%%%%%%%%%%%%%%%%


\begin{problem} 
Suppose that $f(x) = \sum_{k=1}^{\infty} \frac{k^2x^k}{3^k+1}$.  Determine whether the series represented by $f(2)$ converges or diverges.

%Note that they can sub in  x=2, or find the radius of convergence and use it to answer the question.

\begin{freeResponse}
The series represented by $f(2)$ is 
$$
\sum_{k=1}^{\infty} \frac{k^2 2^k}{3^k+1}.
$$
First note that $\frac{k^2 2^k}{3^k+1} \leq \frac{k^2 2^k}{3^k}$ for all $k$. To apply the Ratio Test to the series with terms given by this upper bounding sequence, we compute the limit
$$
\lim_{k\rightarrow \infty} \frac{(k+1)^2 2^{k+1}}{3^{k+1}} \cdot \frac{3^k}{k^2 2^k} = \lim_{k\rightarrow \infty} \frac{2(k+1)^2}{3k^2} = \frac{2}{3}.
$$
It follows that the series represented by $f(2)$ converges.
\end{freeResponse}
\end{problem}


\begin{problem} 
Find the radius and interval of convergence for the following power series:

\begin{tabular}{lll}
I.  $f(x) = \sum_{k=1}^{\infty} \frac{x^k}{k!}$ \qquad  \qquad II. $f(x) = \sum_{k=1}^{\infty} \frac{(-1)^kx^{2k}}{\sqrt{k}}$  \qquad  \qquad III. $f(x) = \sum_{k=1}^{\infty} k! x^k$
\end{tabular}

\begin{freeResponse}
I. We apply the Ratio Test to the series 
$$
\sum_{k=1}^{\infty} \left|\frac{x^k}{k!}\right| = \sum_{k=1}^{\infty} \frac{|x|^k}{k!}
$$
in order to test for absolute convergence. The relevant limit is 
$$
\lim_{k \rightarrow \infty} \frac{|x|^{k+1}}{(k+1)!} \cdot \frac{k!}{|x|^k} = \lim_{k \rightarrow \infty} \frac{|x|}{k+1} = 0
$$
for all $x$. The radius of convergence is therefore $\infty$ and the interval of convergence is $(-\infty,\infty)$.

II. We once again test for absolute convergence by considering the series 
$$
\sum_{k=1}^{\infty} \left|\frac{(-1)^kx^{2k}}{\sqrt{k}}\right| = \sum_{k=1}^{\infty} \frac{x^{2k}}{\sqrt{k}}.
$$
Applying the Ratio Test, we have
$$
\lim_{k\rightarrow \infty} \frac{x^{2(k+1)}}{\sqrt{k+1}} \frac{\sqrt{k}}{x^{2k}} = \lim_{k\rightarrow \infty} x^2 \frac{\sqrt{k}}{\sqrt{k+1}} = x^2.
$$
A sufficient condition for convergence is $x^2 < 1$, or $-1 < x < 1$. The radius of convergence is therefore $1$. To find the interval of convergence, we test the endpoints $f(1)$ and $f(-1)$ separately. The series represented by $f(1)$ is 
$$
\sum_{k=1}^{\infty} \frac{(-1)^k}{\sqrt{k}},
$$
which converges by the Alternating Series Test. Similarly, the series represented by $f(-1)$ converges by the Alternating Series Test. The interval of convergence is therefore $[-1,1]$. 

III. By comparing growth rates, we see that for any $x > 0$, 
$$
\lim_{k\rightarrow \infty} k! x^k  = \infty,
$$
and for any $x < 0$, the limit does not exist. It follows that the radius of convergence for this series is zero. The ``interval" of convergence is the singleton set $\{0\}$. 
\end{freeResponse}
\end{problem}


\end{document}
